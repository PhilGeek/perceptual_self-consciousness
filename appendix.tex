So consider Table~\ref{table:powers}. The powers in the \emph{logos} are specifically from the Puzzling Disanalogies and The Argument from Relatives. For the most part, the powers in the \emph{ergon} are from the dramatic prologue and the dramatic interludes that occur up to an including Socrates' investigation into the possibility of Critias' account of \emph{sōphrosunē}. Occurrences of the proper objects of the activities of these powers in the \emph{ergon} suffice for the relevant power to be on the list. A difficulty is posed by audition. The dialogue is Socrates' verbal report to an unnamed person of a conversation that he had primarily with Charmides and Critias. That means that the entire dialogue was heard but it would be both unreasonable and uninformative to include the entire dialogue as the object of the power of audition. But while not all implied activities of the relevant powers are included some are. (Thus, for example, included is Socrates report of Charmides look, 155c7–d1—he does not say that he saw the look, but then again, he did not smell or taste it either.) So there is no mechanical application of criteria, judgment is involved, and judgment may be reasonably queried. 

Another complication concerns the senses taken all together (\emph{peri pasōn tōn aisthēseon}, 167d7–10). Recall that there were two substantive interpretations either:
\begin{enumerate}[(a)]
	\item it is the rest of the senses besides vision and audition, or
	\item in a proleptic anticipation of Aristotle's \emph{koinē aisthēsis}, it is a special sense that takes itself and the ordinary senses as objects, but does not sense what the ordinary senses sense.
\end{enumerate}
So, in Table~\ref{table:powers}, the entries for the senses all together in the \emph{ergon} column will be divided between these two interpretations, marked by the labels (a) and (b). As it happens, there is at least no obvious occurence of the power as represented by the second interpretation in the \emph{ergon}. By contrast, touch, for example, a sense other than vision and audition, figures in the drama.

\begin{table}[htb!]
	\caption{Powers in the \emph{Logos} and \emph{Ergon}}
	\label{table:powers}
	\centering
	\begin{tabular}{lll}
		% \toprule
		\textbf{\emph{Dunamis}} & \textbf{\emph{Logos}} & \textbf{\emph{Ergon}}\\
		% \midrule
		             sight (\emph{opsis}) &              167c–d, 168d9–e2 &              153b7–8, 153d4–154a2, 154b8–10, 154c5–8, \\
		             hearing (\emph{akounē}) &              167d4–6, 168d3–8 &              153a6–b2,\\
		             the senses all together (\emph{peri pasōn tōn aisthēseon}) &              167d7–10 &              (a) (b) \textsc{N/A}\\
		             appetite (\emph{epithumia}) &              167e1–3 &              153a1–3\\
		             wish (\emph{boulēsis}) &              167e4–6 &              r6c3\\
		             love (\emph{eros}) &              167e7–9 &              153d2–4, 154a3–6, 154b8–d2, \\
		             fear (\emph{phobos}) &              167e10–168a2 &              r8c3\\
		             opinion (\emph{doxa}) &              168a3–5 &              r9c3\\
		            greater (\emph{meizon}) &             168b5–c3 &             r10c3\\
		            double (\emph{diplasion}) &             168c4–8 &             r11c3\\
		            more (\emph{pleon}) &             168c9 &             r12c3\\
		            heavier (\emph{baruteron}) &             168c9–10 &             r13c3\\
		            older (\emph{presbuteron}) &             168c10 &             r14c3\\
		            move (\emph{kinein}) &             168e9–10 &             r15c3\\
					burn (\emph{kaien}) &                168e10 &             r16c3\\
		% \bottomrule
	\end{tabular}
\end{table}