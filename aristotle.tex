%!TEX root = /Users/markelikalderon/Documents/Git/perceptual_self-consciousness/perceptual_self-consciousness.tex
\chapter{Perceiving that We See and Hear} % (fold)
\label{cha:perceiving}

\section{Introduction} % (fold)
\label{sec:introduction2}

We perceive that we see and hear (\emph{De Anima} 3.2 425b12). Aristotle does not establish this so much as he presupposes it. In so doing, he poses a how-possible question (on how-possible questions see \citealt{Cassam:2007lq}). Given that we see and hear, how is it possible that we do? How-possible questions are not posed out of the blue. If they are apt, then what occasions the question is a doubt or unclarity about that very possibility. What occasions Aristotle's how-possible question is the aporetic doubt about reflexive intentional powers developed in the \emph{Charmides}. Or so shall I argue. There are at least two major points of contact. 

First, in what \citet{caston02} dubs The Duplication Argument (\emph{De Anima} 3.2 425b13–15), Aristotle claims that a perceiving of a seeing would perceive as well the object seen. Why this should be so has puzzled some commentators (\citealt[435]{Hicks:1907uq}, \citealt[121–2]{Hamlyn:2002ys}, and \citealt[44–5]{Kosman:2014ab}). Such puzzlement is abated if this claim is recognized as an instance of transitivity (\citealt{McCabe:2007ss}, \citealt{McCabe:2007jb}) or transparency (\citealt{Tsouna:2022aa}) whose denial was itself crticized in the \emph{Charmides} (most explicitly in The Argument from Benefit, \emph{Charmides} 170a–175b, but doubts were raised as early as The Puzzling Disanalogies, \emph{Charmides} 167c–168a).

Second, Aristotle raises a difficulty based on the claim that seeing is of color or that which possesses color (\emph{De Anima} 425b17–9). Given this, if perceiving that we see is a seeing that we see, then the seeing that is seen must be colored. Again, why this should be so has puzzled some commentators (\citealt[122]{Hamlyn:2002ys}, \citealt[45–6]{Kosman:2014ab}). And, again, such puzzlement is abated if this claim is recognized as an instance of a key premise from The Argument from Relatives (\emph{Charmides} 168b2–169c2). Specifically, it is an instance of the claim that in order for an psychic power to apply to its object that object must have a certain nature of being, color or being colored in the case of sight (\emph{Charmides} 168d1–5).

There are two \emph{desiderata} on an adequate interpretation of the opening \emph{aporia} of \emph{De Anima} 3.2. 

The first, then, is that it should be read in light of the puzzle about how to coherently combine intentionality and reflexivity in the \emph{Charmides}. A number of commentators have observed the influence of the \emph{Charmides} on \emph{De Anima} 3.2 (). Though this happy consensus is tempered somewhat by different understandings that they display of the \emph{Charmides} and its influence on \emph{De Anima} 3.2. 

The second \emph{desiderata}, again supported by a number of commentators (), is that the opening \emph{aporia} of \emph{De Anima} 3.2 should cohere with the content of chapter 2, insofar as possible. \citet[121]{Hamlyn:2002ys} claims that the chapter is ``rambling''. Perhaps. But that is a conclusion we should only reach after having tried our best to discover a unifying thread or threads to the chapter and failed. And there is room for disagreement about whether there is a unifying thread or threads and what these might be.

The joint satisfaction of the two \emph{desiderata} thus does not constrain interpretation to uniqueness. In this chapter, I shall elaborate and defend an interpretation of the opening \emph{aporia} of \emph{De Anima} 3.2 that aspires to jointly satisfy these two \emph{desiderata}. Specifically, I shall elaborate and defend an Alexendrian interpretation, but only in the sense of defending a key claim of Alexander of Aphrodisias' interpretation in \emph{Quaestiones} 3.7.

% section introduction (end)

\section{The Opening Aporia} % (fold)
\label{sec:the_opening_aporia}

\emph{De Anima} 3.2 opens and closes with \emph{aporiai}. Whereas the opening \emph{aporia} concerns perceptual self-consciousness, the closing \emph{aporia} concerns apperceptive unity. Spe\-cifically the opening \emph{aporia} concerns perceiving that we see and hear, and the closing \emph{aporia} concerns perceiving that something is white and sweet. These \emph{aporiai} are linked in \emph{De Somno et Vigilia} 2 455a12ff by admitting of a common solution. Perhaps, the opening and closing \emph{aporiai} of \emph{De Anima} 3.2 are related as well.

\subsection{A Schematic Representation} % (fold)
\label{sub:a_schematic_representation}

It will be useful to have some overall sense of the opening \emph{aporia}. Towards that end, consider the following schematic representation of the aporetic reasoning:

\begin{enumerate}[(1)]
	\item Since we perceive that we see and hear, it must either be (a) by \emph{opsis} that one perceives that one sees or (b) some other [\emph{aisthēsis}]. (\emph{De Anima} 3.2 425b12–3, The How-Possible Question)
	\item But the same [\emph{aisthēsis}] will be of \emph{opsis} and the underlying color. So either there will be two [\emph{aisthēseis}] of the same, or the same [\emph{aisthēsis}] will be of itself. (\emph{De Anima} 3.2 4225b13–15, The Duplication Argument)
	\item Further, if the \emph{aisthēsis} which is of \emph{opsis} is different, either (a) this will go on to infinity or (b) some \emph{aisthēsis} will be of itself. So one should grant this of the first. (\emph{De Anima} 3.2 425b15-7, The Regress Argument)
	\item But there is a puzzle. For if to perceive by \emph{opsis} is to see, and if it is color or what has color that is seen, then if one is to see that which sees (\emph{to horōn}), that which sees (\emph{to horōn}) must have color in the primary way. (\emph{De Anima} 3.2 425b17–20, The Argument from Relatives)
	\item But it is clear that to perceive by \emph{opsis} is no single thing; for even when we do not see, it is by \emph{opsis} that we discriminate dark and light, though not in the same way. (\emph{De Anima} 3.2 425b20–2, The Argument from Darkness and Light)
	\item Further, that which sees (\emph{to horōn}) is in a sense colored. For each sense organ (\emph{aisthētērion}) receives the sense object without its matter; that is why even after the sense objects have gone, sensations and images remain in the sense organ (\emph{aisthētēríos}). (\emph{De Anima} 3.2 425b22–5, The Argument from Residual Imagery)
\end{enumerate} 

Though a close paraphrase of the opening \emph{aporia}, part of what makes the representation of the aporetic reasoning schematic is its leaving key terms whose interpretation is contested untranslated. 

Consider \emph{opsis}, a persistent thread with which the \emph{aporia} is woven, occurring as it does in steps (1)–(5) of the schematic representation. There are three available readings:
\begin{enumerate}[(1)]
	\item The Power Reading: \emph{Opsis} is the power of sight.
	\item The Activity Reading: \emph{Opsis} is the activity of sight, the characteristic exercise of that power (\citealt{caston02}).
	\item The Corporeal Reading: \emph{Opsis} is the corporeal instrument of sight, the sense organ, in this case, the eye (\citealt{Trubowitz:2025aa}).
\end{enumerate}
The Power Reading is perhaps the most natural, independent of context. In this regard, the English is like the Greek where, independent of context, ``sight'' is most naturally understood as a perceptual power, the power to see. But as \citet{caston02} argues, \emph{opsis} may, in addition, be used to designate, not a perceptual power, but the activity that is its exercise. Again, and in this regard, the English is like the Greek, ``sight'' may designate, less the power to see, but the seeing of something. But \emph{opsis} can also be understood as designating, not the power, nor its characteristic exercise, seeing, but the corporal instrument, or sense organ, involved in the power's exercise, the eyes (\citealt[]{Bonitz:1870aa}, \citealt[]{Hicks:1907uq}, and \citealt{Trubowitz:2025aa}). There is no corresponding understanding of ``sight'' in English. For what it is worth, none is represented in the OED.

% Since \emph{opsis} is a key term in the aporetic reasoning, opting for one or another or the alternative readings will bear, not only on how one sees that reasoning being specifically implemented, but on how that reasoning is structured as well.

% subsection a_schematic_representation (end)

\subsection{The How-Possible Question} % (fold)
\label{sub:the_how_possible_question}

\begin{quote}
	Since we perceive that we see and hear, it must either be (a) by \emph{opsis} that one perceives that one sees or (b) some other [\emph{aisthēsis}]. (\emph{De Anima} 3.2 425b12–3, The How-Possible Question)
\end{quote}

% subsection the_how_possible_question (end)

\subsection{The Duplication Argument} % (fold)
\label{sub:the_duplication_argument}

\begin{quote}
	But the same [\emph{aisthēsis}] will be of \emph{opsis} and the underlying color. So either there will be two [\emph{aisthēseis}] of the same, or the same [\emph{aisthēsis}] will be of itself. (\emph{De Anima} 3.2 4225b13–15, The Duplication Argument)
\end{quote}

% subsection the_duplication_argument (end)

\subsection{The Regress Argument} % (fold)
\label{sub:the_regress_argument}

\begin{quote}
	Further, if the \emph{aisthēsis} which is of \emph{opsis} is different, either (a) this will go on to infinity or (b) some \emph{aisthēsis} will be of itself. So one should grant this of the first. (\emph{De Anima} 3.2 425b15-7, The Regress Argument)
\end{quote}

% subsection the_regress_argument (end)

\subsection{The Argument from Relatives} % (fold)
\label{sub:the_argument_from_relatives}

\begin{quote}
	But there is a puzzle. For if to perceive by \emph{opsis} is to see, and if it is color or what has color that is seen, then if one is to see that which sees (\emph{to horōn}), that which sees (\emph{to horōn}) must have color in the primary way. (\emph{De Anima} 3.2 425b17–20, The Argument from Relatives)
\end{quote}

While \emph{to horōn} is unproblematically translated as ``that which sees'', its interpretation remains contestable. There are three readings:
\begin{enumerate}[(1)]
	\item The Perceiver Reading: That which sees (\emph{to horōn}) is the agent of the perceptual activity, the perceiver.
	\item The Power Reading: That which sees (\emph{to horōn}) is the power, sight, by which the perceiver sees what they do.
	\item The Corporeal Reading: That which sees (\emph{to horōn}) is the corporeal instrument, the sense organ, by which they perceiver exercises their power of sight, the eyes.
\end{enumerate}

% subsection the_argument_from_relatives (end)

\subsection{The Argument from Darkness and Light} % (fold)
\label{sub:the_argument_from_darkness}

\begin{quote}
	But it is clear that to perceive by \emph{opsis} is no single thing; for even when we do not see, it is by \emph{opsis} that we discriminate dark and light, though not in the same way. (\emph{De Anima} 3.2 425b20–2, The Argument from Darkness and Light)
\end{quote}

This is the first of two suggestions about how the \emph{aporia} raised by The Argument from Relatives may be mitigated. As we shall see, these two suggestions pull in different directions and a doubt may be raised about their consistency with one another.

To perceive by \emph{opsis} is no single thing. To see is to perceive by \emph{opsis}—to that extent the supposition that generates the \emph{aporia} is correct. But not every case of perceiving by \emph{opsis} is a seeing. That is the insight that is meant to at least partially resolve the \emph{aporia}.

Allow me to make an observation and a clarification.

The observation is that the power of sight has multiple manifestations. Seeing is the exercise of sight. But so is perceiving by \emph{opsis} that is not a seeing. In the unlovely vocabulary of modern metaphysics, sight is a multitrack power. 

The clarification concerns seeing. If seeing is not just perceiving by \emph{opsis}, then how are we to understand what seeing is? The possibility that is meant to at least partially resolve the \emph{aporia} is the possibility of perceiving by \emph{opsis} that is not seeing. But that is only as clear as our understanding of seeing that perceiving by \emph{opsis} is denied to be. So what is the relevant notion of seeing? Seeing is perceiving color in light. The \emph{aporia} turned on the fact that seeing is of color or of colored things. And color is visible in light (\emph{De Anima} 2.7 418a26–b3). Color is the power to move the actually transparent, and light, understood as a state of illumination, is the actually transparent.

The proposed resolution of the \emph{aporia} should now be clear, at least in outline (\emph{pace} \citealt[122]{Hamlyn:2002ys}). If we perceive by \emph{opsis} that which sees, and perceiving by \emph{opsis} is seeing, and seeing is of color or colored things, then that which sees must itself be colored. But, if we can perceive by \emph{opsis} in a way that is not seeing, and if, more specifically, we can perceive by \emph{opsis} that which sees without seeing it, then no commitment to it being colored is incurred.

The difficulty posed by The Argument from Darkness and Light is less understanding the general outline of the proposed resolution, than understanding the example that is meant to establish that there is a perceiving by \emph{opsis} that is not seeing—that we discriminate dark and light, though not in the same way.

The most straightforward example establishing this would be a case where one perceives by \emph{opsis} in the absence of seeing. But, importantly, that is not required. Perceiving by \emph{opsis} may be distinct from seeing and yet occur along with it. (Thus, Themistius includes the supplement \emph{kai hotan horōmen} after \emph{mē horōmen}, \emph{In libros De Anima paraphrasis} 83.23, see \citealt[105, 181 n3]{Todd:1996aa} and \citealt[238]{Browne:1986aa}) Let us consider whether Aristotle's example is better understood in the first or second way.

The first reading admits of two variants that differ in their understanding of the absence of seeing.

Consider, then, the initial variant of the first reading. A transparent body such as air or water is actually transparent due to the presence of a fiery substance. When the fiery substance is absent, darkness supervenes. If color is the power to move what is actually transparent, then in complete darkness, there is nothing for the chromatic powers to move and are thus inactive. The perceiver is thus incapable of seeing the colors of the the scene before them when shrouded in darkness. And yet, darkness is something we visually experience. We thus perceive the darkness by \emph{opsis}. Thus if seeing is of color or of colored things, then the perceiver's visual experience of darkness is not a seeing—it is at least not the seeing of the inactive colors of the scene before them. (Aquinas offers such an interpretation, \emph{Sentencia libri De Anima}, 73–97.)

Perceiving darkness may be a straightforward case of perceiving by \emph{opsis} that is not seeing the colored scene before one (though as we shall see a doubt may be raised), but what about perceiving light? In the case of total darkness, it was the absence of light that prevented the colors from acting upon what is actually transparent, and so prevented the perceiver from seeing the colored scene before them. But in the present case light is not absent but present and perceived. So how is the perception of light an obstacle to seeing the colored scene before one? (Aquinas remains silent on this question.)

Later in the chapter (\emph{De Anima} 3.2 426b1–2), in the course of arguing that perception is a kind of ratio, Aristotle claims that an excess of brightness or darkness destroys sight (by destroying the ratio that enables sight). Perhaps, perceiving light by \emph{opsis} that is not seeing is meant to be an experience of excessive brightness that obscures the colored scene before one. (Compare Plato's description of the dazzling experience of brightness, \emph{lampros}, or brilliance, \emph{stilbos}, \emph{Timaeus} 68a1–6, see \citealt[72–7]{Kalderon:2022kl}.) A flashlight shown in a person's face can obscure the person weilding it. In such a case, one is blinded by the light.

If that is how we are to understand lightness perception, then one can undertand the qualification, ``though not on the same way'', as follows. Perceiving excessive darkness by \emph{opsis} is not seeing the colored scene in the same way that perceiving excessive light by \emph{opsis} is. The obstacle to seeing is different in each case. They are different ways of destroying the ratio that constitutes the power to see color (see \citealt[275]{Ross:1961uq}, for criticism).

On the initial variant of first reading, sight is only partially incapacitated. Sight is incapacitated in that, under conditions of excessive darkness or brightness, one cannot see the colors arrayed in the scene before one. But sight remains in operation. One visually experiences the darkness. And the visual experience of brilliance is dazzling. So perceiving by \emph{opsis} that is not the seeing of color remains the exercise of the power of sight. Otherwise we would lose the contrast with the different ways that sight is incapacitated when asleep, or having fainted, or rendered unconscious by having the veins in one's neck compressed (\emph{De Somno et Vigilia} 455b). When asleep or unconscious one does no perceive by \emph{opsis} the excessive darkness or brightness that surrounds one.

Can the distinction between seeing color and perceiving by \emph{opsis} be sustained on the initial variant of the first reading? A doubt may be raised given the phenomenology of our experience of excessive darkness and brightness. Darkness has a color that is revealed in our experience of it. Darkness is black. Contrast our experience of silence. Whereas out experience of the absence of light has a chromatic quality, our experience of the absence of sound lacks an audible quality. (What pitch is it? What timbre does it possess? For discussion see \citealt{Sorensen:2008kx,Sorensen:2009aa}.) And arguably, so too for excessive brightness. Timaeus claims that brilliance causes all sorts of colors to appear (\emph{Timaeus} 68a5–6). But these were meant to be cases of perceiving by \emph{opsis} that did not involve the visual awareness of color.

The second variant of the first reading addresses this issue by sidestepping it. On the first reading, the perception of darkness and light is meant to be a case of perceiving by \emph{opsis} in the absence of seeing color. The first variant sought to establish this by excessive darkness and excessive brightness obscuring the colored scene set before the perceiver, so that no element of that scene was in fact perceived. The second variant allows that elements of the scene be perceptually discriminated in the way that the initial variant precludes. So consider looking at a scene in low light. We can perceive the relative brightness of objects in a sufficiently dark room, even if we cannot distinguish their hue. \emph{La nuit, tous les chats sont gris}. On this variant, we perceive dark and light but not the chromatic hues. We thus have an example of perceiving relative brightness by \emph{opsis} without seeing color, where seeing color is understood as, specifically, seeing chromatic hues. That darkness is black is no objection. Nor is the multi-colored character of the dazzling experience of excessive brightness. Moreover, the qualification ``though not in the same way'' may be read, not as contrasting the perception of darkness from the perception of light, but by contrasting each with seeing color as \citet[275]{Ross:1961uq} recommends.

In this way is it an advance over the initial variant. And yet difficulties remain. The present reading assimilaties discriminating darkness and light to discriminating darkness from light. But that is not what the text says. Moreover, the idea that black, white, and gray are achromatic colors that contrast with the chromatic hues such as red and green seems alien to a color scheme that takes the colors to be mixtures of black and white understood as light and dark.

Consider, then, the second reading. On the second reading, the example is not meant to be a case of perception by \emph{opsis} in the absence of seeing color. Rather, perception by \emph{opsis}, though distinct from seeing color, may occur along with it. The objects of perception by \emph{opsis} are darkness and light. 

Begin with the perception of light. In a transparent medium illuminated by the presence of the fiery substance, the chromatic powers of the scene move what is actually transparent and so affect the organ of sight, the eyes, thus enabling the seeing of that scene. In such a case, there clearly may be seeing of color or colored things. In seeing the calico cat in daylight, say, I see its tripartite color. But I also visually experience the daylight. But perhaps I do not see the daylight despite seeing the colors seen through it. Light may be colored but only in a sense that contrasts with the primary sense in which the cat is tricolored. That light is only in a sense colored is insufficient for the perception of light to count as seeing color. After all, light could not be colored in the primary sense. In the primary sense, color is the power to move what is actually transparent. Light is the actual transparent. To be colored, light would have to have the power to move itself in a way that Aristotle deems impossible. How, then, is light presented in the daylit scene, if not by being colored in the primary sense? It is by being moved by chromatic powers in a manner that affects sight that light is perceived by \emph{opsis}.

Consider then the perception of darkness. Recall that on the present reading perception by \emph{opsis} is distinct from seeing color, though it may occur along with it. Unlike the first reading, it need not be a case of perception by \emph{opsis} in the absence of seeing. It was that thought that prompted the idea that darkness was complete in a manner incompatible with seeing. But having abandoned the first reading, there is no longer the pressure to understand darkness as total darkness. So consider the calico cat. It is sitting next to a stone wall, perfectly still, eyes closed, enjoying the sunlight. Since it is late afternoon, the cat casts an elongated shadow. Darkness does not so much as obscure this scene as it is a perceived element of it. The shadow is dark. But it is not completely dark. The rough texture of the stone wall may still be discerned despite the cat's shadow being cast upon it. Its shadow is not so much the complete absence of light as the diminution of light relative to the overall illumination. After all, shadows may persist despite a degree of light pollution from ambient reflection. What is within the cat's cast shadow is illuminated by significantly less light than the overall scene. Shadows may obscure a scene—one may hide in the shadows—but only to a degree. Darker shadows may obscure the color of what lies within but not so their paler siblings. If light is revealed to sight by the colors seen through it, dark is revealed to sight by the relative resistance it offers to seeing color, a resistance which is frequently less than total.

On this reading, the perception by \emph{opsis} of darkness and light is like, if not exactly like, the visual perception of the common sensibles. There is no perceiving the shape of a thing without seeing the color that pervades its surface. But color and not shape is the proper object of vision. Similarly, there is no perceiving the light that pervades a scene withut seeing the colored objects it illuminates. But color and not light is the proper object of vision. Light may be perceptible to one sense alone, and so not itself a common sensible, but it is not perceptible in itself, and so is not a proper sensible. Light does not contain within itself the power of its own visibility. Light is perceptible only insofar as it enables the colors to be perceptible by moving it. Moreover, sight is for the sake of seeing colors, not perceiving light. Light, like the common sensibles, is only ever presented to sight insofar as color is seen, but unlike the common sensibles, it is sensed by no other sense. And sight is for the sake of neither.

The qualification ``though not in the same way'' must be read, not as contrasting the perception of darkness from the perception of light, but by contrasting each with seeing color as \citet[275]{Ross:1961uq} recommends.

Let us return to the \emph{aporia} set out in The Argument from Relatives. If we perceive by \emph{opsis} that which sees, and perceiving by \emph{opsis} is seeing, and seeing is of color or colored things, then that which sees must itself be colored. But, if we can perceive by \emph{opsis} in a way that is not seeing, and if, more specifically, we can perceive by \emph{opsis} that which sees without seeing it, then no commitment to it being colored is incurred. The Argument from Darkness and Light, if succesful on some construal, establishes at most that one can perceive by \emph{opsis} without seeing a colored thing. But that would only help resolve the puzzlement if, more specifically, one can perceive by \emph{opsis} that which sees without seeing it. But nothing so far has been said in support of that claim. The Argument from Darkness and Light is at best an incomplete sketch of a resolution of the \emph{aporia}.

% subsection the_argument_from_darkness (end)

\subsection{The Argument from Residual Imagery} % (fold)
\label{sub:the_argument_from_residual_imagery}

\begin{quote}
	Further, that which sees (\emph{to horōn}) is in a sense colored. For each sense organ (\emph{aisthētērion}) receives the sense object without its matter; that is why even after the sense objects have gone, sensations and images remain in the sense organ (\emph{aisthētēríos}). (\emph{De Anima} 3.2 425b22–5, The Argument from Residual Imagery)
\end{quote}

% subsection the_argument_from_residual_imagery (end)

% section the_opening_aporia (end)

\section{Apperceptive Unity} % (fold)
\label{sec:apperceptive_unity}

% section apperceptive_unity (end)

\section{\emph{De Somno et Vigilia}} % (fold)
\label{sec:_emph_de_somno_et_vigilia}

% section _emph_de_somno_et_vigilia (end)

\section{Coda} % (fold)
\label{sec:coda2}

% section coda (end)

% Chapter perceiving (end)