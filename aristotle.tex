%!TEX root = /Users/markelikalderon/Documents/Git/perceptual_self-consciousness/perceptual_self-consciousness.tex
\chapter{Perceiving that We See and Hear} % (fold)
\label{cha:perceiving}

\section{Introduction} % (fold)
\label{sec:introduction}

We perceive that we see and hear (\emph{De Anima} 3.2 425b12). Aristotle does not establish this so much as he presupposes it. In so doing, he poses a how possible question. Given that we see and hear, how is it possible that we do? How possible questions are not posed out of the blue. If they are apt, then what occasions the question is a doubt about that very possibility. What occasions Aristotle's how possible question is the aporetic doubt about reflexive intentional powers developed in the \emph{Charmides}. Or so shall I argue. There are at least two major points of contact. 

First, in what \citet{caston02} dubs The Duplication Argument (\emph{De Anima} 3.2 425b13–15), Aristotle claims that a perceiving of a seeing would perceive as well the object seen. Why this should be so has puzzled some commentators (\citealt[435]{Hicks:1907uq}, \citealt[121–2]{Hamlyn:1961ys}). Such puzzlement is abated if this claim is recognized as an instance of transitivity or transparency whose denial was itself crticized in the \emph{Charmides} (most explicitly in The Argument from Benefit, \emph{Charmides} 170a–175b).

Second, Aristotle raises a difficulty based on the claim that seeing is of color or that which possesses color. Given this, if perceiving that we see is a seeing that we see, then the seeing that we see must be colored. Again, why this should be so has puzzled some commentators. And, again, such puzzlement is abated if this claim is recognized as an instance of a key premise from The Argument from Relatives. Specifically, it is an instance of the claim that in order for a psychic power to apply to its intentional object that object must have a certain nature of being.


% section introduction (end)

\section{The Opening Aporia} % (fold)
\label{sec:the_opening_aporia}



% section the_opening_aporia (end)

% Chapter perceiving (end)