%!TEX root = /Users/markelikalderon/Documents/Git/perceptual_self-consciousness/perceptual_self-consciousness.tex
\chapter{Perceiving that We See and Hear} % (fold)
\label{cha:perceiving}

\section{Introduction} % (fold)
\label{sec:introduction2}

We perceive that we see and hear (\emph{De Anima} 3.2 425b12). Aristotle does not establish this so much as he presupposes it. In so doing, he poses a how-possible question (on how-possible questions see \citealt{Cassam:2007lq}). Given that we see and hear, how is it possible that we do? How-possible questions are not posed out of the blue. If they are apt, then what occasions the question is a doubt or unclarity about that very possibility. What occasions Aristotle's how-possible question is the aporetic doubt about reflexive intentional powers developed in the \emph{Charmides}. Or so shall I argue. There are at least three major points of contact. 

First, like Socrates, Aristotle maintains that the objects of perception are transcendent:
\begin{quote}
	For perception is surely not the perception of itself, but there is something beyond the perception, which must be prior to the perception. For that which moves is prior in nature to that which is moved. (Aristotle, \emph{Metaphysica} {\sbl Γ} 1010b 35-6)
\end{quote}
What is perceived is distinct from the perceiving of it. In Sartre's terminology, the intentional object is transcendent in the sense that it goes beyond the conscious act directed upon it. In the case of perception this follows from the causal role of the object of perception which acts upon the perceiver's sense organ and so must be distinct and prior to perception. Aristotle makes explicit the tension between intentionality and reflexivity first raised in the \emph{Charmides}. If the object of perception is transcendent, then perception cannot be of itself. Does Aristotle reject reflexivity then? A qualification complicates matters:
\begin{quote}
	It seems that knowledge and perception and opinion and understanding have always something else as their object, but themselves only by the side (\emph{en parergō}). (Aristotle, \emph{Metaphysica} {\sbl Μ} 1074b 35–6)
\end{quote}
Aristotle thus allows intentional activities, while primarily of other things, to be of themselves as well, though not primarily, but only (\emph{en parergō}).

Second, in what \citet{caston02} dubs The Duplication Argument (\emph{De Anima} 3.2 425b13–15), Aristotle claims that a perceiving of a seeing would perceive as well the object seen. Why this should be so has puzzled some commentators (\citealt[435]{Hicks:1907uq}, \citealt[121–2]{Hamlyn:2002ys}, and \citealt[44–5]{Kosman:2014ab}). Such puzzlement is abated if this claim is recognized as an instance of transitivity (\citealt{McCabe:2007ss}, \citealt{McCabe:2007jb}) or transparency (\citealt{Tsouna:2022aa}) whose denial was itself crticized in the \emph{Charmides} (most explicitly in The Argument from Benefit, \emph{Charmides} 170a–175b, but doubts were raised as early as The Puzzling Disanalogies, \emph{Charmides} 167c–168a).

Third, Aristotle raises a difficulty based on the claim that seeing is of color or that which possesses color (\emph{De Anima} 425b17–9). Given this, if perceiving that we see is a seeing that we see, then the seeing that is seen must be colored. Again, why this should be so has puzzled some commentators (\citealt[122]{Hamlyn:2002ys}, \citealt[45–6]{Kosman:2014ab}). And, again, such puzzlement is abated if this claim is recognized as an instance of a key premise from The Argument from Relatives (\emph{Charmides} 168b2–169c2). Specifically, it is an instance of the claim that in order for an psychic power to apply to its object that object must have a certain nature of being, color or being colored in the case of sight (\emph{Charmides} 168d1–5).

There are two \emph{desiderata} on an adequate interpretation of the opening \emph{aporia} of \emph{De Anima} 3.2. 

The first, then, is that it should be read in light of the puzzle about how to coherently combine intentionality and reflexivity in the \emph{Charmides}. A number of commentators have observed the influence of the \emph{Charmides} on \emph{De Anima} 3.2 (). Though this happy consensus is tempered somewhat by different understandings that they display of the \emph{Charmides} and its influence on \emph{De Anima} 3.2. 

The second \emph{desiderata}, again supported by a number of commentators (), is that the opening \emph{aporia} of \emph{De Anima} 3.2 should cohere with the content of chapter 2, insofar as possible. \citet[121]{Hamlyn:2002ys} claims that the chapter is ``rambling''. Perhaps. But that is a conclusion we should only reach after having tried our best to discover a unifying thread or threads to the chapter and failed. And there is room for disagreement about whether there is a unifying thread or threads and what these might be.

The joint satisfaction of the two \emph{desiderata} thus does not constrain interpretation to uniqueness. In this chapter, I shall elaborate and defend an interpretation of the opening \emph{aporia} of \emph{De Anima} 3.2 that aspires to jointly satisfy these two \emph{desiderata}. Specifically, I shall elaborate and defend an Alexendrian interpretation, but only in the sense of defending a key claim of Alexander of Aphrodisias' interpretation in \emph{Quaestiones} 3.7.

% section introduction (end)

\section{The Opening Aporia} % (fold)
\label{sec:the_opening_aporia}

\emph{De Anima} 3.2 opens and closes with \emph{aporiai}. Whereas the opening \emph{aporia} concerns perceptual self-consciousness, the closing \emph{aporia} concerns apperceptive unity. Spe\-cifically the opening \emph{aporia} concerns perceiving that we see and hear, and the closing \emph{aporia} concerns perceiving that something is white and sweet. These \emph{aporiai} are linked in \emph{De Somno et Vigilia} 2 455a12ff by admitting of a common solution. Perhaps, the opening and closing \emph{aporiai} of \emph{De Anima} 3.2 are related as well.

\subsection{A Schematic Representation} % (fold)
\label{sub:a_schematic_representation}

It will be useful to have some overall sense of the opening \emph{aporia}. Towards that end, consider the following schematic representation of the aporetic reasoning:

\begin{enumerate}[(1)]
	\item \textsc{The How-Possible Question}: Since we perceive that we see and hear, it must either be (a) by \emph{opsis} that one perceives that one sees or (b) some other [\emph{aisthēsis}]. (\emph{De Anima} 3.2 425b12–3)
	\item \textsc{The Duplication Argument}: But the same [\emph{aisthēsis}] will be of \emph{opsis} and the underlying color. So either there will be two [\emph{aisthēseis}] of the same, or the same [\emph{aisthēsis}] will be of itself. (\emph{De Anima} 3.2 4225b13–15)
	\item \textsc{The Regress Argument}: Further, if the \emph{aisthēsis} which is of \emph{opsis} is different, either (a) this will go on to infinity or (b) some \emph{aisthēsis} will be of itself. So one should grant this of the first. (\emph{De Anima} 3.2 425b15-7)
	\item \textsc{The Argument from Relatives}: But there is a puzzle. For if to perceive by \emph{opsis} is to see, and if it is color or what has color that is seen, then if one is to see that which sees (\emph{to horōn}), that which sees (\emph{to horōn}) must have color in the primary way. (\emph{De Anima} 3.2 425b17–20)
	\item \textsc{The Argument from Darkness and Light}: But it is clear that to perceive by \emph{opsis} is no single thing; for even when we do not see, it is by \emph{opsis} that we discriminate dark and light, though not in the same way. (\emph{De Anima} 3.2 425b20–2)
	\item \textsc{The Argument from Residual Imagery}: Further, that which sees (\emph{to horōn}) is in a sense colored. For each sense organ (\emph{aisthētērion}) receives the sense object without its matter; that is why even after the sense objects have gone, sensations and images remain in the sense organ (\emph{aisthētēríos}). (\emph{De Anima} 3.2 425b22–5)
\end{enumerate} 

Though a close paraphrase of the opening \emph{aporia}, part of what makes the representation of the aporetic reasoning schematic is its leaving key terms whose interpretation is contested untranslated. 

Consider \emph{opsis}, a persistent thread with which the \emph{aporia} is woven, occurring as it does in steps (1)–(5) of the schematic representation. There are three available readings:
\begin{enumerate}[(1)]
	\item \textsc{The Power Reading}: \emph{Opsis} is the power of sight.
	\item \textsc{The Activity Reading}: \emph{Opsis} is the activity of sight, the characteristic exercise of that power (\citealt{caston02}).
	\item \textsc{The Corporeal Reading}: \emph{Opsis} is the corporeal instrument of sight, the sense organ, in this case, the eyes (\citealt{Trubowitz:2025aa}).
\end{enumerate}
The Power Reading is perhaps the most natural, independent of context. In this regard, the English is like the Greek where, independent of context, ``sight'' is most naturally understood as a perceptual power, the power to see. But as \citet{caston02} argues, \emph{opsis} may, in addition, be used to designate, not a perceptual power, but the activity that is its exercise. Again, and in this regard, the English is like the Greek, ``sight'' may designate, less the power to see, but the seeing of something. But \emph{opsis} can also be understood as designating, not the power, nor its characteristic exercise, seeing, but the corporal instrument, or sense organ, involved in the power's exercise, the eyes (\citealt[]{Bonitz:1870aa}, \citealt[]{Hicks:1907uq}, and \citealt{Trubowitz:2025aa}). There is no corresponding understanding of ``sight'' in English. For what it is worth, none is represented in the OED.

Importantly, \emph{opsis} might not admit of a univocal interpretation throughout our passage. Given its context, one occurrence might only admit of the power reading, say, while another the activity or corporeal reading. So our interpretative choices, here, potentially affects our understanding of the overall structure of the aporetic reasoning.

While the overall structure of the passage remains to be determined, a traditional understanding is at least initially plausible (see Alexander of Aphrodisias \emph{Quaestiones} 3.7; for doubts, see \citealt{Kosman:2014ab} and \citealt{Osborne:1983le}).

% Since \emph{opsis} is a key term in the aporetic reasoning, opting for one or another or the alternative readings will bear, not only on how one sees that reasoning being specifically implemented, but on how that reasoning is structured as well.

% subsection a_schematic_representation (end)

\subsection{The How-Possible Question} % (fold)
\label{sub:the_how_possible_question}

\begin{quote}
	\textsc{The How-Possible Question}: Since we perceive that we see and hear, it must either be (a) by \emph{opsis} that one perceives that one sees or (b) some other [\emph{aisthēsis}]. (\emph{De Anima} 3.2 425b12–3)
\end{quote}

We perceive that we see and hear. Aristotle seems to accept this claim. About this claim, Aristotle poses a how-possible question: How is it possible that we perceive that we see and hear? Two alternative answers are offered, it is either by \emph{opsis} or some other \emph{aisthēsis} that we perceive that we see and hear. How-possible questions are not posed out of the blue. If they are apt, then what occasions the question is a doubt or unclarity about that very possibility. (Contrast the persistent pestering questioning of a toddler continually asking ``Why?'') It is the \emph{aporia} about reflexive intentional powers developed in the \emph{Charmides} that occasions and renders apt Aristotle's question.

About the claim that gives rise to the how-possible question—the we perceive that we see and hear—there are questions about its scope, about how to understand perceiving, as well as whether the claim extends beyond seeing and hearing to each of the senses.

Begin with the scope of the claim. We perceive that we see and hear. Always? Or is it only some seeings and hearings that we perceive? As \citet[]{McCabe:2007jb} observes, the present tense of the verb does not settle the matter. If like Leibniz one distinguishes perception and apperception (\emph{Principes de la Nature et de la Grâce} §4, \emph{La Monadologie} §14), then one would reject that we always perceive that we see and hear:
\begin{quote}
	Thus it is well to make the distinction between \emph{perception}, which is the inner state of the Monad representing outer things, and \emph{apperception}, which is \emph{consciousness} or the reflective knowledge of this inner state, and which is not given to all souls nor to the same soul at all times. (\emph{Principes de la Nature et de la Grâce} §4, \citealt[411]{Latta:1898aa})
\end{quote}
So we may perceive that we see and hear, but only sometimes. By contrast, Sartre, standing at the culmination of a tradition that includes Augustine, Peter John Olivi, and Descartes, doubts the very coherence of the notion of an unconscious perception and so would maintain, instead, that we always perceive that we see and hear. These alternative understandings bear on the modal status of the claim. If we always perceive that we see and hear, this can encourage undestanding the claim as necessary as Sartre does. If we only sometimes perceive that we see and hear, this, by contrast, can encourage understanding the claim as a contingent empirical claim. However, given the puzzle about reflexive powers as elaborated in \emph{Charmides}, the distinction matters little. If perceptual self-consciousness is reflexive, then that it always happens may be puzzling, but it remains so if it only sometimes happens.

% subsection the_how_possible_question (end)

\subsection{The Duplication Argument} % (fold)
\label{sub:the_duplication_argument}

\begin{quote}
	\textsc{The Duplication Argument}: But the same [\emph{aisthēsis}] will be of \emph{opsis} and the underlying color. So either there will be two [\emph{aisthēseis}] of the same, or the same [\emph{aisthēsis}] will be of itself. (\emph{De Anima} 3.2 4225b13–15)
\end{quote}

% subsection the_duplication_argument (end)

\subsection{The Regress Argument} % (fold)
\label{sub:the_regress_argument}

\begin{quote}
	\textsc{The Regress Argument}: Further, if the \emph{aisthēsis} which is of \emph{opsis} is different, either (a) this will go on to infinity or (b) some \emph{aisthēsis} will be of itself. So one should grant this of the first. (\emph{De Anima} 3.2 425b15-7)
\end{quote}



% subsection the_regress_argument (end)

\subsection{The Argument from Relatives} % (fold)
\label{sub:the_argument_from_relatives}

\begin{quote}
	\textsc{The Argument from Relatives}: But there is a puzzle. For if to perceive by \emph{opsis} is to see, and if it is color or what has color that is seen, then if one is to see that which sees (\emph{to horōn}), that which sees (\emph{to horōn}) must have color in the primary way. (\emph{De Anima} 3.2 425b17–20)
\end{quote}

The label for this passage, ``The Argument from Relatives'', is contentious but ultimately justified given the origin of this line of thought in the \emph{Charmides} (168b2–169c2). That argument cast doubt on whether intentional psychic powers could be reflexive, where reflexive powers apply to themselves or or at least their exercise. Perception is a psychic power characteristic of animals. Moreover, perception is of something. Perception is thus an intentional power since its exercise takes an object. In seeing, there is something seen. In hearing, there is something heard. In the \emph{Charmides}, The Argument from Relatives proceeds from a general premise that in order for a power to be of something that thing must have a certain nature or being (\emph{ousia}). So if an intentional power applies either to itself or to its exercise, these must themselves have the relevant nature or being (\emph{Charmides} 168d1–5). What is seen by sight is colored (\emph{Charmides} 168d9–e2). So if in seeing the colored scene one sees as well the seeing of it, this perceptual activity, seeing, must itself be colored.

Compare now our passage. The Regress Argument that precedes it raises the possibility that some \emph{aisthēsis} will be of itself. \emph{Aisthēsis} is the exercise of an intentional psychic power. So the possibility raised is that some \emph{aisthēsis} are reflexive. But this is what gives rise to the \emph{aporia}. This is the only element of \emph{De Anima} 3.2 425b12–25 that is explicitly described as aporetic. So we are confronting a puzzle about whether intentional psychic powers might also be reflexive, just like in the \emph{Charmides} (168b2–169c2). Moreover, the puzzle that Aristotle describes turns on the claim that seeing is of color or colored things. But this is an instance of the generality that drove the puzzle about how to coherently combine intentionality and reflexivity in the \emph{Charmides}. In order for a power to be of something it must have a certain nature or being. In the case of sight, the relevant nature or being is color or being colored. Seeing is of color or colored things because only such things have the nature or being required for sight to be of them. Furthermore, the argument proceeds in the way it did in the \emph{Charmides}. If seeing is of color or colored things, then reflexive seeings must themselves be colored.

This last parallel is formal. It abstract character obscures an important linguistic difference. Unlike, in the \emph{Charmides}, the thought is that to perceive by \emph{opsis} seeing involves seeing that which sees (\emph{to horōn}). Whether and to what extent Aristotle's puzzlement departs from the Socratic puzzlement described in the \emph{Charmides} depends, in part, in how (\emph{to horōn}) is understood.

While \emph{to horōn} is unproblematically translated as ``that which sees'', its interpretation remains contestable. There are three readings:
\begin{enumerate}[(1)]
	\item \textsc{The Perceiver Reading}: That which sees (\emph{to horōn}) is the agent of the perceptual activity, the perceiver, or more specifically, the seer.
	\item \textsc{The Power Reading}: That which sees (\emph{to horōn}) is the power, sight, by which the perceiver sees what they do.
	\item \textsc{The Activity Reading}: That which sees (\emph{to horōn}) is the activity of sight, the seeing
	\item \textsc{The Corporeal Reading}: That which sees (\emph{to horōn}) is the corporeal instrument, the sense organ, the eyes, by which the perceiver exercises their power of sight in seeing what they do.
\end{enumerate}

% subsection the_argument_from_relatives (end)

\subsection{The Argument from Darkness and Light} % (fold)
\label{sub:the_argument_from_darkness}

\begin{quote}
	\textsc{The Argument from Darkness and Light}: But it is clear that to perceive by \emph{opsis} is no single thing; for even when we do not see, it is by \emph{opsis} that we discriminate dark and light, though not in the same way. (\emph{De Anima} 3.2 425b20–2)
\end{quote}

This is the first of two suggestions about how the \emph{aporia} raised by The Argument from Relatives might be resolved. As we shall see, these two suggestions pull in different directions and a doubt may be raised about their coherence with one another.

% To perceive by \emph{opsis} is no single thing. Notice what Aristotle does not say. He does not say that perceiving by sight is spoken of in many ways.

To perceive by \emph{opsis} is no single thing. To see is to perceive by \emph{opsis}—to that extent the supposition that generates the \emph{aporia} is correct. But not every case of perceiving by \emph{opsis} is a seeing. That is the insight that is meant to at least partially resolve the \emph{aporia}.

Allow me to make an observation and a clarification.

The observation is that the power of sight has multiple manifestations. Seeing is the exercise of sight. But so is the perceiving by \emph{opsis} that is not a seeing. In the unlovely vocabulary of modern metaphysics, sight is a multitrack power. Aristotle seems anyway committed to this. Consider the perception of the common sensibles. The visual perception of the common sensibles is distinct from the seeing of colors. Whereas seeing presents the colors, the perception of shape, say, presents shape and not color, even if these experiences co-occur.

The clarification concerns seeing. The possibility that is meant to at least partially resolve the \emph{aporia} is the possibility of perceiving by \emph{opsis} that is not seeing. But that is only as clear as our understanding of seeing that perceiving by \emph{opsis} is denied to be. So what is the relevant notion of seeing? Seeing is perceiving color in light. The \emph{aporia} turned on the fact that seeing is of color or of colored things. And color is visible in light (\emph{De Anima} 2.7 418a26–b3). Color is the power to move the actually transparent, and light, understood as a state of illumination, is the actually transparent.

% If seeing is not just perceiving by \emph{opsis}, then how are we to understand what seeing is? 

The proposed resolution of the \emph{aporia} should now be clear, at least in outline (\emph{pace} \citealt[122]{Hamlyn:2002ys}). If we perceive by \emph{opsis} that which sees, and perceiving by \emph{opsis} is seeing, and seeing is of color or colored things, then that which sees must itself be colored. But, if we can perceive by \emph{opsis} in a way that is not seeing, and if, more specifically, we can perceive by \emph{opsis} that which sees without seeing it, then no commitment to it being colored is incurred.

The difficulty posed by The Argument from Darkness and Light is less understanding the general outline of the proposed resolution, than understanding the example that is meant to establish that there is a perceiving by \emph{opsis} that is not seeing—that we discriminate dark and light, though not in the same way.

The most straightforward example establishing this would be a case where one perceives by \emph{opsis} in the absence of seeing. But, importantly, that is not required. Perceiving by \emph{opsis} may be distinct from seeing and yet occur along with it. (Thus, Themistius includes the supplement \emph{kai hotan horōmen} after \emph{mē horōmen}, \emph{In libros De Anima paraphrasis} 83.23, see \citealt[105, 181 n3]{Todd:1996aa} and \citealt[238]{Browne:1986aa}). So there are two ways to understand the case. It is either a case where:
\begin{enumerate}[(1)]
	\item Perceiving by \emph{opsis} occurs in the absence of seeing, or
	\item Perceiving by \emph{opsis} is distinct from seeing and yet may occur along with it
\end{enumerate}
And there are at least two ways to understand perceiving by \emph{opsis} in the absence of seeing. It is either a case where:
\begin{enumerate}[(a)]
	\item Under conditions of excessive darkness or light we perceive by \emph{opsis} that we do not see the colors in the scene before us, or
	\item Under conditions of low light, only relative brightness, and not chromatic hue, is perceived
\end{enumerate}
The alternative where perceiving by \emph{opsis} is not seeing color but may occur along with it is a case where:
\begin{enumerate}[(c)]
	\item A sensible, distinct from the colors, is perceived by \emph{opsis}.
\end{enumerate}
A hybrid interpretation combining elements of (a) and (c) is also possible:
\begin{enumerate}[(d)]
	\item Under conditions of excessive darkness, we perceive by \emph{opsis} that we do no see color, and under normal illumination, we perceive by \emph{opsis} light, a sensible distinct from color.
\end{enumerate}
Let us consider these alternatives in turn.


(a) \emph{Under conditions of excessive darkness or light we perceive by \emph{opsis} that we do not see the colors in the scene before us.} A transparent body, such as air or water, is actually transparent due to the presence of a fiery substance. When the fiery substance is absent, darkness supervenes. If color is the power to move what is actually transparent, then in complete darkness, there is nothing for the chromatic powers to move and are thus inactive. The perceiver is thus incapable of seeing the colors of the the scene before them when shrouded in darkness. And yet, darkness is something we visually experience. We thus perceive the darkness by \emph{opsis}. Thus if seeing is of color or of colored things, then the perceiver's visual experience of darkness is not a seeing—it is at least not the seeing of the inactive colors of the scene before them. (Aquinas offers such an interpretation, \emph{Sentencia libri De Anima}, 73–97.)

Perceiving darkness may be a straightforward case of perceiving by \emph{opsis} that is not seeing the colored scene before one, but what about perceiving light? In the case of total darkness, it was the absence of light that prevented the colors from acting upon what is actually transparent, and so prevented the perceiver from seeing the colored scene before them. But in the present case, light is not absent but present and perceived. So how is the perception of light an obstacle to seeing the colored scene before one? (Aquinas remains silent on this question.)

% (though as we shall see a doubt may be raised)

Later in the chapter (\emph{De Anima} 3.2 426b1–2), in the course of arguing that perception is a kind of ratio, Aristotle claims that an excess of brightness or darkness destroys sight (by destroying the ratio that enables sight). Perhaps, perceiving light by \emph{opsis} that is not seeing is meant to be an experience of excessive brightness that obscures the colored scene before one. (Compare Plato's description of the dazzling experience of brightness, \emph{lampros}, or brilliance, \emph{stilbos}, \emph{Timaeus} 68a1–6, see \citealt[72–7]{Kalderon:2022kl}.) In the darkness, a flashlight suddenly shown in a person's face can obscure the person weilding it. In such a case, one is blinded by the light.

If that is how we are to understand lightness perception, then one can undertand the qualification, ``though not on the same way'', as follows. Perceiving excessive darkness by \emph{opsis} is not seeing the colored scene in the same way that perceiving excessive light by \emph{opsis} is. The obstacle to seeing is different in each case. They are different ways of destroying the ratio that constitutes the power to see color. Against this Ross writes:
\begin{quote}
	In light of 422a21–23 \ldots, one might be tempted to think that {\sbl άλλ´ούχ ώσαύτως} \ldots means `but we do not perceive darkness in the same way as light'. But that is not A.'s point here; his point is that the perception of light and darkness is differenet from ordinary sight, which is the perception of \emph{colour}. \citep[275]{Ross:1961uq}
\end{quote}

On this reading, sight is only partially incapacitated. Sight is incapacitated in that, under conditions of excessive darkness or brightness, one cannot see the colors arrayed in the scene before one. But sight remains in operation. One visually experiences the darkness. And the visual experience of brilliance is dazzling. So perceiving by \emph{opsis} that is not the seeing of color remains the exercise of the power of sight. Otherwise we would lose the contrast with the different ways that sight is incapacitated when asleep, or having fainted, or rendered unconscious by having the veins in one's neck compressed (\emph{De Somno et Vigilia} 455b). When asleep or unconscious one does no perceive by \emph{opsis} the excessive darkness or brightness that surrounds one, let alone see the colors of the ambient scene.

Can the distinction between seeing color and perceiving by \emph{opsis} be sustained on this reading? A doubt may be raised given the phenomenology of our experience of excessive darkness and brightness. Darkness has a color that is revealed in our experience of it. Darkness is black. Contrast our experience of silence. Whereas out experience of the absence of light has a chromatic quality, our experience of the absence of sound lacks an audible quality. (What pitch is it? What timbre does it possess? For discussion see \citealt{Sorensen:2008kx,Sorensen:2009aa}.) And arguably, so too for excessive brightness. Timaeus claims that brilliance causes all sorts of colors to appear (\emph{Timaeus} 68a5–6). But these were meant to be cases of perceiving by \emph{opsis} that did not involve the visual awareness of color.

(b) \emph{Under conditions of low light, only relative brightness, and not chromatic hue, is perceived.} The perception of darkness and light is meant to be a case of perceiving by \emph{opsis} in the absence of seeing color. The first variant sought to establish this by excessive darkness and brightness obscuring the colored scene set before the perceiver, so that no element of that scene was in fact perceived. The second variant allows that elements of the scene be perceptually discriminated in the way that the initial variant precludes. So consider looking at a scene in low light. We can perceive the relative brightness of objects in a sufficiently dark room, even if we cannot distinguish their hue. \emph{La nuit, tous les chats sont gris}. Even the neighbor's calico cat appears gray at night. On this variant, we perceive dark and light but not the chromatic hues. We thus have an example of perceiving relative brightness by \emph{opsis} without seeing color, where seeing color is understood as, specifically, seeing chromatic hues. That darkness is black is no objection. Nor is the multi-colored character of the dazzling experience of excessive brightness. Moreover, the qualification ``though not in the same way'' may be read, not as contrasting the perception of darkness from the perception of light, but by contrasting each with seeing color as \citet[275]{Ross:1961uq} recommends.

In this way is it an advance over the initial variant. And yet difficulties remain. The present reading assimilaties discriminating darkness and light to discriminating darkness from light. But that is not what the text says. Moreover, the idea that black, white, and gray are achromatic colors that contrast with the chromatic hues such as red and green seems alien to a color scheme that takes the colors to be mixtures of black and white understood as light and dark.

(c) \emph{A sensible, distinct from the colors, is perceived by \emph{opsis}.} On this alternative, the example is not meant to be a case of perception by \emph{opsis} in the absence of seeing color. Rather, perception by \emph{opsis}, though distinct from seeing color, may occur along with it. The objects of perception by \emph{opsis} are darkness and light as opposed to the colors. 

Begin with the perception of light. In a transparent medium illuminated by the presence of the fiery substance, the chromatic powers of the scene move what is actually transparent and so affect the organ of sight, the eyes, thus enabling the seeing of that scene. In such a case, there clearly is seeing of color or colored things. In seeing the calico cat in daylight, say, I see its tripartite color. But I also visually experience the daylight. But perhaps I do not see the daylight despite seeing the colors seen through it. Light may be colored but only in a sense that contrasts with the primary sense in which the cat is tricolored. That light is only in a sense colored is insufficient for the perception of light to count as seeing color. After all, light could not be colored in the primary sense. In the primary sense, color is the power to move what is actually transparent. Light is the actual transparent. To be colored, light would have to have the power to move itself in a way that Aristotle deems impossible. How, then, is light presented in the daylit scene, if not by being colored in the primary sense? It is by being moved by chromatic powers in a manner that affects sight that light is perceived by \emph{opsis}.

Consider then the perception of darkness. Recall that on the present reading perception by \emph{opsis} is distinct from seeing color, though it may occur along with it. Unlike the first reading, it need not be a case of perception by \emph{opsis} in the absence of seeing. It was that thought that prompted the idea that darkness was complete in a manner incompatible with seeing. But having abandoned the first reading, there is no longer the pressure to understand darkness as total darkness. So consider the calico cat. It is sitting next to a stone wall, perfectly still, eyes closed, enjoying the sunlight. Since it is late afternoon, the cat casts an elongated shadow. Darkness does not so much as obscure this scene as it is a perceived element of it. The shadow is dark. But it is not completely dark. The rough texture of the stone wall may still be discerned despite the cat's shadow being cast upon it. Its shadow is not so much the complete absence of light as the diminution of light relative to the overall illumination. After all, shadows may persist despite a degree of light pollution from ambient reflection. What is within the cat's cast shadow is illuminated by significantly less light than the overall scene. Shadows may obscure a scene—one may hide in the shadows—but only to a degree. Darker shadows may obscure the color of what lies within but not so their paler siblings. If light is revealed to sight by the colors seen through it, dark is revealed to sight by the relative resistance it offers to seeing color, a resistance which is frequently less than total.

On this reading, the perception by \emph{opsis} of darkness and light is like, if not exactly like, the visual perception of the common sensibles (on the common sensibles see \emph{De Anima} 2.5 417a10–20, 3.3 428b 22–30, \emph{De Sensu} 437a9, 552b4–10, \emph{De Memoria} 450a9–12). There is no perceiving the shape of a thing without seeing the color that pervades its surface. But color and not shape is the proper object of vision. Similarly, there is no perceiving the light that pervades a scene withut seeing the colored objects it illuminates. But color and not light is the proper object of vision. Light may be perceptible to one sense alone, and so not itself a common sensible, but it 
 % is not perceptible in itself, and so is not a proper sensible. Light does not contain within itself the power of its own visibility. Light is perceptible only insofar as it enables the colors to be perceptible by moving it. Moreover, sight is for the sake of seeing colors, not perceiving light. Light, like the common sensibles, is only ever presented to sight insofar as color is seen, but unlike the common sensibles, it is sensed by no other sense. And sight is for the sake of neither.

The qualification ``though not in the same way'' must be read, not as contrasting the perception of darkness from the perception of light, but by contrasting each with seeing color as \citet[275]{Ross:1961uq} recommends.

This reading, however, is not without difficulty. A question may be raised about how it coheres with Aristotle's taxonomy of the objects of perception (\emph{De Anima} 2.6). Specifically he distinguishes proper (\emph{idion}) and common (\emph{koina}) objects and both are contrasted with incidental (\emph{kata sumbebēkota}) objects. Incidental (\emph{kata sumbebēkota}) sensibles are perceptible but not perceptible in themselves. The pale thing seen in the distance is the son of Cleon. In that sense, the son of Cleon is perceived. But while paleness may be a power to move what is actually transparent, being the son of Cleon is not. That is why being the son of Cleon, while perceptible, is not perceptible in itself—it does not contain within itself the power of its own visibility. By contrast, the proper and common sensibles are perceptible in themselves. But this class divides. Either they are perceptible to one sense alone in which case they are proper (\emph{idion}) objects, or they are perceptible to more than one sense, in which case they are common (\emph{koina}) objects.

Darkness and lightness are perceptible. Unlike incidental (\emph{kata sumbebēkota}) sensibles, they are perceptible in themselves. They are, more specifically visible. But the proper (\emph{idion}) object of vision is color (and perhaps that which has no name but is seen in the dark). But darkness and lightness are visibilia distinct from the colors (and that which has no name). So they are not the proper (\emph{idion}) objects of vision. However, darkness and lightness are perceptible to one sense alone, sight. So they are not common (\emph{koina}) objects. The puzzle is that the principles that generate the taxonomy seem to induce an exhaustive ordering in which darkness and lightness do not find a place. 

(d) \emph{Under conditions of excessive darkness, we perceive by \emph{opsis} that we do not see color, and under normal illumination, we perceive by \emph{opsis} light, a sensible distinct from color.} This is a hybrid interpretation, including, as it does, the claim about darkness perception in (a) with the claim about lightness perception in (c). Simplicius (most likely Pseudo-Simplicius) offers a such a hybrid interpretation (\emph{In libros Aristotelis De anima commentaria} 189 13–28). Concerning lightness perception, Simplicius writes ``sight does not only perceive colours that have light falling on them, but also perceives light on its own (for we see air when it is lit)'' (\emph{In libros Aristotelis de anima commentaria} 189 19–21, \citealt[42]{Blumenthal:2000rf}). And concerning when we try and fail to see the colored scene, due to excessive darkness say, Simplicius writes:
\begin{quote}
	[I]t perceives itself, both not seeing anything and seeing; it perceives itself acting, clearly, when it sees, and at the same time it has simultaneous perceptual awareness of its own activity, and in the case when it does not see it perceives itself being not entirely inactive (otherwise it would not be simultaneously aware because awareness is in addition to some activity), but is trying to see, and in trying it is in act not in respect of the seeing but of the trying, so that the awareness is not of seeing, but of the trying which is, so to speak, failing: so the sense of sight makes the judgement that it is not seeing. (Simplicius, \emph{In libros Aristotelis de anima commentaria} 189 23–8, \citealt[42]{Blumenthal:2000rf})
\end{quote}
Simplicius has an important insight, here: that the failure to see the colored scene under conditions of excessive darkness remains, nonetheless, the exercise of sight when we experience the obscuring darkness. Sight is not entirely inactive. It is active with respect to the perceiving the surrounding darkness. The qualification, ``entirely'', can be read as signalling that while under conditions of excessive darkness sight is active insofar as we experience the darkness, it is at least not active with respect to seeing the colored scene before one. On the hybrid interpretation, the qualification, ``though not in the same way'', can be read as marking a double distinction. Darkness perception as understood by (a) is contrasted with lightness perception as understood by (c), and both are contrasted with seeing color. On the hybrid interpretation, \citet[275]{Ross:1961uq}, then, is only partially right.

Let us return to the \emph{aporia} set out in The Argument from Relatives. If we perceive by \emph{opsis} that which sees, and perceiving by \emph{opsis} is seeing, and seeing is of color or colored things, then that which sees must itself be colored. But, if we can perceive by \emph{opsis} in a way that is not seeing, and if, more specifically, we can perceive by \emph{opsis} that which sees without seeing it, then no commitment to it being colored is incurred. Something need not be colored to be perceptible by \emph{opsis}, but the more general claim that for a power to be of someting it must have a certain nature or being may yet be true. It is just that the way something has to be to be perceived by \emph{opsis} is different from the way something has to be to be seen.

When one inquires into what way something must be to be perceived by \emph{opsis}, a complication is encountered. Perhaps perceiving by \emph{opsis} is itself no single thing. Consider the alternatives that competing interpretations of this notion put forward. Perceiving light in addition to the illuminated colored scene seems different in kind from the visual experience of darkness that results from trying and failing to see the colored scene under conditions of excessive darkness. If perceiving by \emph{opsis} is no single thing, then there will be no single nature or being that must be shared by all. 

A further reason for thinking that perceiving by \emph{opsis} is no single thing is the contrast between darkness and lightness perception, on the one hand, and the reflexive case—that we can perceive by \emph{opsis} that which sees. It is unlikely that that which sees shares a nature or being with light and dark, the way it would have to if perceiving by \emph{opsis} were a single thing. Alexander of Aphrodisias makes a suggestion concerning the relevant nature or being in the reflexive case:
\begin{quote}
	
\end{quote}

% The Argument from Darkness and Light, if successful on some construal, establishes at most that one can perceive by \emph{opsis} without seeing a colored thing. But that would only help resolve the puzzlement if, more specifically, one can perceive by \emph{opsis} that which sees without seeing it. But nothing so far has been said in support of that claim. To that extent at least, the Argument from Darkness and Light is at best an incomplete resolution of the \emph{aporia}.

% subsection the_argument_from_darkness (end)

\subsection{The Argument from Residual Imagery} % (fold)
\label{sub:the_argument_from_residual_imagery}

\begin{quote}
	\textsc{The Argument from Residual Imagery}: Further, that which sees (\emph{to horōn}) is in a sense colored. For each sense organ (\emph{aisthētērion}) receives the sense object without its matter; that is why even after the sense objects have gone, sensations and images remain in the sense organ (\emph{aisthētēríos}). (\emph{De Anima} 3.2 425b22–5)
\end{quote}

% subsection the_argument_from_residual_imagery (end)

% section the_opening_aporia (end)

\section{Apperceptive Unity} % (fold)
\label{sec:apperceptive_unity}

% section apperceptive_unity (end)

\section{\emph{De Somno et Vigilia}} % (fold)
\label{sec:_emph_de_somno_et_vigilia}

% section _emph_de_somno_et_vigilia (end)

\section{Coda} % (fold)
\label{sec:coda2}

% section coda (end)

% Chapter perceiving (end)