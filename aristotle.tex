%!TEX root = /Users/markelikalderon/Documents/Git/perceptual_self-consciousness/perceptual_self-consciousness.tex
\chapter{Perceiving that We See and Hear} % (fold)
\label{cha:perceiving}

\section{Introduction} % (fold)
\label{sec:introduction}

We perceive that we see and hear (\emph{De Anima} 3.2 425b12). Aristotle does not establish this so much as he presupposes it. In so doing, he poses a how-possible question (on how-possible questions see \citealt{Cassam:2007lq}). Given that we see and hear, how is it possible that we do? How-possible questions are not posed out of the blue. If they are apt, then what occasions the question is a doubt or unclarity about that very possibility. What occasions Aristotle's how-possible question is the aporetic doubt about reflexive intentional powers developed in the \emph{Charmides}. Or so shall I argue. There are at least two major points of contact. 

First, in what \citet{caston02} dubs The Duplication Argument (\emph{De Anima} 3.2 425b13–15), Aristotle claims that a perceiving of a seeing would perceive as well the object seen. Why this should be so has puzzled some commentators (\citealt[435]{Hicks:1907uq}, \citealt[121–2]{Hamlyn:1961ys}, and \citealt[44–5]{Kosman:2014ab}). Such puzzlement is abated if this claim is recognized as an instance of transitivity (\citealt{McCabe:2007ss}, \citealt{McCabe:2007jb}) or transparency (\citealt{Tsouna:2022aa}) whose denial was itself crticized in the \emph{Charmides} (most explicitly in The Argument from Benefit, \emph{Charmides} 170a–175b, but doubts were raised as early as The Puzzling Disanalogies, \emph{Charmides} 167c–168a).

Second, Aristotle raises a difficulty based on the claim that seeing is of color or that which possesses color (\emph{De Anima} 425b17–9). Given this, if perceiving that we see is a seeing that we see, then the seeing that is seen must be colored. Again, why this should be so has puzzled some commentators (\citealt[122]{Hamlyn:1961ys}, \citealt[45–6]{Kosman:2014ab}). And, again, such puzzlement is abated if this claim is recognized as an instance of a key premise from The Argument from Relatives (\emph{Charmides} 168b2–169c2). Specifically, it is an instance of the claim that in order for an psychic power to apply to its object that object must have a certain nature of being, color or being colored in the case of sight (\emph{Charmides} 168d1–5).

There are two \emph{desiderata} on an adequate interpretation of the opening \emph{aporia} of \emph{De Anima} 3.2. 

The first, then, is that it should be read in light of the puzzle about how to coherently combine intentionality and reflexivity in the \emph{Charmides}. A number of commentators have observed the influence of the \emph{Charmides} on \emph{De Anima} 3.2 (). Though this happy consensus is tempered somewhat by different understandings that they display of the \emph{Charmides} and its influence on \emph{De Anima} 3.2. 

The second \emph{desiderata}, again supported by a number of commentators (), is that the opening \emph{aporia} of \emph{De Anima} 3.2 should cohere with the content of chapter 2, insofar as possible. \citet[121]{Hamlyn:1961ys} claims that the chapter is ``rambling''. Perhaps. But that is a conclusion we should only reach after having tried our best to discover a unifying thread or threads to the chapter and failed. And there is room for disagreement about whether there is a unifying thread or threads and what these might be.

The joint satisfaction of the two \emph{desiderata} thus does not constrain interpretation to uniqueness. In this chapter, I shall elaborate and defend an interpretation of the opening \emph{aporia} of \emph{De Anima} 3.2 that aspires to jointly satisfy these two \emph{desiderata}. Specifically, I shall elaborate and defend an Alexendrian interpretation, but only in the sense of defending a key claim of Alexander of Aphrodisias' interpretation in \emph{Quaestiones} 3.7.

% section introduction (end)

\section{The Opening Aporia} % (fold)
\label{sec:the_opening_aporia}



% section the_opening_aporia (end)

% Chapter perceiving (end)