%!TEX root = /Users/markelikalderon/Documents/Git/perceptual_self-consciousness/perceptual_self-consciousness.tex
\chapter{First Excursus} % (fold)
\label{cha:first_excursus}

\section{Atopos, Interpretation, and Understanding} % (fold)
\label{sec:atopos_interpretation_and_understanding}

% section atopos_interpretation_and_understanding (end)

\section{Two Ways Forward} % (fold)
\label{sec:coda}

Etymologically, \emph{aporia} derives from \emph{aporos} which itself derives from \emph{a poros}. It is a lack of passage, or a difficult passage (as in Xenophon \emph{Anabasis} 5.6.10). In the context of dialectic, it is a difficulty, and in rationally experiencing \emph{aporia} (the way Socrates does in contrast with Critias \emph{Charmides} 169c3–6) one fails to see a way forward through this difficulty.

Our discussion of the Socratic \emph{aporia} has put us in a position to see at least two potential ways forward.

\textsc{The Argument from Relatives} reveals a tension between reflexivity and intentionality. For a psychic power to apply to itself or its exercise, where in so doing it takes either the power or its exercise as its intentional object, it must have a certain nature or being, the nature or being shared by its intentional objects generally. The problem is that it is implausible that the relevant psychic powers or their activities have the requisite nature or being. How then are we to make sense of reflexive psychic powers if such there be? How are we to understand perceptual self-consciousness?

Perhaps the crucial premise of \textsc{The Argument from Relatives} is the obstacle to the possibility of reflexive psychic powers: That in order for a power to apply to something it must have a certain nature or being (\emph{ousia}). However, even stated in full generality, this principle seems plausible. Indeed, Aristotle seems to accept it. Consider active powers. Active powers only act on that which is capable of receiving their activity (\emph{Metaphysica} {\sbl Θ} 1). So consider the power to burn: fire can only consume the flammable. Active powers only act on those things that possess the relevant passive power. So possession of the relevant passive power is the nature or being required for the active power to apply to it. Similarly for passive powers. Passive powers are only ever exercised when acted upon by a corresponding active power.

% Perceptual powers are passive, or at least have a passive element. Perceptual powers are, in Nietzsche's vocabulary, reactive capacities (for discussion or Nietzsche on reactive powers see \citealt{Deleuze:2006as}, for its relation to Aristotle on perception see \citealt[27]{Kalderon:2015fr}). Perceptual powers only act be reacting to the presence of a sensible particular. Aristotle makes this point by means of an analogy with combustion (\emph{De anima} 2.5 417a3–10). The presence of the sensible particular ignites sensory consciousness. Thus vision only acts by reacting to the presence of colored particular. So color is the nature or being that a particular must possess in order to be seen.

Maybe it is not, after all, the crucial premise of \textsc{The Argument from Relatives} that is the obstacle here. Assuming the argument can be made sound by a Great Man deploying the method of division, then there is a genuine conflict between reflexivity and intentionality. Given the \emph{aporia}, one could either hang onto intentionality and reject reflexivity or hang onto reflexivity and reject intentionality.

Rejecting reflexivity is not to reject self-knowledge or perceptual self-conscious\-ness but only a conception of what these amount to. Perhaps self-knowledge and perceptual self-consciousness are genuine psychic phenomena, but they are misunderstood as reflexive. Rather, they are reflective. That is to say that the psychic power or its exercise is genuinely an intentional object, but not of that very power. Rather, they are the intentional object of a distinct reflective power whose exercise is applied to these. Thus \textsc{Reflexive} is rejected and \textsc{Higher Order} is retained in order to ensure that the power or its activity may be the intentional object of the act that affords awareness of it. (This is roughly McCabe's \citeyear{McCabe:2007ss} position.)

Retaining reflexivity need not involve rejecting the crucial premise of \textsc{The Argument from Relatives}. Perhaps, it is the way that premise combines with intentionality that ought to be rejected. A psychic power, such as sight and hearing, when exercised, may apply to their intentional object. And in order for a psychic power to apply to a thing as its intentional object, then it must be a certain way, colored in the case of vision, sonorous in the case of audition. But powers can apply to things without taking those things as intentional objects. Consider natural powers unconnected to the soul, such as the power to burn. Is the only way for a psychic power to apply to a thing is by taking it as an intentional object?

For vision to apply to a thing such that it is seen it must be colored. If in seeing what one does and so being aware that one sees must seeing be the intentional object of the seeing that afforded this self-awareness? Or is the intentional object restricted to the colored particular? If the latter, then the seeing is not among its intentional objects. Of course, in order for sight to apply to its activity and so afford the perceiver an awareness from within of their seeing the scene before them, seeing must be some way. But the way it must be for sight to apply need not be being colored. Maybe the way seeing must be in order to afford this reflexive self-awareness is in being the exercise of sight in seeing what one does (Alexander of Aphrodisias makes a similar suggestion with respect to his commentary on Aristotle's \emph{De Anima} 3.2 in \emph{Quaestiones} 3.7.) So understood, perceptual self-awareness accompanies, potentially at least, every episode of perception (Peter John Olivi seems to have held such a view, see \citealt{Brower-Toland:2024qa}; Descartes, another thinker influenced by Augustine, see \citealt{Menn:1998nr}, accepts this, as does the avowed Cartesian, Sartre.). It may be true that something must be some way for a power to apply to it. But the way something must be to be the intentional object of the power's activity need not be the way that activity must be for the power to reflexively apply to it. Perhaps the power applies in a different way to its own activity than it does to the object of that activity. Thus \textsc{Reflexive} is retained not by denying a crucial premise of The Argument from Relatives but by denying that the power or its activity is the intentional object of that activity.

There is a potential phenomenological insight here. The insight, if it is one, has positive and negative lessons: It both supports the retention of reflexivity and the rejection of intentionality and provides an independent reason to reject the replacement of reflexivity with reflectivity as a means of retaining intentionality. 

Walking through an ancient woodlands, I descend a steep slope and come upon clearing bringing into view the ruined remains of The Abbey of St Mary and St Thomas the Martyr at Lesnes. It was founded by Richard de Lucy in 1178, perhaps in penance for his role in the death of Thomas Becket, the Archbishop of Canterbury. The community of Lesnes Abbey drained the marshlands, now southeastern suburbs of London, and were responsible for maintaining the southern bank of the Thames nearby. The abbey was closed by Cardinal Wolsey in 1525 as part of a larger program to abolish smaller communities. Moving from the woodlands to the manicured lawn of the clearing, the abbey comes into view. Lesnes Abbey is in a ruined state, lacking a roof and many of the walls having been pulled downed, though the outline of the courtyard can still be seen.

My attention is drawn to architectural details still discernible: An arched doorway on the Western side of the courtyard, a lancet window (narrow, pointed, and arched), the base, all that remains, of a column. The pulpit is now little more than a heap of stones, and the crypt, containing the heart of the great great granddaughter of Richard de Lucy, Roesia of Dover, is mercilessly exposed to the elements.

% I approach the ruined column. It stands a little over a meter high. There is a main column whose base is set upon a larger base. Cascading behind the main column are series of lesser columns. The remains of the column proper are a warm hue, while its base and that which it is set upon are blackened. The structure is before me. In perceiving the structure, it is set before my mind, as Hume would say. My conscious attention is directed upon the structure, a material body distinct from my consciousness. The ruined column may be the object of my visual experience, but in undergoing a visual experience that affords me an awareness of the column, I am aware from within of my seeing that column. While I may be absorbed in the ruined column, in closely observing it, there remains a sense in which I am aware from within of my perceiving it. In perceiving the structure, I undergo a kind of perceptual self-consciousness, but my perceiving is not set before this perceptual consciousness in the way that the ruined column is. The seeing of the ruined column is less seen that lived. The intrusion of the Sartrean vocabulary is to highlight the first-personal aspect of the reflexive awareness afforded by consciously seeing the remains of the column. If the seeing were seen, the perceiver would not be aware of the seeing from within, but, rather, it would be set before the perceiver, and though the seeing is their seeing, it would be experienced \emph{qua} other. Compare, Aristotle insists that the healer and the healed are distinct and so if the doctor and patient are one, they are so incidentally, a fact that he expresses by saying that the doctor in healing himself does so by treating himself \emph{qua} other. Similarly, if perception could perceive itself, it could only do so \emph{qua} other \citep[8]{Rodl:2007aa}. If the seeing were seen rather than lived, we would stand in an alienated spectatorial relation to our own perceptual activity in a way that we manifestly do not (\citealt{Moran:2001aa}, see also \citealt{shoemaker96} for a case against a perceptual model of self-consciousness). But that is all that a reflective substitute for the reflexive could afford. If the seeing and the reflexive awareness of so seeing arise through the application of sight, then it is plausible that these arise from different modes or aspects of the power's manifest activity, and that different conditions must be met for the remains to be seen and for my being aware from within of my seeing of them.

In viewing the ruins of Lesnes Abbey, the remains in view may be the object of my visual experience. In so viewing, they remains are set before my mind, as Hume would say. Moreover, in undergoing a visual experience that affords me awareness of the remains, I am aware from within of my seeing what I do. But in being aware from within, my seeing is not set before me in the way that the remains are. In seeing the remains, the remains may be known to me, but my seeing of them is not known but lived. The point of applying the Sartrean rhetoric is to highlight the first-personal aspect of the reflexive awareness afforded by consciously seeing the remains of the abbey. If the seeing were seen, the perceiver would not be aware of the seeing from within, but, rather, it would be set before the perceiver, and though the seeing is their seeing, it would be experienced \emph{qua} other. Consider the following comparison. Aristotle insists that the healer and the healed are distinct, and so if the physician and patient are one, then they are so incidentally, a fact that he expresses by saying that the physician in healing himself does so by treating himself \emph{qua} other (\emph{Metaphysica} {\sbl Δ} 12 1019a15–18). Similarly, if perception could perceive itself, it could only do so \emph{qua} other \citep[8]{Rodl:2007aa}. If the seeing were seen rather than lived, we would stand in an alienated spectatorial relation to our own perceptual activity in a way that we manifestly do not (\citealt{Moran:2001aa}, see also \citealt{shoemaker96} for an influential case against perceptual models of self-consciousness). But that is all that a reflective substitute for the reflexive could so much as afford us. If the seeing and the reflexive awareness of so seeing arise through the application of sight, then it is plausible that these arise from different modes or aspects of the power's manifest activity, and that different conditions must be met for the remains to be seen and for my being aware from within of my seeing of them.

The \emph{Charmides} does not provide us with an account of perceptual self-conscious\-ness. It merely provides an \emph{aporia} about reflexive powers and a sketch of a dialectical program for how that \emph{aporia} might be resolved. Like \citet{Kosman:2014aa}, and unlike \citet{McCabe:2007ss}, I believe that we should retain reflexivity and reject intentionality. I believe that thinking through the \emph{aporia} will lead one to something like Sartre's non-positional (or non-thetic, or pre-reflective, on the differences between these see \citealt{Webber:2002aa}) consciousness, but I am less sanguine than Kosman is that this might actually be Plato's view. If it seems strange to find a proleptic anticipation Sartre's notion of non-positional consciousness in an ancient \emph{aporia}, there are clear historical precedents for Sartre's notion. Ibn Sina and Suhrawardi posit a primitive form of self-awareness that makes reflective self-awareness possible (for discussion see \citealt{Kaukua:2014si}). And similar ideas can be found in the Latin West (for discussion see \citealt{Cory:2013oh}). And the idea re-emerges with Fichte (for discussion see \citealt{Heinrich:1966aa}). Like everything new on Earth, its path was laid down beforehand, and for a long time.

% Notice that the central negative claim of The Argument from Relatives—that psychic powers applying to their activities and taking them as intentional objects is either impossible or at least open to serious doubt—is a claim that Sartre could happily accept. The intentional object of a conscious act transcends that act. Though the conscious act may be directed upon its object, the act could not contain what lies beyond. While an episode of seeing may be the object of a distinct psychic act
%
%
% The object is set before consciousness

% section coda (end)



% chapter first_excursus (end)