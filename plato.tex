%!TEX root = /Users/markelikalderon/Documents/Git/perceptual_self-consciousness/perceptual_self-consciousness.tex
\chapter{The Third Offering to Zeus the Savior} % (fold)
\label{cha:offering}

\section{Introduction} % (fold)
\label{sec:introduction}



Our concern is with perceptual self-consciousness. But in order to frame the Socratic puzzles about perceptual self-consciousness, we shall begin by briefly discussing Critias' proposed definition of \emph{sōsphrosynē} as a kind of self-knowledge and its background since this will structure the hypothetical reflexive perceptual powers. 

The puzzles about the reflexive being of psychic powers such as \emph{sōsphrosynē}, self-knowledge, and perceptual self-consciousness, are themselves puzzling. For the \emph{aporiai} are reached on the back of claims to which Socrates, a participant of the dialogue, has elicited assent from Critias, but these are also claims that Socrates, the narrator of the dialogue, explicitly denies. The claims of the Socratic \emph{elenchus} leading to \emph{aporiai} in the argumentative portions of the dialogue must then be assessed against the claims of the dramatic portions of the text. The \emph{logos} must somehow be harmonized with the \emph{ergon}. As \citet{Schmid:1998aa} observes, the dialogue whose central puzzles concern reflexive modes of being is itself a text that ``relates itself to itself''. What might Plato signal thereby?

Addressing this requires that we distinguish three different perspectives in the dialogue:

\begin{enumerate}[(1)]
	\item \emph{Perspective 1}: The first is the perspective of Plato, the author of the dialogue, and his contemporary audience.
	\item \emph{Perspective 2}: The second is the perspective of Socrates, not the participant of the dialogue but its narrator, and his unnamed audience.
	\item \emph{Perspective 3}: The final perspectives are the perspectives of the participants of the dialogue, principally Socrates, Chaerephon, Charmides, and Critias.
\end{enumerate}

We need to attend to the perspective of Plato and his contemporary readership. For the common historical experience of the recent past will, at the various least, make certain aspects of the text salient in a way those aspects may not be to a reader, like ourselves, with a very different experience. Specifically, Critias and Charmides are relatives of Plato and participants in the reign of the Thirty Tyrants but did not survive it. Those who lived through that political disaster will naturally attend to certain aspects of the dialogue.

We need to attend to the perspective of Socrates' narration and its reception by his unnamed audience. Importantly, Socrates in narrating the dialogue either contradicts or says something in tension with what Socrates, the participant of the dialogue, claims. Specifically, Socrates in narrating the dialogue makes claims about perception that contradict the claims about perception made by the narrated Socrates in the course of his criticism of Critias. As we have seen, this combines with the first perspective. It is Plato who writes Socrates in narrating the dialogue make claims inconsistent with the claims the narrated Socrates makes. Whether or not this deliberate, understanding why this is so bears on the meaning of the text. 

We need to attend to the perspectives of the participants of the dialogue. The narrated events that precede Socrates' conversation with Charmides and then Critias are relevant to and so perhaps contains lessons about the main topic of the dialogue, \emph{sōphrosunē}. This too combines with the first perspective. For contrast the later tyrannical careers of Critias and Charmides and Critias and Charmides as they appear in the dialogue. How do the participants differ from their later selves? Perhaps more interestingly, can we observe the germs of their future tyranny? How would they have to be different to avoid there deaths defending tyranny?

The perspective of Plato and his contemporary readership raises a complication that I shall raise without resolving, since it is mostly irrelevant to my specific concerns. The dating of the dialogue matters. Most commentators hold that the dialogue was written after the reign of the Thirty Tyrants. But at least one commentator, \citet[108–9]{Schleiermarcher:1836aa} no less, maintains that the dialogue was written during their reign. \emph{Sōsphrosynē} is both an individual virtue as well as a civic virtue. So just as an individual may be \emph{sophron} so too may be the governance of a \emph{polis}. Even given our incomplete and imperfect understanding of the untranslatable virtue, given all accounts, we may be confident that the reign of the Thirty Tyrants was not \emph{sophron} (for example, Socrates was pressured by the Thirty to level a false accusation against a citizen whose property they sought to seize). Potential criticism of the Thirty might take different forms if during their reign or afterwards \citep[42]{Hyland:1981aa}, and this might lead the reader to differently interpret the text.

% section introduction (end)

\section{The Third Offering} % (fold)
\label{sec:the_third_offering}

Puzzles about reflexive psychic powers arise in the context of assessing Critias' proposal that \emph{sōphrosunē} is a kind of self-knowledge. Specifically, \emph{sōphrosunē} is a kind of knowledge (\emph{epistēmē}). But it is unlike the knowledge involved in \emph{technai} such as medicine or architecture. Such knowledge is of a subject matter () that is distinct from it. So a physician who possesses the art of medicine has knowledge of a certain subject matter, health, and the physician's knowledge is distinct from its subject matter. Knowledge may be knowledge of something, but Critias maintains that \emph{sōphrosunē} is knowledge of itself and other knowledges and their lack. This is the proposal that Socrates agrees to investigate. That investigation has two parts:

\begin{quote}
	Once more then, I said, as our third offering to the Saviour, let us consider afresh, in the first place, whether such a thing as this is possible or not——to know that one knows, and does not know, what one knows and what one does not know; and secondly, if this is perfectly possible, what benefit we get by knowing it. (\emph{Charmides} 167a—b; \citealt[57]{Lamb:1927qw})
\end{quote}


% section the_third_offering (end)

\section{The Disanalogies} % (fold)
\label{sec:the_disanalogies}

\begin{quotation}
	Come then, I said, Critias, consider if you can show yourself any more resourceful than I am; for I am puzzled [\sbl{ἀπορῶ}]. Shall I explain to you in what way?
	
	By all means, he replied.
	
	Well, I said, what all this comes to, if your last statement was correct, it is merely that there is one knowledge which is precisely a knowledge of itself and the other knowledges, and moreover is a knowledge of the lack of knowledge [\sbl{ἀνεπιστημοσύνμς}] at the same time.
	
	Certainly.
	
	Then mark what a strange [\sbl{ἄτοπον}] statement it is that we are attempting to make, my friend: for if you will consider it as applied to other cases, you will surely see——so I believe——its impossibility [\sbl{ἀδύνατον}]. (\emph{Charmides} 167b—c; \citealt[57]{Lamb:1927qw}, modified)
\end{quotation}

Socrates is puzzled and explains that this is due to the strangeness (\emph{atopon}) of  Critias' account. His account is strange because it is unlike other more familiar cases. In these cases, the application Critias' account to them results in a manifest impossibility (\emph{adūnaton}). Socrates' puzzlement, here, does not so much as cast doubt on Critias' account, in the sense of providing a positive reason, however provisional, for rejecting that account, as it is an invitation to further inquiry (here I am in agreement with \citealt{Politis:2008nv}). It is likely that Critias understands Socrates puzzlement in this way. For Critias has earlier charged Socrates with eristic refutation:
\begin{quote}
	There you are, Socrates, he said: you push your investigation up to the real question at issue—in what \emph{sōphrosynē} differs from all the other knowledges—but you then proceed to seek some resemblance between it and them; whereas there is no such thing. (166b–c; \citealt[53]{Lamb:1927qw}, modified.)
\end{quote}	
Were Socrates pressing the disanalogies as a reason to reject the account, Critias would have good grounds to revive this complaint in a way that he declines to do. So it is neither an enthymeme (Aristotle, \emph{Rhetoric} 1402b15) nor an epagogic argument as many commentators maintain (see, for example, \citealt[41]{Robinson:1941yb}), but rather provides a motive for further inquiry, the results of which are the conclusion of The Argument from Relatives.

The disanalogies fall into three groups There are perceptual, conative, and cognitive cases:
\begin{enumerate}[(1)]
	\item \emph{Perceptual}: sight (\emph{opsis}), hearing (\emph{akounē}), and the other senses more generally (\emph{peri pasōn tōn aisthēseon}) (167c8–d10)
	\item \emph{Conative}: appetite (\emph{epithumia}), wish (\emph{boulēsis}), love (\emph{eros}), and fear (\emph{phobos}) (167e1–168a?)
	\item \emph{Cognitive}: opinion (\emph{doxan}) (168a?)
\end{enumerate}

The first case, sight, makes explicit the parallels with Critias' account:
\begin{quote}
	Ask yourself if you think there is a sort of vision which is not the vision of things that we see in the ordinary way, but a vision of itself and of the other sorts of visions, and of the lack of vision likewise; which, while being vision, sees no colour, but only of itself and the other sorts of vision. (167c–d; \citealt[59]{Lamb:1927qw})
\end{quote}
Like the self-knowledge with which Critias identifies \emph{sōphrosunē}, the hypothetical form of vision is reflexive (it is a vision of itself) and higher-order (it is a vision of the other sorts of visions and not only these but their lack as well). But the hypothetical form of vision has an additional element to which Critias has yet to accede. The other sorts of visions take as their objects the colors. But the vision of itself and the other sorts of visions does not take the colors as its object the way that the other sorts of visions do. So this vision is intransitive or nontransparent. The vision of itself may take the other sorts of visions as its objects but it does not, in turn, take the objects of these other sorts of vision as its own object. The vision of itself that takes the other sorts of vision as an object does not see through them to colorful scenes that they disclose.

% section the_disanalogies (end)

\section{The Argument from Relatives} % (fold)
\label{sec:the_argument_from_relatives}



% section the_argument_from_relatives (end)


% Chapter offering (end)