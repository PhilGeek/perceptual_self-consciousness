%!TEX root = /Users/markelikalderon/Documents/Git/perceptual_self-consciousness/perceptual_self-consciousness.tex
\chapter{The Third Offering to Zeus the Savior} % (fold)
\label{cha:offering}

\section{Introduction} % (fold)
\label{sec:introduction}

At the center of Plato's \emph{Charmides} is a puzzle. That puzzle concerns a mode of reflexive being. Its target is described in such general terms to highlight the potential proleptic nature of this dialogue. Whereas \citet{Kahn:1988aa} interprets the \emph{Charmides} as proleptically anticipating themes discussed in \emph{Res Publica}, it might equally be interpreted as proleptically anticipating the puzzles about reflexive being, such as the self-predicating nature of the Forms, raised in the \emph{Parmenides}. In the \emph{Charmides}, the reflexive mode of being pertains to certain psychic powers such as \emph{sōsphrosynē}, self-knowledge, and perceptual self-consciousness. \emph{Sōsphrosynē} is the notoriously untranslatable term for a virtue central to the self-conception of Athenian aristocrats, not least those who were admirer's of Sparta. Critias, an aristocratic admirer of Sparta, proposes at certain point that \emph{sōsphrosynē} might simply be a kind of self-knowledge. And in discussing whether the relevant kind of self-knowledge is possible, the possibility of analogous kinds of reflexive perceptual powers are discussed. These reflexive perceptual powers are hypothetical analogues introduced by Socrates to emphasize the strangeness of the reflexive epistemic power with which Critias identifies \emph{sōsphrosynē}. These hypothetical perceptual powers might reasonably be interpreted as capacities for perceptual self-consciousness. 

Though a reasonable interpretation it remains a substantive one. There may be grounds to recommend alternatives. However, my philosophical interest, throughout this essay, concerns the nature of perceptual self-consciousness. In attempting to understand the nature of perceptual self-consciousness, at least a partial advance can be made by successfully resolving certain \emph{aporiai} concerning it. It is in that spirit that I am approaching the \emph{Charmides}. For as long as the hypothetical reflexive perceptual powers may reasonably be interpreted as capacities for perceptual self-consciousness, then a challenge remains for anyone who seeks to understand perceptual self-consciousness.

That is not to say that I am uninterested in exegesis. By no means. Indeed, I shall argue, throughout this present essay, that we can undertand at least some aspects of the nature of perceptual self-consciousness by coming to an understanding of traditional texts concerning concerning that subject matter. That is the sense in which the present investigation is aptly described as a hermeneutic phenomenology. 

% Allow me to address, briefly, all too briefly, two objections to the foregoing. My aim is less to quell dissent than to plead for further hearing.
%
% First, one might wonder why focus on ancient texts when a lot of water has flowed under the bridge since? Relatedly, why consult texts from antiquity, as opposed to neuroscience or any other relevant empirical discipline?
%
% There is a lot to be said, but let these preliminary remarks suffice for now. First, we, all of us, enjoy, or suffer if you must, perceptual self-consciousness. So too Plato. So reflection on the problems raised in understanding perceptual self-conscious by a perceptually self-conscious agent is worth considering, not least if raised by a thinker of the stature of Plato. Second, though a lot of water has flowed under the bridge, perhaps not all of it is potable. At any rate, the reader is not being asked to consider ancient accounts of perceptual self-consciousness so much as ancient puzzles as to its nature. Third, neuroscience and other relavant empirical disciplines answer questions against a philosophical background, however implicit (consider the philosophical background behind Anil Seth's claim that perception is a mode of veridical hallucination, this could only be a seventeenth century inheritance). If your questions are not their questions, or if you do not share their philosophical background assumptions, then you must look elsewhere. And \emph{aporiai} raised by respected predecessors is a good place to start.
%
% A Lutheran objection, from one long grown impatient, ``What the right hand gives, the left hand takes away.'' On the one hand, you say that you want to understand perceptual self-consciousness by coming to an understanding of traditional texts concerning that subject matter, but on the other hand, you also say that the puzzle about reflexive perceptual powers is a puzzle about perceptual self-consciousness on an interpretation for which their may be grounds to recommend an alternative. Are you serious about exegesis or not? The implied sense that I am not is driven by the assumption that a traditional text admits of a uniquely true interpretation. But is that really plausible? We tend to think that great works of art or literature are great, in part, because they are hermeneutically fecund, because they admit of endless interpretative possibilities. Moreover, these diverse interpretations potentially yield diverse insights. Might not something similar be true of traditional philosophical texts? If we reflect, say, on the diversity of opinion on display in the commentary tradition on \emph{De anima}, is this best understood as a history of hermeneutic blunders? Or is it rather a history of diverse insights into the subject matter of that treatise yielded by thinkers who came to an understanding of that text, given their own aims, historical background, and philosophical and scientific assumptions? This might be especially true of Platonic texts whose aporetic character often seems like a deliberate provocation for the reader to come to an understanding of its subject matter for themselves. An attitude perhaps reflected in his pedagogy, for he seems to have raised independent thinkers rather than dogmatic adherents to Platonic doctrine.

% We shall revisit these methodological issues as we proceed.

Our concern is with perceptual self-consciousness. But in order to frame the Socratic puzzles about perceptual self-consciousness, we shall begin by briefly discussing Critias' proposed definition of \emph{sōsphrosynē} as a kind of self-knowledge and its background since this will structure the hypothetical reflexive perceptual powers. 

The puzzles about the reflexive being of psychic powers such as \emph{sōsphrosynē}, self-knowledge, and perceptual self-consciousness, are themselves puzzling. For the \emph{aporiai} are reached on the back of claims to which Socrates, a participant of the dialogue, has elicited assent from Critias, but these are also claims that Socrates, the narrator of the dialogue, explicitly denies. The claims of the Socratic \emph{elenchus} leading to \emph{aporiai} in the argumentative portions of the dialogue must then be assessed against the claims of the dramatic portions of the text. The \emph{logos} must somehow be harmonized with the \emph{ergon}. As \citet{Schmid:1998aa} observes, the dialogue whose central puzzles concern reflexive modes of being is itself a text that ``relates itself to itself''. What might Plato signal thereby?

Addressing this requires that we distinguish three different perspectives in the dialogue:

\begin{enumerate}[(1)]
	\item \emph{Perspective 1}: The first is the perspective of Plato, the author of the dialogue, and his contemporary audience.
	\item \emph{Perspective 2}: The second is the perspective of Socrates, not the participant of the dialogue but its narrator, and his unnamed audience.
	\item \emph{Perspective 3}: The final perspectives are the perspectives of the participants of the dialogue, principally Socrates, Chaerephon, Charmides, and Critias.
\end{enumerate}

We need to attend to the perspective of Plato and his contemporary readership. For the common historical experience of the recent past will, at the various least, make certain aspects of the text salient in a way those aspects may not be to a reader, like ourselves, with a very different experience. Specifically, Critias and Charmides are relatives of Plato and participants in the reign of the Thirty Tyrants but did not survive it. Those who lived through that political disaster will naturally attend to certain aspects of the dialogue.

We need to attend to the perspective of Socrates' narration and its reception by his unnamed audience. Importantly, Socrates in narrating the dialogue either contradicts or says something in tension with what Socrates, the participant of the dialogue, claims. Specifically, Socrates in narrating the dialogue makes claims about perception that contradict the claims about perception made by the narrated Socrates in the course of his criticism of Critias. As we have seen, this combines with the first perspective. It is Plato who writes Socrates in narrating the dialogue make claims inconsistent with the claims the narrated Socrates makes. Whether or not this deliberate, understanding why this is so bears on the meaning of the text. 

We need to attend to the perspectives of the participants of the dialogue. The narrated events that precede Socrates' conversation with Charmides and then Critias are relevant to and so perhaps contains lessons about the main topic of the dialogue, \emph{sōphrosunē}. This too combines with the first perspective. For contrast the later tyrannical careers of Critias and Charmides and Critias and Charmides as they appear in the dialogue. How do the participants differ from their later selves? Perhaps more interestingly, can we observe the germs of their future tyranny? How would they have to be different to avoid there deaths defending tyranny?

The perspective of Plato and his contemporary readership raises a complication that I shall raise without resolving, since it is mostly irrelevant to my specific concerns. The dating of the dialogue matters. Most commentators hold that the dialogue was written after the reign of the Thirty Tyrants. But at least one commentator, \citet[108–9]{Schleiermarcher:1836aa} no less, maintains that the dialogue was written during their reign. \emph{Sōsphrosynē} is both an individual virtue as well as a civic virtue. So just as an individual may be \emph{sophron} so too may be the governance of a \emph{polis}. Even given our incomplete and imperfect understanding of the untranslatable virtue, given all accounts, we may be confident that the reign of the Thirty Tyrants was not \emph{sophron} (for example, Socrates was pressured by the Thirty to level a false accusation against a citizen whose property they sought to seize). Potential criticism of the Thirty might take different forms if during their reign or afterwards \citep[42]{Hyland:1981aa}, and this might lead the reader to differently interpret the text.

% section introduction (end)

\section{The Third Offering} % (fold)
\label{sec:the_third_offering}

Puzzles about reflexive psychic powers arise in the context of assessing Critias' proposal that \emph{sōphrosunē} is a kind of self-knowledge. Specifically, \emph{sōphrosunē} is a kind of knowledge (\emph{epistēmē}). But it is unlike the knowledge involved in \emph{technai} such as medicine or architecture. Such knowledge is of a subject matter () that is distinct from it. So a physician who possesses the art of medicine has knowledge of a certain subject matter, health, and the physician's knowledge is distinct from its subject matter. Knowledge may be knowledge of something, but Critias maintains that \emph{sōphrosunē} is knowledge of itself and other knowledges and their lack. This is the proposal that Socrates agrees to investigate. That investigation has two parts:

\begin{quote}
	Once more then, I said, as our third offering to the Saviour, let us consider afresh, in the first place, whether such a thing as this is possible or not——to know that one knows, and does not know, what one knows and what one does not know; and secondly, if this is perfectly possible, what benefit we get by knowing it. (\emph{Charmides} 167a—b; \citealt[57]{Lamb:1927qw})
\end{quote}


% section the_third_offering (end)

\section{The Argument from Analogy} % (fold)
\label{sec:the_argument_from_analogy}

\begin{quotation}
	Come then, I said, Critias, consider if you can show yourself any more resourceful than I am; for I am at a loss [\sbl{ἀπορῶ}]. Shall I explain to you in what way?
	
	By all means, he replied.
	
	Well, I said, what all this comes to, if your last statement was correct, it is merely that there is one knowledge which is precisely a knowledge of itself and the other knowledges, and moreover is a knowledge of the lack of knowledge [\sbl{ἀνεπιστημοσύνμς}] at the same time.
	
	Certainly.
	
	Then mark what a strange [\sbl{ἄτοπον}] statement it is that we are attempting to make, my friend: for if you will consider it as applied to other cases, you will surely see——so I believe——its impossibility [\sbl{ἀδύνατον}]. (\emph{Charmides} 167b—c; \citealt[57]{Lamb:1927qw}, modified)
\end{quotation}

% section the_argument_from_analogy (end)

\section{The Argument from Relatives} % (fold)
\label{sec:the_argument_from_relatives}



% section the_argument_from_relatives (end)


% Chapter offering (end)