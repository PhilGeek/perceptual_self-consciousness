%!TEX root = /Users/markelikalderon/Documents/Git/perceptual_self-consciousness/perceptual_self-consciousness.tex
\chapter{The Third Offering to Zeus the Savior} % (fold)
\label{cha:offering}

\section{Introduction} % (fold)
\label{sec:introduction}



% Our concern is with perceptual self-consciousness. But in order to frame the Socratic puzzles about perceptual self-consciousness, we shall begin by briefly discussing Critias' proposed definition of \emph{sōsphrosynē} as a kind of self-knowledge and its background since this will structure the hypothetical reflexive perceptual powers.
%
% The puzzles about the reflexive being of psychic powers such as \emph{sōsphrosynē}, self-knowledge, and perceptual self-consciousness, are themselves puzzling. For the \emph{aporiai} are reached on the back of claims to which Socrates, a participant of the dialogue, has elicited assent from Critias, but these are also claims that Socrates, the narrator of the dialogue, explicitly denies. The claims of the Socratic \emph{elenchus} leading to \emph{aporiai} in the argumentative portions of the dialogue must then be assessed against the claims of the dramatic portions of the text. The \emph{logos} must somehow be harmonized with the \emph{ergon}. As \citet{Schmid:1998aa} observes, the dialogue whose central puzzles concern reflexive modes of being is itself a text that ``relates itself to itself''. What might Plato signal thereby?
%
% Addressing this requires that we distinguish three different perspectives in the dialogue:
%
% \begin{enumerate}[(1)]
% 	\item \emph{Perspective 1}: The first is the perspective of Plato, the author of the dialogue, and his contemporary audience.
% 	\item \emph{Perspective 2}: The second is the perspective of Socrates, not the participant of the dialogue but its narrator, and his unnamed audience.
% 	\item \emph{Perspective 3}: The final perspectives are the perspectives of the participants of the dialogue, principally Socrates, Chaerephon, Charmides, and Critias.
% \end{enumerate}
%
% We need to attend to the perspective of Plato and his contemporary readership. For the common historical experience of the recent past will, at the various least, make certain aspects of the text salient in a way those aspects may not be to a reader, like ourselves, with a very different experience. Specifically, Critias and Charmides are relatives of Plato and participants in the reign of the Thirty Tyrants but did not survive it. Those who lived through that political disaster will naturally attend to certain aspects of the dialogue.
%
% We need to attend to the perspective of Socrates' narration and its reception by his unnamed audience. Importantly, Socrates in narrating the dialogue either contradicts or says something in tension with what Socrates, the participant of the dialogue, claims. Specifically, Socrates in narrating the dialogue makes claims about perception that contradict the claims about perception made by the narrated Socrates in the course of his criticism of Critias. As we have seen, this combines with the first perspective. It is Plato who writes Socrates in narrating the dialogue make claims inconsistent with the claims the narrated Socrates makes. Whether or not this deliberate, understanding why this is so bears on the meaning of the text.
%
% We need to attend to the perspectives of the participants of the dialogue. The narrated events that precede Socrates' conversation with Charmides and then Critias are relevant to and so perhaps contains lessons about the main topic of the dialogue, \emph{sōphrosunē}. This too combines with the first perspective. For contrast the later tyrannical careers of Critias and Charmides and Critias and Charmides as they appear in the dialogue. How do the participants differ from their later selves? Perhaps more interestingly, can we observe the germs of their future tyranny? How would they have to be different to avoid there deaths defending tyranny?
%
% The perspective of Plato and his contemporary readership raises a complication that I shall raise without resolving, since it is mostly irrelevant to my specific concerns. The dating of the dialogue matters. Most commentators hold that the dialogue was written after the reign of the Thirty Tyrants. But at least one commentator, \citet[108–9]{Schleiermarcher:1836aa} no less, maintains that the dialogue was written during their reign. \emph{Sōsphrosynē} is both an individual virtue as well as a civic virtue. So just as an individual may be \emph{sophron} so too may be the governance of a \emph{polis}. Even given our incomplete and imperfect understanding of the untranslatable virtue, given all accounts, we may be confident that the reign of the Thirty Tyrants was not \emph{sophron} (for example, Socrates was pressured by the Thirty to level a false accusation against a citizen whose property they sought to seize). Potential criticism of the Thirty might take different forms if during their reign or afterwards \citep[42]{Hyland:1981aa}, and this might lead the reader to differently interpret the text.

% section introduction (end)

\section{The Third Offering} % (fold)
\label{sec:the_third_offering}

Puzzles about reflexive psychic powers arise in the context of assessing Critias' proposal that \emph{sōphrosunē}—the notoriously untranslatable virtue central to the self-conception of philolaconian Athenian aristocrats—is a kind of self-knowledge. Specifically, \emph{sōphrosunē} is a kind of knowledge (\emph{epistēmē}). But it is unlike the knowledge involved in \emph{technai} such as medicine or architecture. Such knowledge is of a subject matter (\emph{mathēma}) that is distinct from it. So a physician who possesses the art of medicine has knowledge of a certain subject matter, health and disease, and the physician's knowledge is distinct from its subject matter. Knowledge may be knowledge of something, but Critias maintains that \emph{sōphrosunē} is knowledge of itself, and other knowledges, and their lack, and of no other thing. This is the account that Socrates proposes to investigate. That investigation has two parts:

\begin{quotation}
	\emph{Socrates}: Then only the \emph{sophron} person will know himself, and will be able to discern what he really knows and does not know, and have the power of judging what other people likewise know and think they know, in the cases where they do know, and again, what they think they know without knowing it; everyone else will be unable And so this is being \emph{sophron}, or \emph{sōphrosunē}, and knowing oneself—that one should know what one knows and does not know. Is that what you mean?
	
	\emph{Critias}: It is, he replied.
	
	\emph{Socrates}: Once more then, I said, as our third offering to the Saviour, let us consider afresh, in the first place, whether such a thing as this is possible or not——to know that one knows, and does not know, what one knows and what one does not know; and secondly, if this is perfectly possible, what benefit we get by knowing it. (\emph{Charmides} 167a9–b4; \citealt[57]{Lamb:1927qw})
\end{quotation}

When Critias, in his long speech (164d4–165c4), proposes that \emph{sōphrosunē} is a kind of self-knowledge, Socrates first emphasizes, as we would put it, the intentional character of knowledge: If \emph{sōphrosunē} is knowing (\emph{gignõskein}), then it must be knowledge (\emph{epistēmē}) of something. If \emph{sōphrosunē} is knowledge of something, it is natural to ask what is it knowledge of? The third offering is meant to be Critias' full answer to this question, albeit an answer that has been refined through Socratic examination. The third offering, then, is a specification of the object or intentional content of \emph{sōphrosunē}.

The third offering to the Savior (\emph{to triton tō sōtēri} 167a9) is traditionally a ritual libation to Zeus \emph{sōtēr} on the third pour of a symposium (Aeschylus Fr. 55, Pindar \emph{Isthmian} 6.5, see also Plato \emph{Res Publica} 583b and \emph{Philebus} 66d, for the theological background see \citealt{Cook:1914la} and  \citealt{Jim:2022ay}, and for the epithet see \citealt{Rothrauff:1966nh}). The application of the ritual libation to Critias' account may be understood along the lines described in the \emph{Timaeus} and the \emph{Critias}. It is reasonable to call upon God at the outset of an undertaking, and the Gods are invoked to pray for their approval of that undertaking (\emph{Timaeus} 27c1–7). In the present case, the divine invocation is Socrates' describing the account as the third offering to the Savior. And since the present undertaking is an account of \emph{sōphrosunē}, the invocation is a prayer that the account may endure if true and meet with divine approval (\emph{Critias} 106a1–b6), but should the account be false, that a just penalty be imposed—the just penalty in the case of error being correction. Finally, the invocation is a prayer for knowledge, the most perfect of medicines, so that one speak truly in the future. (For further discussion of the significance of the offering see \citealt[191–2]{Tsouna:2022aa})

According to the third offering \emph{sōphrosunē} is knowledge (\emph{epistēmē}) of:
\begin{enumerate}[(1)]
	\item itself (\emph{autē heautēs} 166c3),
	\item other knowledges,
	\item their absence (\emph{anepistēmosunēs} 167c2, 166e7–8),
	\item and no other thing
\end{enumerate}

Allow me to briefly comment on each of these aspects of the content of \emph{sōphrosunē}.

(1) \emph{Itself}: When Socrates points out that, on a previous account, one could be and act \emph{sophron} without knowing that one is \emph{sophron} (164c5–6), Critias pivots and identifies \emph{sōphrosunē} with self-knowledge (164d4–165c4). While it is perhaps uncontroversial that \emph{sōphrosunē} should involve self-knowledge, at least as an element, Critias is making the grander claim that this very thing, self-knowledge, just is \emph{sōphrosunē} (164d3–4, on the significance of this, see \citealt{Kosman:2014aa}). No doubt hoping to secure Socrates' assent (\citealt[23–4]{Tuckey:1951aa}, \citealt[81]{Hyland:1981aa}, \citealt[161–2]{Tsouna:2022aa}), Critias' invokes and interprets the Delphic inscription ``Know Thyself'' (\emph{Gnōthi sauton}) in developing his new account. As Critias develops this account under Socratic examination, \emph{sōphrosunē} is claimed to be knowing oneself (\emph{gignōskein heauton} 165b4), knowledge of oneself (\emph{epistēmē heatou} 165e1), and finally knowledge of itself (\emph{epistēmē autē heautēs} 166c3). Bracketing the slide from \emph{gnosis} to \emph{epistēmē} (which \citealt{Hyland:1981aa} and \citealt{Schmid:1998aa} regard as a poisoned chalice), one might reasonably query the move from knowing oneself to knowledge of itself—after all, there has been an uncommented upon shift from a personal to an impersonal reflexivity. To be sure, in knowing oneself, what is known is not separate from the subject of such knowing, but that does not entail that what is known is the knowledge itself. Conversely, in possessing a knowledge which is knowledge of itself, does one really know oneself? Later Critias will answer this question in the affirmative (169d9–e5): Just as in possessing swiftness one is similar to it and so swift (a proleptic anticipation of the self-predicating nature of the Forms), when one possesses knowledge of itself one will be similar to it and so know oneself. Thus, the content of the knowledge that constitutes \emph{sōphrosunē} is, in modern parlance, reflexive: Such knowledge takes as its object that very knowledge, at least in part.

(2) \emph{Other Knowledges}: Such knowledge is not only of itself, but it is also of other knowledges. When Socrates inquires into the content of \emph{sōphrosunē}—specifically what, according to Critias, it is knowledge of, he emphasizes that, in a range of familiar cases, what is known is distinct from the knowledge of them (166a3–7). In Sartre's terminology, the intentional object is transcendent in the sense that it goes beyond the conscious act that is directed upon it. Thus arithmetic (or perhaps calculating or reckoning, \emph{logistikē}) involves knowledge of the odd and the even and their quantitative relations where these are distinct from such knowledge (166a5–11). So too for weighing (\emph{statikē}) where the heavy and the light are distinct from the knowledge of them (166b1–4). Presumably the same holds for knowledge of medicine and architechture. What's known in each of these cases is distinct from the knowledge of it. This leads Socrates to ask what \emph{sōphrosunē} is knowledge of such that is distinct from this knowledge (166b5–6)? Critias responds that while every other form of knowledge is knowledge of something distinct from itself, \emph{sōphrosunē} is different—it alone is knowledge of these knowledges and of itself (166b9–c3). \emph{Sōphrosunē} is, according to Critias, a kind of sovereign knowledge, governing all other forms of knowledge. It is this sovereign status that that distinguishes \emph{sōphrosunē} from the other knowledges that it governs and justifies why it alone should take itself as an object and so depart from the pattern displayed by other forms of knowledge. Thus, the content of the knowledge that constitutes \emph{sōphrosunē} is, in modern parlance, higher-order: Such knowledge takes as its object other knowledges, at least in part.

(3) \emph{Their Absence}: Critias having characterized \emph{sōphrosunē} as knowledge which alone is of itself and other knowledges, Socrates suggests a refinement that Critias readily accepts. If \emph{sōphrosunē} is knowledge of other knowledges it must also be of their lack (\emph{anepistēmosunēs}). Later, knowledge of good will be claimed to involve knowledge of its opposite, evil (174b9–c3). This suggests that a more general conception of knowledge may be in play here. On this conception, knowledge is, or at least involves, a discriminatory power. To know a thing one must be able to discriminate it from its opposite.  Thus a physician in possessing knowledge of medicine has the power to discriminate health from its opposite, disease. (Further evidence for this conception can be found in \emph{Phaedo} 97d1-5, \emph{Res Publica} 333d–334a. Aristotle will take up and develop this Academic conception in \emph{Topica} 105b5, 110b20, 155b30-34, 164a1 and \emph{Metaphysica} Theta 2). So in the present case, the \emph{sophron} would have the power to discriminate knowledge from its opposite, ignorance. Socrates understands this discriminatory power discursively. It involves being able to test (\emph{exetasai} 167a2) what one knows (\emph{eidōs} 167a3) and does not know and the power to examine (\emph{episkopein} 167a3) what others know and do not know and this occurs in the medium of conversation. The discursive dimension of this discriminatory capacity means that knowledge of other knowledges and their lack has first- and third-personal aspects (the perspectives of the speaker and their conversational participants, respectively). Though Critias readily accepts the Socratic refinement, he perhaps understands its significance differently. A \emph{sophron} ruler will not only know what they know and do not know, but they will also know what their subordinates know and do not know and so will be able to assign them appropriate responsibilities in the functioning of the city state. So the sovereign knowledge is the knowledge of a \emph{sophron} sovereign. Thus, the content of the knowledge that constitutes \emph{sōphrosunē} involves not only knowledge but importantly its lack. 

(4) \emph{And No Other Thing}: Critias claims that (a) \emph{sōphrosunē} alone (\emph{monē} 166c2, 166e5) is of itself and other knowledges. By contrast, (b) all other knowledges are not of themselves or other knowledges but rather have proprietary objects or subject matters that are distinct from such knowledge. Later (167b10–c2), Socrates will add a further element: that (c) \emph{sōphrosunē} is knowledge of itself and other knowledges and their lack and no other thing (\emph{ouk allou tinos} 167b11). So the content of \emph{sōphrosunē} is exhausted by these, having no further aspect to its subject matter. Just as the previous feature—that the content of \emph{sōphrosunē} should include the absence of knowledge—is a Socratic refinement, so too is the present feature. So far, Critias has only explicitly claimed that \emph{sōphrosunē} is unique in being knowledge of itself and other knowledges and has acceded to Socrates' claim that it is also of the absence of knowledge. Now it is being claimed to be exclusively of these objects (Perhaps Socrates has introduced a novel idea that Critias fails to notice, \citealt[37–8]{Duncombe:2020gi}, or perhaps he is merely making explicit what was implicit in Critias' account, \citealt[]{Tsouna:2022aa}). (a)–(c) has an unstated implication that will be made fully explicit in The Argument from Benefit (172b–175b) but begins to emerge in Socrates' pressing puzzling disanalogies to Critias' account (discussed in section~\ref{sec:puzzling_disanalogies}). If \emph{sōphrosunē} alone is of itself and other knowledges and their lack, and this exhausts the content of \emph{sōphrosunē}, then \emph{sōphrosunē} will not have as part of its content the proprietary objects of the other knowledges. So \emph{sōphrosunē} is intransitive or nontransparent. \emph{Sōphrosunē} may take the other knowledges as its objects but it does not, in turn, take the objects of these other knowledges as its own. (Compare \citealt[190]{Tsouna:2022aa}.) This is a puzzling result. While one may reasonably know that another possesses knowledge that one lacks—say, if that knowledge is manifest in successful action, say, this is harder to maintain in the first-person case. How can one know that one knows without thereby knowing what one knows? Thus, the content of the knowledge that constitutes \emph{sōphrosunē} is exhausted by itself, other knowledges, and their lack.

Examining the identification of \emph{sōphrosunē} with a form of self-knowledge occasions Critias' charge that Socrates is engaged in eristic refutation (166c3–6). Perhaps, in the background, Socrates and Critias are working with different conceptions of dialectical reasoning, and it is this that gives rise to Critias' misunderstanding of Socrates' motives (for a reading of this contrast see \citealt[chapter 4]{Schmid:1998aa}). At any rate, Critias entered the conversation defensively, in the spirit of competition and desirous of victory (\emph{agōniōn kai philotimōs} 162c1–2), angry that Charmides has bungled the defence of an idea originating with Critias (just ``as a poet does with an actor who mishandles his verses'' 162d2–3; \citealt[41]{Lamb:1927qw}). Perhaps Socrates feels the need to moderate Critias' intemperance (ironic, given the topic of conversation) for he provides an explicit rationale for examining Critias' identification of \emph{sōphrosunē} with a form of self-knowledge. That rationale will consist in a certain puzzlement or \emph{aporia} that arises when Critias' account is applied to a range of familiar cases.

% \citet[44]{Tuckey:1951aa} observes an ambiguity, present in both the Greek and the English, that raises an issue that will be addressed throughout the remainder of this essay, specifically whether knowledge of knowledge ``refers to possibility a \emph{particular act} of knowledge being its own object, or to the possibility of knowledge \emph{in general} being the object of knowledge''. Whereas \citet[772–3]{caston02} argues for the activity reading, \citet[218]{Tuozzo:2011aa} argues instead for a power reading: ``the knowledge of medicine, say, is most naturally construed as a disposition'', so not the activity which is the exercise of a power, but the power itself. The puzzling disanologies will begin to shed light on this issue.

% section the_third_offering (end)

\section{Puzzling Disanalogies} % (fold)
\label{sec:puzzling_disanalogies}

Having offered Critias' account of the content of \emph{sōphrosunē} an offering to the Savior, Socrates motivates its examination:
\begin{quotation}
	\emph{Socrates}: Come then, I said, Critias, consider if you can show yourself any more resourceful than I am; for I am puzzled (\emph{aporō}). Shall I explain to you in what way?
	
	\emph{Critias}: By all means, he replied.
	
	\emph{Socrates}: Well, I said, what all this comes to, if your last statement was correct, it is merely that there is one knowledge which is precisely a knowledge of itself and the other knowledges, and moreover is a knowledge of the lack of knowledge at the same time.
	
	\emph{Critias}: Certainly.
	
	\emph{Socrates}: Then mark what a strange (\emph{atopon}) statement it is that we are attempting to make, my friend: for if you will consider it as applied to other cases, you will surely see—so I believe—its impossibility (\emph{adunaton}). (\emph{Charmides} 167b–sc; \citealt[57]{Lamb:1927qw}, modified)
\end{quotation}

Socrates is puzzled and explains that this is due to the strangeness (\emph{atopon}) of  Critias' account. (It is not of this place and so ``strange'' in the sense of ``a stranger in a strange land'', \emph{Exodus} 2:22 KJV, and so foreign rather than absurd.) His account is strange because it is unlike other more familiar cases. In these cases, the application Critias' account to them results in a manifest impossibility (\emph{adunaton}). Notice that, strictly speaking, what is claimed to be impossible is not Critias' account but its application to other cases. Socrates' puzzlement, here, does not so much as cast doubt on Critias' account, in the sense of providing a positive reason, however provisional, for rejecting that account, as it is an invitation to further inquiry (here, in this respect, I am in agreement with \citealt{Politis:2008nv}). It is likely that Critias understands Socrates puzzlement in this way. For Critias has earlier charged Socrates with eristic refutation:
\begin{quote}
	\emph{Critias}: There you are, Socrates, he said: you push your investigation up to the real question at issue—in what \emph{sōphrosunē} differs from all the other knowledges—but you then proceed to seek some resemblance between it and them; whereas there is no such thing. (\emph{Charmides} 166b–c; \citealt[53]{Lamb:1927qw}, modified.)
\end{quote}	
Were Socrates pressing the puzzling disanalogies as a reason to reject the account, Critias would have good grounds to revive this complaint in a way that he declines to do. So it is neither an enthymeme (Aristotle, \emph{Rhetorica} 1402b15) nor an epagogic argument as many commentators maintain (see, for example, \citealt[41]{Robinson:1941yb}), but rather provides a motive for further inquiry (the results of which are the conclusion of The Argument from Relatives). Indeed, at the end of this discussion, Socrates makes this point explicitly:
\begin{quote}
	\emph{Socrates}: And it is a strange thing, if it really exists? For we should not affirm as yet it does not exist, but we should consider whether it does exist. (\emph{Charmides} 168a10–1; \citealt[61]{Lamb:1927qw})
\end{quote}
So the puzzling disanalogies do not establish that Critias' strange knowledge does not exist, but it does motivate an examination into whether it does in fact exist. Critias, mollified, agrees (168a1).

The disanalogies fall into three groups. There are perceptual, conative, and cognitive cases:
\begin{enumerate}[(1)]
	\item \emph{Perceptual}: sight (\emph{opsis}), hearing (\emph{akounē}), and the senses all together (\emph{peri pasōn tōn aisthēseon}) (167c8–d10)
	\item \emph{Conative}: appetite (\emph{epithumia}), wish (\emph{boulēsis}), love (\emph{eros}), and fear (\emph{phobos}) (167e1–168a2)
	\item \emph{Cognitive}: opinion (\emph{doxan}) (168a3–5)
\end{enumerate}
(Compare the taxonomies of \citealt[114–8]{Hyland:1981aa} and \citealt[90]{Schmid:1998aa}. Ignoring the grouping induced by ordering, \citealt[207 n37]{Tsouna:2022aa}, claims that there is no textual evidence for any such taxonomy, but Tsouna's real complaint is that any such taxonomy plays no role in the argument. However, this last thought is vitiated by the disanalogies not constituting an argument against Critias' account but rather providing a reason to examine it.) All of these cases either are or involve psychic powers. Only living beings have perceptual, conative, and cognitive powers. And only living beings exercise such powers. So though Socrates never explicitly claims that these powers are powers of the soul (\emph{psuchē}), insofar as the soul is the principle of life, only beings with souls have such powers. Not only do such cases involve psychic powers, but all such powers are intentional in the minimal sense that they take an object. In exercising these powers there is something that is seen, heard, or perceived more generally. There is something desired, wished, loved, or feared. There is something about which one has an opinion. Moreover, knowledge, generally, and the self-knowledge with which Critias identifies \emph{sōphrosunē}, specifically, are themselves intentional psychic powers.

The first case, sight, makes explicit the parallels with Critias' account:
\begin{quote}
	\emph{Socrates}: Ask yourself if you think there is a sort of vision which is not the vision of things that we see in the ordinary way, but a vision of itself and of the other sorts of visions, and of the lack of vision likewise; which, while being vision, sees no colour, but only of itself and the other sorts of vision. (\emph{Charmides} 167c–d; \citealt[59]{Lamb:1927qw})
\end{quote}
The hypothetical form of vision closely, if not perfectly, parallels the self-knowledge with which Critias identifies \emph{sōphrosunē}. Consider then the four features of that self-knowledge described in section~\ref{sec:the_third_offering}. The hypothetical form of vision is of:
\begin{enumerate}[(1)]
	\item itself (\emph{heautēs} 167c9)
	\item other visions
	\item their absence (\emph{mē opseōn} 167c10)
	\item and no other thing (or at least not of the things we see in the ordinary way, namely color)
\end{enumerate}
And presumably, like the self-knowledge with which Critias identifies \emph{sōphrosunē}, it has first- and third-personal aspects. That is to say, that it is a vision of the visions enjoyed, or suffered if you will, by the perceptual subject as well as a vision of the visions of other perceptual subjects (this latter is not made explicit in the \emph{logos} but figures prominently in the \emph{ergon}, for example at 155b8–c1). So the other sorts of visions envisioned may be enjoyed by the perceiver themself or by other perceivers. 

So far so similar. But there are differences as well.

First, Socrates begins to make explicit an implicit commitment of the third offering. We observed that if \emph{sōphrosunē} alone is of itself and other knowledges and their lack, and this exhausts the content of \emph{sōphrosunē}, then \emph{sōphrosunē} will not have as part of its content the proprietary objects of the other knowledges. This implication is made explicit with the hypothetical form of vision: It is not of things that we see in the ordinary way. So the hypothetical vision is intransitive or nontransparent. It may take other visions as its objects but it does not, in turn, take the objects of these other visions as its own object. The vision of itself that takes the other sorts of vision as an object does not see through them to colorful scenes that they disclose.

Second, and relatedly, ordinary vision, unlike the hypothetical vision, is of colors. As Socrates applies Critias' account to other more familiar cases, the relevant psychic powers, be they perceptual, conative, or cognitive, each take their proper objects (if I may help myself to this Peripatetic anachronism). Thus vision is of colors, and colors are the objects of no other sense. One can neither hear colors nor feel them (despite the claims of certain psychics to be color-feelers, \citealt{Duplessis:1975aa}). It is in this sense that color is the proper object of sight (I prescind here from the further Peripatetic thought that perceiving proper objects are immune to error, see \citealt[chapter 4.2]{Kalderon:2015fr} for discussion). Nor are colors the proper objects of the conative or cognitive powers. We may have opinions about the colors of things (the color of the \emph{peplops} adorning the statue of Athena during the Panathenea, say), but the opinable and not the colors discursively articulated in the opinable are the proper object of \emph{doxa} (for a contemporary discussion of this point see \citealt[section 5]{Kalderon:2011fk}). 

The terminology may be Peripatetic, but the claim that ordinary sensory powers have proper objects, in the sense of objects that are perceptible to one sense alone, is genuinely Platonic. Consider the following passage from the \emph{Theaetetus}:
\begin{quotation}
	\emph{Socrates}: And are you also willing to admit that what you perceive through one power, you can't perceive through another? For instance, what you perceive through hearing, you couldn't perceive through sight, and similarly what you perceive through sight you couldn't perceive through hearing?
	
	\emph{Critias}: I could hardly refuse to grant that. (\emph{Theaetetus} 184e8–185a3; Levett and Burnyeat in \citealt[204]{Cooper:1997fk})
\end{quotation}
Notice that Plato links objects being perceptible to one sense alone to a conception of the senses as powers. Two thoughts seem to be at work here: that powers are individuated by their proper exercise, and that the proper exercise of a sensory power is the presentation of its proper object in sensory awareness. These two claims in conjunction with specific claims about the proper objects of vision and audition imply that sight just is the capacity to see color, and audition just is the capacity to hear sound. Moreover, Plato is willing to generalize these claims to other psychic powers. They figure, for example, in an argument that knowledge and opinion are distinct powers (\emph{Res Publica} V 477--478). And in the \emph{Charmides}, we see that for each of the ordinary forms of the relevant powers, Socrates specifies a proper object:
\begin{enumerate}[(1)]
	\item vision (\emph{opsis}): color (\emph{chrōma} 167d1)
	\item hearing (\emph{akounē}): sound (\emph{phōnēs} 167d4)
	\item the senses all together (\emph{peri pasōn tōn aisthēseon}): the sensible (\emph{aisthanmenē} 167d9, though as we shall see the generality raises an issue here)
	\item appetite (\emph{epithumia}): pleasure (\emph{hēdonēs} 167e1)
	\item wish (\emph{boulēsis}): good (\emph{agathon} 167e4)
	\item love (\emph{eros}): beauty (\emph{kalou} 167e8)
	\item fear (\emph{phobos}): the dreadful (\emph{deinōn} 168a1)
	\item opinion (\emph{doxa}): the opinable
\end{enumerate}

This general claim, that the ordinary forms of the relevant psychic powers take proper objects, is not to be confused with a similar claim previously encountered. Specifically, ordinary branches of knowledge are distinguished by their proprietary objects or subject matters. Thus medicine is of health and disease just as arithmetic is of the odd and the even. But the proper object of \emph{epistēmē} is \emph{mathēma} (the knowable or, more literally, the lesson learned). It is this object that distinguishes \emph{epistēmē} from \emph{doxa}. Health and disease, no less than the odd and the even, at least when discursively articulated and organized into a science, are distinct species of the epistemic genus \emph{mathēma}.

The second perceptual case, hearing, closely follows this pattern:
\begin{quotation}
	\emph{Socrates}: And what do you say to a sort of hearing which hears not a single sound, but hears itself and the other sorts of hearing and lack of hearing
	
	\emph{Critias}: I reject that also. (167d4–6; \citealt[59]{Lamb:1927qw})
\end{quotation}
The hypothetical form of hearing is of:
\begin{enumerate}[(1)]
	\item itself
	\item other hearings
	\item their absence
	\item and no other thing (or at least no other thing that we hear in the ordinary way, namely, sound)
\end{enumerate}
Just as color is the proper object of vision, sound is plausibly the proper object of audition (though see \citealt[chapter 4.2]{Kalderon:2018oe} for criticism). And since sound is what ordinary hearings hear, the hearing of these hearings is intransitive or nontransparent. Since hearing itself and other hearings involves hearing no sound, one does not hear through the hearings to their proper objects.

The third perceptual case is difficult to interpret:
\begin{quotation}
	\emph{Socrates}: Then take all the senses together as a whole, and consider if you think there is any sense of the senses and of itself, but insensible of any of the things of which the other senses are sensible?
	
	\emph{Critias}: I do not. (\emph{Charmides} 167d7–10; \citealt[59]{Lamb:1927qw})
\end{quotation}

There are at least three general ways to understand this passage (167d4–6): 
\begin{enumerate}[(1)]
	\item On the first reading, having discussed vision and audition, instead of enumerating the rest of senses and applying Critias' account to each, Socrates, in effect, says ``and so on for the rest of the senses'' (\citealt[113–4]{Bloch:1973aa} and \citealt[89]{Schmid:1998aa}). Thus, for example, what is deemed impossible is a sense of smell that smells itself and other smellings but does not smell what other smellings smell, and so on for all the other senses such as taste and touch. So understood, the passage describes the general application of Critias' account to each of the senses. 
	\item On the second reading, the impossibility does not pertain to Critias' account as applied to the rest of the senses. Rather, among all the senses, there is a special sense that takes itself and the ordinary senses as objects, but does not sense what the ordinary senses sense (\citealt[214–5]{Tuozzo:2011aa}).  And that is what is deemed impossible. So understood, this passage is a proleptic anticipation of, and perhaps inspiration for, Aristotle's notion of \emph{koinē aisthēsis}. It at least anticipates one of the many functions that scholars have attributed to this Peripatetic psychic power. 
	\item On the third reading, there is a deliberate indeterminacy to this passage. In effect, it can be read as an invitation to reflect on all the ways that Critias' account might be applied to the senses generally (\citealt[202]{Tsouna:2022aa}). So understood, the first two readings are merely alternatives to be considered in further discussion of these matters.
\end{enumerate}

The third perceptual case does not perfectly parallel Critias' account of \emph{sōphrosunē}. How exactly it departs from that account depends upon how our passage (167d4–6) is best interpreted.

First, the Socratic refinement goes unmentioned. While the sense is of itself and the other senses, no mention is made of it also being of their lack. At least on the first reading, the Socratic refinement is intelligible, and it is open to understand Socrates' query as elliptical, the refinement not explicitly stated but implicitly understood. So what would be impossible, among other things, is a sense of smell that does not smell what other smellings smell but only itself and other smellings and their lack. But on the second reading, where Socrates is proleptically anticipating \emph{koinē aisthēsis}, this is harder to maintain. Can the perceiver sense the absence of sensing? What exactly are we imagine their experience to be like? Are we to imagine them as conscious while their senses are inoperative, like Ibn Sina's Flying Man (\emph{al-Nafs} 1.1, 5.7)? But unlike in Ibn Sina's Flying Man thought experiment, this residual consciousness is meant to be sensory, making the conceivability task more difficult, if indeed possible at all.

Second, it is unclear whether the sensible is, or even could be, a proper object. On the first reading, the sensible is not itself a proper object but is rather a generic formal description of the proper objects of the ordinary senses. It may be possible on the second reading that it is a proper object, if the sensible is a genus of which the proper objects of the ordinary senses are species, but this is speculative insofar as it lacks a firm textual basis.

Do the perceptual cases canvassed by Socrates refer to perceptual powers or perceptual activities that are their exercise? On the former reading, Socrates would be asking where there is a power of sight that takes itself and other powers of sight and their absence as objects. On the latter reading, Socrates would be asking whether there is a seeing that sees itself and other seeings and their absence. 

The power reading is supported by the fact that \emph{opsis} (sight), \emph{akoē} (hearing), and \emph{aisthēsis} (perception) typically, if not invariably, refer to perceptual powers rather than their episodic exercise. And against \citet[772–3]{caston02}, \citet[218 n18]{Tuozzo:2011aa} argues that the perceptual cases are meant to be analogous to \emph{epistēmē}, and since knowledge involves or is constituted by the possession of a power, then it would be reasonable to understand sight, hearing, and the senses taken altogether as themselves powers. 

The activities reading is supported by the plural form used in the formulations—sight of sights (167c10), hearing of hearings (167d4–5), sense of senses (). Though \emph{opsis}, \emph{akoē}, and \emph{aisthēsis} are typically used for perceptual powers, they can be used to designate the activities of these powers, and their plural occurrences strongly speak in favor of the activity reading. It is ordinary seeings and hearings that are the objects of the hypothetical form of vision and audition. Holding fast to the powers reading, the plural would force us to attribute different kinds of powers of sight and audition respectively, but that is implausible in this context. 

As to Tuozzo's complaint against Caston, while contemporary orthodoxy may hold that knowledge is stative as opposed to episodic, knowledge might be spoken of in many ways. If the action potential of knowledge is never actualized or actualized only with difficulty, then there is some pressure to withdraw the knowledge attribution, at least on some understanding of knowledge relevant to the practical circumstances. A band leader in auditions asks ``Do you know Billy's Bounce?'' One musician answers ``Yes'' and immediately plays the head with good phrasing and rhythm. Another answers ``Yes'' and begins to recollect the head, ``Let' see, it's an F blues\ldots How does it go again? \ldots Oh yeah, fifth, um, sharp eleven, fifth, and then root, flat third, third, root, six, I got this\ldots'' It would not be unreasonable for the band leader to conclude that the latter does not know the tune, despite being able to recollect it with effort, and should be sent back to the shed. There is an understanding of knowledge relevant to the practical circumstances that makes this so. The former knows the tune in a way that the latter does not as evinced by their mastery of it. Aristotle makes this thought explicit in \emph{De Anima} 2.5. 

% \citet[214–7]{Tuozzo:2011aa} observes that while some of the proper objects are given a substantive specification others are merely given a formal characterization. The case of vision nicely illustrates this contrast since it begins with the formal characterization of the proper object of vision, what sight is the seeing of, but later provides a substantive specification of this, namely color. The proper objects of vision, hearing, appetite, wish, love, and fear are all substantively specified. However, in the cases of the senses all together, opinion, and knowledge, we are merely given a formal characterization of their proper objects.


% section puzzling_disanalogies (end)

\section{The Argument from Relatives} % (fold)
\label{sec:the_argument_from_relatives}



% section the_argument_from_relatives (end)


% Chapter offering (end)