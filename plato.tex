%!TEX root = /Users/markelikalderon/Documents/Git/perceptual_self-consciousness/perceptual_self-consciousness.tex
\chapter{The Third Offering to Zeus the Savior} % (fold)
\label{cha:offering}

\section{Introduction} % (fold)
\label{sec:introduction}

Among the powers of living beings are perceptual, conative, and cognitive powers. Insofar as these are powers of living beings, and only beings with souls (\emph{psuches}) live, then these are psychic powers. And since they are of something (\emph{tinos})—there is something seen, heard, desired, wished, loved, feared, opined, or known, then they are intentional psychic powers.

A puzzle arises when we consider whether these powers might also be reflexive. Can these psychic powers be applied to themselves, or at least to their exercise, such that the powers themselves, or their exercise, are their intentional object? So, for example, can there be a visual awareness of sight or seeing where the sight, or the seeing, is its intentional object?

An initial difficulty is that, in a familiar range of cases, the intentional objects of these psychic powers are distinct both from these powers and their exercise. The color seen is distinct from sight or the seeing of it. The dreadful is distinct from the fear of it. Should this be a general feature of intentionality—that the intentional object transcends (in Sartre's sense) the the psychic powers and the activities that they give rise to, then this would preclude such powers from being, at the same time, reflexive, in the sense of the powers, or their activities, being their own intentional objects.

There is a tension, then, between the intentional character of these psychic powers and their alleged reflexivity. This tension forms the basis of a puzzle or \emph{aporia} that is at the heart of the \emph{Charmides}.



% Our concern is with perceptual self-consciousness. But in order to frame the Socratic puzzles about perceptual self-consciousness, we shall begin by briefly discussing Critias' proposed definition of \emph{sōsphrosynē} as a kind of self-knowledge and its background since this will structure the hypothetical reflexive perceptual powers.
%
% The puzzles about the reflexive being of psychic powers such as \emph{sōsphrosynē}, self-knowledge, and perceptual self-consciousness, are themselves puzzling. For the \emph{aporiai} are reached on the back of claims to which Socrates, a participant of the dialogue, has elicited assent from Critias, but these are also claims that Socrates, the narrator of the dialogue, explicitly denies. The claims of the Socratic \emph{elenchus} leading to \emph{aporiai} in the argumentative portions of the dialogue must then be assessed against the claims of the dramatic portions of the text. The \emph{logos} must somehow be harmonized with the \emph{ergon}. As \citet{Schmid:1998aa} observes, the dialogue whose central puzzles concern reflexive modes of being is itself a text that ``relates itself to itself''. What might Plato signal thereby?
%
% Addressing this requires that we distinguish three different perspectives in the dialogue:
%
% \begin{enumerate}[(1)]
% 	\item \emph{Perspective 1}: The first is the perspective of Plato, the author of the dialogue, and his contemporary audience.
% 	\item \emph{Perspective 2}: The second is the perspective of Socrates, not the participant of the dialogue but its narrator, and his unnamed audience.
% 	\item \emph{Perspective 3}: The final perspectives are the perspectives of the participants of the dialogue, principally Socrates, Chaerephon, Charmides, and Critias.
% \end{enumerate}
%
% We need to attend to the perspective of Plato and his contemporary readership. For the common historical experience of the recent past will, at the various least, make certain aspects of the text salient in a way those aspects may not be to a reader, like ourselves, with a very different experience. Specifically, Critias and Charmides are relatives of Plato and participants in the reign of the Thirty Tyrants but did not survive it. Those who lived through that political disaster will naturally attend to certain aspects of the dialogue.
%
% We need to attend to the perspective of Socrates' narration and its reception by his unnamed audience. Importantly, Socrates in narrating the dialogue either contradicts or says something in tension with what Socrates, the participant of the dialogue, claims. Specifically, Socrates in narrating the dialogue makes claims about perception that contradict the claims about perception made by the narrated Socrates in the course of his criticism of Critias. As we have seen, this combines with the first perspective. It is Plato who writes Socrates in narrating the dialogue make claims inconsistent with the claims the narrated Socrates makes. Whether or not this deliberate, understanding why this is so bears on the meaning of the text.
%
% We need to attend to the perspectives of the participants of the dialogue. The narrated events that precede Socrates' conversation with Charmides and then Critias are relevant to and so perhaps contains lessons about the main topic of the dialogue, \emph{sōphrosunē}. This too combines with the first perspective. For contrast the later tyrannical careers of Critias and Charmides and Critias and Charmides as they appear in the dialogue. How do the participants differ from their later selves? Perhaps more interestingly, can we observe the germs of their future tyranny? How would they have to be different to avoid there deaths defending tyranny?
%
% The perspective of Plato and his contemporary readership raises a complication that I shall raise without resolving, since it is mostly irrelevant to my specific concerns. The dating of the dialogue matters. Most commentators hold that the dialogue was written after the reign of the Thirty Tyrants. But at least one commentator, \citet[108–9]{Schleiermarcher:1836aa} no less, maintains that the dialogue was written during their reign. \emph{Sōsphrosynē} is both an individual virtue as well as a civic virtue. So just as an individual may be \emph{sophron} so too may be the governance of a \emph{polis}. Even given our incomplete and imperfect understanding of the untranslatable virtue, given all accounts, we may be confident that the reign of the Thirty Tyrants was not \emph{sophron} (for example, Socrates was pressured by the Thirty to level a false accusation against a citizen whose property they sought to seize). Potential criticism of the Thirty might take different forms if during their reign or afterwards \citep[42]{Hyland:1981aa}, and this might lead the reader to differently interpret the text.

% section introduction (end)

\section{The Third Offering} % (fold)
\label{sec:the_third_offering}

Puzzles about reflexive powers arise in the context of assessing Critias' proposal that \emph{sōphrosunē}—the notoriously untranslatable virtue central to the self-conception of philo-Laconian aristocrats—is a kind of self-knowledge. Specifically, \emph{sōphrosunē} is a kind of knowledge (\emph{epistēmē}). But it is unlike the knowledge involved in \emph{technai} such as medicine or architecture. Such knowledge is of a subject matter (\emph{mathēma}) that is distinct from it. So a physician who possesses the art of medicine has knowledge of a certain subject matter, health and disease, and the physician's knowledge is distinct from this subject matter. Knowledge may be knowledge of something, but Critias maintains that \emph{sōphrosunē} alone is knowledge of itself and other knowledges and their lack and of no other thing. This is the account that Socrates proposes to investigate. That investigation has two parts:
\begin{quotation}
	\textsc{Socrates}: Then only the \emph{sophron} person will know himself, and will be able to discern what he really knows and does not know, and have the power of judging what other people likewise know and think they know, in the cases where they do know, and again, what they think they know without knowing it; everyone else will be unable. And so this is being \emph{sophron}, or \emph{sōphrosunē}, and knowing oneself—that one should know what one knows and does not know. Is that what you mean?
	
	\textsc{Critias}: It is, he replied.
	
	\textsc{Socrates}: Once more then, I said, as our third offering to the Saviour, let us consider afresh, in the first place, whether such a thing as this is possible or not——to know that one knows, and does not know, what one knows and what one does not know; and secondly, if this is perfectly possible, what benefit we get by knowing it. (\emph{Charmides} 167a9–b4; \citealt[57]{Lamb:1927qw})
\end{quotation}
The first part of the investigation concerns the possibility of Critias' account of \emph{sōphrosunē}. Is the self-knowledge with which Critias identifies \emph{sōphrosunē} so much as possible? The second part of the investigation concernes the benefit of \emph{sōphrosunē}. It is, after all a virtue, and should benefit, somehow, the person who possesses \emph{sōphrosunē} and who acts \emph{sophron}. Given Critias' identification of \emph{sōphrosunē} with self-knowledge, the question becomes, what benefit accrues to the possession and use of such knowledge?
 
When Critias, in his long speech (164d4–165c4), proposes that \emph{sōphrosunē} is a kind of self-knowledge, Socrates first emphasizes, as we would put it, the intentional character of knowledge: If \emph{sōphrosunē} is knowing (\emph{gignõskein}), then it must be knowledge (\emph{epistēmē}) of something (\emph{tinos}). If \emph{sōphrosunē} is knowledge of something, it is natural to ask what is it knowledge of? The third offering is meant to be Critias' full answer to this question, albeit an answer that has been refined through Socratic examination. The third offering, then, is a specification of the object or intentional content of \emph{sōphrosunē}. (The third offering to the Savior, \emph{to triton tō sōtēri}, 167a9, is traditionally a ritual libation to Zeus \emph{sōtēr} on the third pour of a symposium, Aeschylus Fr. 55, Pindar \emph{Isthmian} 6.5, see also Plato \emph{Res Publica} 583b, \emph{Philebus} 66d, \emph{Leges} 3 692a, \emph{Epistolae} 7 334d, 340a, for the theological background, see \citealt{Cook:1914la} and  \citealt{Jim:2022ay}, and for the epithet, see \citealt{Rothrauff:1966nh}).

% The ritual dedication of Critias' account may be understood, and elaborated, along the lines described in the \emph{Timaeus} and the \emph{Critias}. It is reasonable to call upon God at the outset of an undertaking, and the Gods are invoked to pray for their approval of that undertaking (\emph{Timaeus} 27c1–7). This practical precept is in play here. Specifically, the divine invocation is Socrates' designating the account as the third offering to the Savior in the hope that it meets with His approval. And since the present undertaking is an account of \emph{sōphrosunē}, the invocation is a prayer (\emph{Critias} 106a1–b6) that the account may endure if true and meet with divine approval, but should the account be false, that a just penalty be imposed—the just penalty in the case of error being correction. Compare Socrates' fear of thinking that he knows something that he does not in fact know (\emph{Charmides} 166c7–d6). Finally, the invocation is a prayer for knowledge, the most perfect of medicines, so that one speak truly in the future. The theme of knowledge as medicine is resonant, if not indeed apt, given that Socrates allowed himself to be cast in the role of a Zalmoxian physician when first introduced to Charmides (156a9–c6). (For further discussion of the third offering see \citealt[203–4]{Lampert:2010xg} and \citealt[191–2]{Tsouna:2022aa}.)

According to the third offering, \emph{sōphrosunē} is knowledge (\emph{epistēmē}) of:
\begin{enumerate}[(1)]
	\item \textsc{Reflexive}: itself (\emph{autē heautēs} 166c3),
	\item \textsc{Higher-Order}: other knowledges,
	\item \textsc{Oppositional}: their absence (\emph{anepistēmosunēs} 167c2, 166e7–8),
	\item \textsc{Exclusive}: and no other thing
\end{enumerate}

Allow me to briefly comment on each of these.

(1) \textsc{Reflexive}: When Socrates points out that, on a previous account, one could be and act \emph{sophron} without knowing that one is \emph{sophron} (164c5–6), Critias pivots and identifies \emph{sōphrosunē} with self-knowledge (164d4–165c4). While it is perhaps uncontroversial that \emph{sōphrosunē} should involve self-knowledge, at least as an element, Critias is making the grander claim that this very thing, self-knowledge, just is \emph{sōphrosunē} (164d3–4, on the significance of this, see \citealt{Kosman:2014aa}). No doubt hoping to secure Socrates' assent (\citealt[23–4]{Tuckey:1951aa}, \citealt[81]{Hyland:1981aa}, \citealt[161–2]{Tsouna:2022aa}), Critias' invokes and interprets the Delphic inscription ``Know Thyself'' (\emph{Gnōthi sauton}) in developing his new account (compare \emph{Apologia Socratis} 21a–23b). As Critias develops this account under Socratic examination, \emph{sōphrosunē} is claimed to be knowing oneself (\emph{gignōskein heauton} 165b4), knowledge of oneself (\emph{epistēmē heatou} 165e1), and finally knowledge of itself (\emph{epistēmē autē heautēs} 166c3). Bracketing the slide from \emph{gnosis} to \emph{epistēmē} (which \citealt{Hyland:1981aa} and \citealt{Schmid:1998aa} regard as a poisoned chalice), one might reasonably query the move from knowing oneself to knowledge of itself (\citealt[33–7, 107–8]{Tuckey:1951aa})—after all, there has been an uncommented upon shift from a personal to an impersonal reflexivity. To be sure, in knowing oneself, what is known is not separate from the subject of such knowing, but that does not entail that what is known is the knowledge itself. Conversely, in possessing a knowledge which is knowledge of itself, does one really know oneself? Later Critias will answer this question in the affirmative (169d9–e5): Just as in possessing swiftness one is similar to it and so swift (a proleptic anticipation of the self-predicating nature of the Forms), when one possesses knowledge of itself one will be similar to it and so know oneself. Thus, the knowledge that constitutes \emph{sōphrosunē} is, in modern parlance, reflexive: Such knowledge takes as its object that very knowledge, at least in part.

(2) \textsc{Higher-Order}: Such knowledge is not only of itself, but it is also of other knowledges. When Socrates inquires into the content of \emph{sōphrosunē}—specifically what, according to Critias, it is knowledge of, he emphasizes that, in a range of familiar cases, what is known is distinct from the knowledge of them (166a3–7). In Sartre's terminology, the intentional object is transcendent in the sense that it goes beyond the conscious act that is directed upon it. Thus arithmetic (or perhaps calculating or reckoning, \emph{logistikē}) involves knowledge of the odd and the even and their quantitative relations where these are distinct from such knowledge (166a5–11). So too for weighing (\emph{statikē}) where the heavy and the light are distinct from the knowledge of them (166b1–4). Presumably the same holds for knowledge of medicine and architecture. What is known in each of these cases is distinct from the knowledge of it. This leads Socrates to ask what \emph{sōphrosunē} is knowledge of such that is distinct from this knowledge (166b5–6)? Critias responds that while every other form of knowledge is knowledge of something distinct from itself, \emph{sōphrosunē} is different—it alone is knowledge of these knowledges and of itself (166b9–c3). \emph{Sōphrosunē} is, according to Critias, a kind of sovereign knowledge, governing all other forms of knowledge. It is this sovereign status that that distinguishes \emph{sōphrosunē} from the other knowledges that it governs and justifies why it alone should take itself as an object and so depart from the pattern displayed by subordinate forms of knowledge. Thus, the content of the knowledge that constitutes \emph{sōphrosunē} is, in modern parlance, higher-order: Such knowledge takes as its object other knowledges, at least in part.

(3) \textsc{Oppositional}: Critias having characterized \emph{sōphrosunē} as knowledge which alone is of itself and other knowledges, Socrates suggests a refinement that Critias readily accepts. If \emph{sōphrosunē} is knowledge of other knowledges it must also be of their lack (\emph{anepistēmosunēs}). Later, knowledge of good will be claimed to involve knowledge of its opposite, evil (174b9–c3). This suggests that a more general conception of knowledge may be in play here. On this conception, knowledge is, or at least involves, a discriminatory power. To know a thing one must be able to discriminate it from its opposite.  Thus a physician in possessing knowledge of medicine has the power to discriminate health from its opposite, disease. (Further evidence for this conception can be found in \emph{Phaedo} 97d1-5, \emph{Res Publica} 333d–334a. Aristotle will take up and develop this Academic conception in \emph{Topica} 105b5, 110b20, 155b30-34, 164a1 and \emph{Metaphysica} {\sbl Θ} 2). So in the present case, the \emph{sophron} would have the power to discriminate knowledge from its opposite, ignorance. Socrates understands this discriminatory power discursively. It involves being able to test (\emph{exetasai} 167a2) what one knows (\emph{eidōs} 167a3) and does not know and the power to examine (\emph{episkopein} 167a3) what others know and do not know and this occurs in the medium of conversation. The discursive dimension of this discriminatory capacity means that knowledge of other knowledges and their lack has first- and third-personal aspects (the perspectives of the speaker and their conversational participants, respectively). Though Critias readily accepts the Socratic refinement, he perhaps understands its significance differently. A \emph{sophron} ruler will not only know what they know and do not know, but they will also know what their subordinates know and do not know and so will be able to assign them appropriate responsibilities in the running of the city. So the sovereign knowledge is the knowledge of a \emph{sophron} sovereign. Thus, the content of the knowledge that constitutes \emph{sōphrosunē} involves not only knowledge but importantly its lack. 

(4) \textsc{Exclusive}: Critias claims that (a) \emph{sōphrosunē} alone (\emph{monē} 166c2, 166e5) is of itself and other knowledges. By contrast, (b) all other knowledges are not of themselves or other knowledges but rather have proprietary objects or subject matters that are distinct from such knowledge. Later (167b10–c2), Socrates will add a further element: that (c) \emph{sōphrosunē} is knowledge of itself and other knowledges and their lack and no other thing (\emph{ouk allou tinos} 167b11). So the content of \emph{sōphrosunē} is exhausted by these, having no further aspect to its subject matter. Just as the previous feature—that the content of \emph{sōphrosunē} should include the absence of knowledge—is a Socratic refinement, so too is the present feature. So far, Critias has only explicitly claimed that \emph{sōphrosunē} is alone in being knowledge of itself and other knowledges and has acceded to Socrates' claim that it is also of the absence of knowledge. Now it is being claimed to be exclusively of these objects (Perhaps Socrates has introduced a novel idea that Critias fails to notice, \citealt[37–8]{Duncombe:2020gi}, or perhaps he is merely making explicit what was implicit in Critias' account, \citealt[]{Tsouna:2022aa}). (a)–(c) has an unstated implication that will be made fully explicit in The Argument from Benefit (172b–175b) but begins to emerge in Socrates' pressing puzzling disanalogies to Critias' account (discussed in section~\ref{sec:puzzling_disanalogies}). If \emph{sōphrosunē} alone is of itself and other knowledges and their lack, and this exhausts the content of \emph{sōphrosunē}, then \emph{sōphrosunē} will not have as part of its content the proprietary objects of the other knowledges. So \emph{sōphrosunē} is intransitive \citep{McCabe:2007ss} or nontransparent \citep[190]{Tsouna:2022aa}. \emph{Sōphrosunē} may take the other knowledges as its objects but it does not, in turn, take the objects of these other knowledges as its own.Thus, for example, while medicine may be among the knowledges known, \emph{sōphrosunē} does not take health and disease, the proprietary subject matter of medicine, as among its objects. Thus, the content of the knowledge that constitutes \emph{sōphrosunē} is exhausted by itself, other knowledges, and their lack.

The commitment to intransitivity or nontransparency is a puzzling result. It occasions the characteristically Sartrean complaint that an element of opacity has been introduced into consciousness (``Consciousness would cease being transparent to itself; its unity would be broken in every direction by unassimilable, opaque screens,'' \citealt[6]{Sartre:1948aa}). While one may reasonably know that another possesses knowledge that one lacks—say, if that knowledge is manifest in successful action of which one is incapable, this is harder to maintain in the first-person case. How can one know that one knows without thereby knowing what one knows? Perhaps knowing something that is presently difficult to recall would be such a case. One would know something without knowing what one knows in the sense of not being able to recall it. Though if one cannot regularly recall what one claims to know, or can only recall it with difficulty, then, at least in certain practical circumstances, there is pressure to withdraw the claim that one knows. But then it would no longer be a case of knowing that one knows without knowing what one knows. It would simply be a case of not knowing, or not knowing sufficiently.

% \citet[44]{Tuckey:1951aa} observes an ambiguity, present in both the Greek and the English, that raises an issue that will be addressed throughout the remainder of this essay, specifically whether knowledge of knowledge ``refers to possibility a \emph{particular act} of knowledge being its own object, or to the possibility of knowledge \emph{in general} being the object of knowledge''. Whereas \citet[772–3]{caston02} argues for the activity reading, \citet[218]{Tuozzo:2011aa} argues instead for a power reading: ``the knowledge of medicine, say, is most naturally construed as a disposition'', so not the activity which is the exercise of a power, but the power itself. The puzzling disanologies will begin to shed light on this issue.

% section the_third_offering (end)

\section{Puzzling Disanalogies} % (fold)
\label{sec:puzzling_disanalogies}

Examining the identification of \emph{sōphrosunē} with a form of self-knowledge occasions Critias' charge that Socrates is engaging in eristic refutation (166c3–6). Perhaps, in the background, they are working with different conceptions of dialectical reasoning (for a reading of this contrast see \citealt[chapter 4]{Schmid:1998aa}), and it is this that gives rise to Critias' misunderstanding of Socrates' motives. At any rate, Critias entered the conversation defensively, in the spirit of competition and desirous of victory (\emph{agōniōn kai philotimōs} 162c1–2), angry that Charmides has bungled the defense of an idea originating with Critias (just ``as a poet does with an actor who mishandles his verses'' 162d2–3; \citealt[41]{Lamb:1927qw}). Perhaps Socrates feels the need to moderate Critias' intemperance (ironic, given the topic of conversation) for he provides an explicit rationale for examining Critias' identification of \emph{sōphrosunē} with a form of self-knowledge. That rationale will consist in a certain puzzlement or \emph{aporia} that arises when Critias' account is applied to a range of more familiar cases.

Having offered Critias' account to the Savior, Socrates motivates its examination:
\begin{quotation}
	\textsc{Socrates}: Come then, I said, Critias, consider if you can show yourself any more resourceful than I am; for I am puzzled (\emph{aporō}). Shall I explain to you in what way?
	
	\textsc{Critias}: By all means, he replied.
	
	\textsc{Socrates}: Well, I said, what all this comes to, if your last statement was correct, it is merely that there is one knowledge which is precisely a knowledge of itself and the other knowledges, and moreover is a knowledge of the lack of knowledge at the same time.
	
	\textsc{Critias}: Certainly.
	
	\textsc{Socrates}: Then mark what a strange (\emph{atopon}) statement it is that we are attempting to make, my friend: for if you will consider it as applied to other cases, you will surely see—so I believe—its impossibility (\emph{adunaton}). (\emph{Charmides} 167b–sc; \citealt[57]{Lamb:1927qw}, modified)
\end{quotation}

Socrates is puzzled and explains that this is due to the strangeness (\emph{atopon}) of  Critias' account. It is not of this place and so ``strange'' in the sense of ``a stranger in a strange land'', \emph{Exodus} 2:22 \textsc{kjv}, and so foreign rather than absurd. His account is strange because it is unlike other more familiar cases. In these cases, the application of Critias' account to them results in a manifest impossibility (\emph{adunaton}). 

Talk of \emph{atopon} may be rhetorically significant. Critias had earlier maintained that \emph{sōphrosunē} is alone (\emph{monē} 166c2, 166e5) knowledge of itself and other knowledges. The self-knowledge that is \emph{sōphrosunē} is sovereign in that it alone governs other knowledges, and so it alone has the sovereign prerogative of knowing itself. Moreover this sovereign knowledge is itself the knowledge of a \emph{sophron} sovereign. In casting this \emph{sophron} sovereign as a stranger, alone in being unlike what is familiar among the citizenry, a potential political criticism is intimated. The \emph{sophron} sovereign, as Critias understands them, is alienated. And a sovereign's alienation from their citizenry risks political instability, consider the democratic insurrection against the Thirty Tyrants that ultimately led to Critias' death. This sense of political instability conveyed to Plato's contemporaries for whom this political disaster was a recent historical memory is an image of Socrates' puzzlement at Critias' strange sovereign knowledge.

Notice that, strictly speaking, what is claimed to be impossible is not Critias' account but its application to other cases. Socrates' puzzlement, here, does not so much as cast doubt on Critias' account, in the sense of providing a positive reason, however provisional, for rejecting that account, as it is an invitation to further inquiry (here, and in this respect, I am in agreement with \citealt{Politis:2008nv}). It is likely that Critias understands Socrates puzzlement in this way. For Critias has earlier charged Socrates with eristic refutation:
\begin{quote}
	\textsc{Critias}: There you are, Socrates, he said: you push your investigation up to the real question at issue—in what \emph{sōphrosunē} differs from all the other knowledges—but you then proceed to seek some resemblance between it and them; whereas there is no such thing. (\emph{Charmides} 166b7–c1; \citealt[53]{Lamb:1927qw}, modified.)
\end{quote}	
Were Socrates pressing the puzzling disanalogies as a reason to reject the account, Critias would have good grounds to revive this complaint in a way that he declines to do. So it is neither an enthymeme (Aristotle, \emph{Rhetorica} 1402b15) nor an epagogic argument as many commentators maintain (see, for example, \citealt[41]{Robinson:1941yb}), but rather provides a motive for further inquiry (the results of which are the conclusion of The Argument from Relatives). Indeed, at the end of this discussion, Socrates makes this point explicitly:
\begin{quote}
	\textsc{Socrates}: And it is a strange thing, if it really exists? For we should not affirm as yet it does not exist, but we should consider whether it does exist. (\emph{Charmides} 168a10–1; \citealt[61]{Lamb:1927qw})
\end{quote}
We should not yet affirm that Critias strange knowledge exists but nor should we deny its existence. Should we deny its existence, considering whether it exists would be pointless. So the puzzling disanalogies do not establish that Critias' strange knowledge does not exist, but it does motivate an examination into whether it does in fact exist. Critias, mollified, agrees (168a1).

\subsection{Taxonomy} % (fold)
\label{sub:taxonomy}

The disanalogies fall into three groups. There are perceptual, conative, and cognitive cases:
\begin{enumerate}[(1)]
	\item \textsc{Perceptual}: sight (\emph{opsis}), hearing (\emph{akounē}), and the senses all together (\emph{peri pasōn tōn aisthēseon}) (167c8–d10)
	\item \textsc{Conative}: appetite (\emph{epithumia}), wish (\emph{boulēsis}), love (\emph{eros}), and fear (\emph{phobos}) (167e1–168a2)
	\item \textsc{Cognitive}: opinion (\emph{doxan}) (168a3–5)
\end{enumerate}
(Compare the taxonomies of \citealt[114–8]{Hyland:1981aa} and \citealt[90]{Schmid:1998aa}. Ignoring the grouping induced by ordering, \citealt[207 n37]{Tsouna:2022aa}, claims that there is no textual evidence for any such taxonomy, but Tsouna's real complaint is that any such taxonomy plays no role in the argument. However, this last thought is vitiated by the disanalogies not constituting an argument against Critias' account but rather providing a reason to examine it.) 

All of these cases either are or involve psychic powers. Only living beings have perceptual, conative, and cognitive powers. And only living beings exercise such powers. So though Socrates never explicitly claims that these powers are powers of the soul (\emph{psuchē}), insofar as the soul is the principle of life, only beings with souls have such powers. 

Not only do such cases involve psychic powers, but all such powers are intentional in the minimal sense that they take an object. In exercising these powers there is something that is seen, heard, or perceived more generally. There is something desired, wished, loved, or feared. There is something about which one has an opinion. And knowledge, generally, and the self-knowledge with which Critias identifies \emph{sōphrosunē}, specifically, are themselves intentional. 

Moreover, at least in a range of familiar cases, what is known is distinct from the knowing of it. And this holds more generally, at least of familiar intentional psychic powers. What is seen is distinct from the seeing of it. What is heard is distinct from the hearing of it. What is desired, wished, loved, and feared are distinct from the desiring, wishing, loving, and fearing. Or so Socrates maintains. So not only are the disanalogies all psychic powers, not only are they all intentional, but they are all such that their intentional object transcends the conscious act directed upon it.

% subsection taxonomy (end)

\subsection{Perception} % (fold)
\label{sub:perception}

Perceptual cases inaugurate the puzzling disanalogies. Critias' account as applied to vision, audition, and the senses, more generally, results in a manifest impossibility.

The first case, sight, makes explicit the parallels with Critias' account:
\begin{quote}
	\textsc{Socrates}: Ask yourself if you think there is a sort of vision which is not the vision of things that we see in the ordinary way, but a vision of itself and of the other sorts of visions, and of the lack of vision likewise; which, while being vision, sees no colour, but only of itself and the other sorts of vision. (\emph{Charmides} 167c–d; \citealt[59]{Lamb:1927qw})
\end{quote}
The hypothetical form of vision closely, if not perfectly, parallels the self-knowledge with which Critias identifies \emph{sōphrosunē}. Consider then the four features of that self-knowledge described in section~\ref{sec:the_third_offering}. The hypothetical form of vision is of:
\begin{enumerate}[(1)]
	\item \textsc{Reflexive}: itself (\emph{heautēs} 167c9)
	\item \textsc{Higher-Order}: other visions
	\item \textsc{Oppositional}: their absence (\emph{mē opseōn} 167c10)
	\item \textsc{Exclusive}: and no other thing (or at least not of the things we see in the ordinary way, namely color)
\end{enumerate}
And presumably, like the self-knowledge with which Critias identifies \emph{sōphrosunē}, it has first- and third-personal aspects. That is to say, that it is a vision of the visions of the perceiver as well as a vision of the visions of other perceivers. This latter is not made explicit in the argument or \emph{logos} but figures prominently in the drama or \emph{ergon}, for example, Socrates seeing Charmides' look at 155b8–c1). So the envisioned visions may be the visions of the perceiver or of other perceivers.

So far so similar. But there are differences as well.

First, Socrates begins to make explicit an implicit commitment of the third offering. We observed that if \emph{sōphrosunē} alone is of itself and other knowledges and their lack, and this exhausts the content of \emph{sōphrosunē}, then \emph{sōphrosunē} will not have as part of its content the proprietary objects of the other knowledges. This implication is made explicit with the hypothetical form of vision: It is not of things that we see in the ordinary way. So the hypothetical vision is intransitive or nontransparent. It may take other visions as its objects but it does not, in turn, take the objects of these other visions as its own object. The vision of itself that takes the other sorts of vision as an object does not see through them to colorful scenes that they disclose. An element of opacity has been introduced into visual consciousness.

Second, and relatedly, ordinary vision, unlike the hypothetical vision, is of colors. As Socrates applies Critias' account to other more familiar cases, the relevant psychic powers, be they perceptual, conative, or cognitive, each take their proper objects (if I may help myself to this Peripatetic anachronism). Thus vision is of colors, and colors are the objects of no other sense. One can neither hear colors nor feel them (despite the claims of certain psychics to be color-feelers, \citealt{Duplessis:1975aa}). It is in this sense that color is the proper object of sight (I prescind here from the further Peripatetic thought that perceiving proper objects are immune to error, see \citealt[chapter 4.2]{Kalderon:2015fr}, and \citealt{Johnstone:2015aa} for discussion). Nor are colors the proper objects of the conative or cognitive powers. We may have opinions about the colors of things (the color of the \emph{peplops} adorning the statue of Athena during the Panathenea, say), but the opinable and not the colors discursively articulated in the opinable are the proper object of \emph{doxa} (for a contemporary discussion of this point see \citealt[section 5]{Kalderon:2011fk}). 

The terminology may be Peripatetic, but the claim that ordinary sensory powers have proper objects, in the sense of objects that are perceptible to one sense alone, is genuinely Platonic. Consider the following passage from the \emph{Theaetetus}:
\begin{quotation}
	\textsc{Socrates}: And are you also willing to admit that what you perceive through one power, you can't perceive through another? For instance, what you perceive through hearing, you couldn't perceive through sight, and similarly what you perceive through sight you couldn't perceive through hearing?
	
	\textsc{Theaetetus}: I could hardly refuse to grant that. (\emph{Theaetetus} 184e8–185a3; Levett and Burnyeat in \citealt[204]{Cooper:1997fk})
\end{quotation}
Notice that Plato links objects being perceptible to one sense alone to a conception of the senses as powers. Two thoughts seem to be at work here: that powers are individuated by their proper exercise, and that the proper exercise of a sensory power is the presentation of its proper object in sensory awareness. These two claims in conjunction with specific claims about the proper objects of vision and audition imply that sight just is the capacity to see color, and audition just is the capacity to hear sound. 

Moreover, Plato is willing to generalize these claims to other psychic powers. They figure, for example, in an argument that knowledge and opinion are distinct powers (\emph{Res Publica} 5 477--478). And in the \emph{Charmides}, we see that for each of the ordinary forms of the relevant powers, Socrates specifies a proper object:
\begin{enumerate}[(1)]
	\item sight (\emph{opsis}): color (\emph{chrōma} 167d1)
	\item hearing (\emph{akounē}): sound (\emph{phōnēs} 167d4)
	\item the senses all together (\emph{peri pasōn tōn aisthēseon}): the sensible (\emph{aisthanmenē} 167d9, though as we shall see the generality raises an issue here)
	\item appetite (\emph{epithumia}): pleasure (\emph{hēdonēs} 167e1)
	\item wish (\emph{boulēsis}): good (\emph{agathon} 167e4)
	\item love (\emph{eros}): beauty (\emph{kalou} 167e8)
	\item fear (\emph{phobos}): the dreadful (\emph{deinōn} 168a1)
	\item opinion (\emph{doxa}): the opinable
\end{enumerate}

This general claim, that ordinary psychic powers take proper objects, is not to be confused with a similar claim previously encountered. Specifically, ordinary branches of knowledge are distinguished by their proprietary objects or subject matters. Thus medicine is of health and disease just as arithmetic is of the odd and the even. But the proper object of \emph{epistēmē} is \emph{mathēma} (the knowable or, more literally, the lesson learned). It is this object that distinguishes \emph{epistēmē} from \emph{doxa}. An opinion is not a lesson learned. Health and disease, no less than the odd and the even, at least when discursively articulated and organized into a science, are distinct species of the epistemic genus \emph{mathēma}.

The second perceptual case, hearing, closely follows this pattern:
\begin{quotation}
	\textsc{Socrates}: And what do you say to a sort of hearing which hears not a single sound, but hears itself and the other sorts of hearing and lack of hearing
	
	\textsc{Critias}: I reject that also. (167d4–6; \citealt[59]{Lamb:1927qw})
\end{quotation}
The hypothetical form of hearing is of:
\begin{enumerate}[(1)]
	\item \textsc{Reflexive}; itself (\emph{autēs} 167d4)
	\item \textsc{Higher-Order}: other hearings
	\item \textsc{Oppositional}: their absence (\emph{tōn mē akoōn} 167d5)
	\item \textsc{Exclusive}: and no other thing (or at least no other thing that we hear in the ordinary way, namely, sound)
\end{enumerate}
Just as color is the proper object of vision, sound is plausibly the proper object of audition (though see \citealt[chapter 4.2]{Kalderon:2018oe} for criticism). And since sound is what ordinary hearings hear, the hearing of these hearings is intransitive or nontransparent. Since hearing itself and other hearings involves hearing no sound, one does not hear through the hearings to their proper objects. An element of opacity has been introduced into auditory consciousness (though perhaps ``resonant interference'' should be substituted for Sartre's ``opacity'' in deference to the auditory nature of the case).

The third perceptual case is difficult to interpret:
\begin{quotation}
	\textsc{Socrates}: Then take all the senses together as a whole, and consider if you think there is any sense of the senses and of itself, but insensible of any of the things of which the other senses are sensible?
	
	\textsc{Critias}: I do not. (\emph{Charmides} 167d7–10; \citealt[59]{Lamb:1927qw})
\end{quotation}

There are at least three general ways to understand this passage (167d4–6): 
\begin{enumerate}[(1)]
	\item On the first reading, having discussed vision and audition, instead of enumerating the rest of senses and applying Critias' account to each, Socrates, in effect, says ``and so on for the rest of the senses'' (\citealt[113–4]{Bloch:1973aa} and \citealt[89]{Schmid:1998aa}). Thus, for example, what is deemed impossible is a sense of smell that smells itself and other smellings but does not smell what other smellings smell, and so on for all the other senses such as taste and touch. So understood, the passage describes the general application of Critias' account to each of the senses. 
	\item On the second reading, the impossibility does not pertain to Critias' account as applied to the rest of the senses. Rather, among all the senses, there is a special sense that takes itself and the ordinary senses as objects, but does not sense what the ordinary senses sense (\citealt[214–5]{Tuozzo:2011aa}).  And that is what is deemed impossible. So understood, this passage is a proleptic anticipation of, and perhaps inspiration for, Aristotle's notion of \emph{koinē aisthēsis} (as it occurs in \emph{De Anima} and \emph{Parva Naturalia}). It at least anticipates one of the many functions that scholars have attributed to this Peripatetic psychic power. 
	\item On the third reading, there is a deliberate indeterminacy to this passage. In effect, it can be read as an invitation to reflect on all the ways that Critias' account might be applied to the senses generally (\citealt[202]{Tsouna:2022aa}). So understood, the first two readings are merely alternatives to be considered in further discussion of these matters.
\end{enumerate}

The third perceptual case does not perfectly parallel Critias' account of \emph{sōphrosunē}. How exactly it departs from that account depends upon how our passage (167d4–6) is best interpreted.

First, the initial Socratic refinement goes unmentioned. While the sense is of itself and the other senses, no mention is made of it also being of their lack. At least on the first reading, the Socratic refinement is intelligible, and it is open to understand Socrates' query as elliptical, the refinement not explicitly stated but implicitly understood. So what would be impossible, among other things, is a sense of smell that does not smell what other smellings smell but only itself and other smellings and their lack. But on the second reading, where Socrates is proleptically anticipating \emph{koinē aisthēsis}, this is harder to maintain. Can the perceiver sense the absence of sensing? Perhaps in the third-person cases—such as seeing another's lack of seeing. But does that make sense in the first-person case—a perceiver's sensing their lack of visual access to the surrounding scene? What exactly are we imagining their experience to be like? Are we to imagine them as conscious while their senses are inoperative, like Ibn Sina's Flying Man (\emph{al-Nafs} 1.1, 5.7)? But unlike in Ibn Sina's case, this residual consciousness is meant to be sensory, making the conceivability task all the more difficult, if indeed possible at all.

Second, it is unclear whether the sensible is, or even could be, a proper object. On the first reading, the sensible is not itself a proper object but is rather a generic formal description of the proper objects of the ordinary senses. Talk of the sensible, so understood, is merely a device of generality that ranges over colors, sounds, smells, and all the other proper objects of perception. It may be possible on the second reading that it is a proper object, if the sensible is a genus of which the proper objects of the ordinary senses are species, but this is speculative insofar as it lacks a firm textual basis.

Do the perceptual cases canvassed by Socrates refer to perceptual powers or perceptual activities that are their exercise? On the former reading, Socrates would be asking where there is a power of sight that takes itself and other powers of sight and their absence as objects. On the latter reading, Socrates would be asking whether there is a seeing that sees itself and other seeings and their absence. An alert reader will notice that the discussion so far has quietly assumed the activity reading.

The power reading is supported by the fact that \emph{opsis} (sight), \emph{akoē} (hearing), and \emph{aisthēsis} (perception) typically, if not invariably, refer to perceptual powers rather than their episodic exercise. And against \citet[772–3]{caston02}, \citet[218 n18]{Tuozzo:2011aa} argues that the perceptual cases are meant to be analogous to \emph{epistēmē}, and since knowledge involves, or is constituted by, the possession of a power, then it would be reasonable to understand sight, hearing, and the senses taken altogether as themselves powers (for further criticism of Caston see \citealt{Johansen:2005hz}). 

The activities reading is supported by the plural form used in the formulations—sight of sights (167c10), hearing of hearings (167d4–5), sense of senses (167d8). Though \emph{opsis}, \emph{akoē}, and \emph{aisthēsis} are typically used for perceptual powers, they can be used to designate the activities of these powers, and their plural occurrences strongly speak in favor of the activity reading. It is ordinary seeings and hearings that are the objects of the hypothetical form of vision and audition. Holding fast to the powers reading, the plural would force us to attribute different kinds of powers of sight and audition respectively, but that is implausible in this context. 

To get a sense of this, consider Tuozzo's interpretation of the different kinds of perceptual powers:
\begin{quote}
	It would be equally possible to construe first-order seeings as dispositions: the seeing of blue is a disposition that needs special circumstances for becoming occurrent, among them (typically) the presence of something blue. \citep[219]{Tuozzo:2011aa}
\end{quote}
So corresponding to a seeing of a color, there is the power, not just of sight, but the power to see that very thing in the given perceptual circumstances. And the suggestion is that the plurality of powers are simply these more specific powers actualized in seeing. Recall the plurality is a plurality of kinds of powers. So the kinds of powers are being individuated, in part, by the kinds of things they present when actualized. In the cases of vision and audition, what is presented are their respective proper objects, color and sound. And so the proper objects must themselves admit of division into kinds. 

This might seem plausible since, elsewhere, Plato provides a taxonomy of color. In the \emph{Timaeus} (67c4–68d7), there are four unmixed colors: white (\emph{leukos}), bright (\emph{lampros}) or brilliant (\emph{stilbos}), red (\emph{eruthros}), and black (\emph{melas}). And there are nine mixed colors that result when the unmixed colors are combined in certain proportions (\emph{Timaeus} 68b5–c7): golden (\emph{zanthos}), purple (\emph{alourgos}), violet (\emph{orphninos}), tawny (\emph{purros}), gray (\emph{phaios}), yellow (\emph{ōchros}), dark blue (\emph{kuaneos}), light blue (\emph{glaukos}), and leek green (\emph{prasinos}). (For discussion see \citealt{James:1996pb}, \citealt{Ierodiakonou:2005ly}, and \citealt{Kalderon:2022kl}.) On Timaeus' account, these would be the kinds of colors, and the corresponding visual powers—the power to see \emph{purros}, say—would be limited to the kinds of proper objects that individuate them. And this raises a worry. Notice that on Timaeus' account, there is no place for seeing blue, since blue crosscuts the distinct kinds of colors, \emph{kuaneos} and \emph{glaukos}. The more serious points is that there may be more seeings than kinds of colors. Even the mixed colors admit of discriminable shades. But these determinate chromatic shades are not kinds of colors, or at least they are not explanatory kinds. If anything they are the \emph{explanandum} not the \emph{explananda}.

As to Tuozzo's complaint against Caston, while contemporary orthodoxy may hold that knowledge is stative as opposed to episodic, knowledge may be spoken of in many ways. If the action potential of knowledge is never actualized, or actualized only with difficulty, then there is some pressure to withdraw the knowledge attribution, at least on some understanding of knowledge relevant to the practical circumstances. A band leader in auditions asks ``Do you know `Billy's Bounce'?'' One musician answers ``Yes'' and immediately plays the head with good phrasing and rhythm. Another answers ``Yes'' and begins to recollect the head, ``Let' see, it's an F blues\ldots How does it go again? \ldots Oh yeah, fifth, um, sharp eleven, fifth, and then root, flat third, third, root, six, I got this\ldots'' It would not be unreasonable for the band leader to conclude that the latter does not know the tune, despite being able to recollect it with effort, or at least does not know it well enough, and should be summarily sent back to the shed. There is an understanding of knowledge relevant to the practical circumstances that makes this so. The former knows the tune in a way that the latter does not as evinced by their mastery of it. 

Aristotle makes this thought explicit in \emph{De Anima} 2.5 417a22–b1. He contrasts an educable person ignorant of a point of grammar with their having learned that point of grammar. Since they were educable, they had the power to come to know through learning. In learning, this power is exercised, and they actually become a knower. Aristotle calls this the first actuality. However, Aristotle also maintains that the first actuality is, at the same time, a second potentiality, at least in the traditional post-Aristotelean vocabulary. In learning that point of grammar, the person now has the power to apply that grammatical point in a variety of contexts. Knowledge may be stative, on some relevant understanding, but it constitutes action potential that may be actualized in a variety of practical circumstances. In some contexts, it is the masterful actualization of this potential that counts as knowledge. Tuozzo, in effect overlooks Aristotle's insight that a first actuality may also be a second potentiality.

Suppose that knowledge and actuality are linked. Suppose, further, that they are linked in that part of what it is to be knowledge is for it to be actual, in some relevant sense. That the possession of knowledge is the realization of epistemic development is a way in which knowledge and actuality are linked. The actual and the potential are spoken of in many ways (\emph{Metaphysica} {\sbl Δ} 12, {\sbl Θ} 1). So what counts as actual can vary in different practical circumstances. Thus if knowledge is linked with actuality, knowledge attributions would vary as well. In the practical circumstances that make salient the first actuality of knowledge (for example when the learned is contrasted with the ignorant if educable), knowledge is attributable. But in practical circumstances that make salient the second actuality, the actualization of the action potential of knowledge (for example when our two auditioning musicians are contrasted), there is some pressure to withdraw the knowledge attribution if the power is not exercised, or irregularly exercised with difficulty. Since in these circumstances knowledge is restricted to what can actually be acted upon, this too would be a way in which knowledge and actuality are linked. So knowledge being spoken of in many ways is inherited, at least in part, from the actual and the potential being spoken of in many ways, given the link between knowledge and actuality.

The point may be taken, but its relevance may be questioned. Why attribute Peripatetic second potentiality to Plato? What grounds might there be for its proleptic anticipation in the \emph{Charmides}? If there is no good answer to these questions, then Caston's activity reading cannot be so defended. However, there are textual grounds for at least the beginning of an answer to the second question. The accounts of \emph{sōphrosunē} as doing one's own things (161b4–162b11, 162c1–163c8) and doing good things (163d1–164c6) may have been rejected, but that \emph{sōphrosunē} is a power to act \emph{sophron} has never been questioned. So once \emph{sōphrosunē} is identified with a kind of self-knowledge, that knowledge must constitute the power to act \emph{sophron}. But then, in the traditional post-Aristotelian vocabulary, this knowledge would be a second potentiality and the \emph{sophron} actions that it gives rise to would be second actualities. \emph{Sophron} action would the masterful realization of the self-knowledge that constitutes \emph{sōphrosunē}. 

The motive for the initial Socratic refinement—that \emph{sōphrosunē} must be knowledge of knowledge and, importantly, its lack—may provide further grounds. I suggested that Socrates proposes this refinement against the background of a more general conception of knowledge as a discriminatory power (section~\ref{sec:the_third_offering}). To know a thing one must be able to discriminate it from its opposite. So to know knowledge, one must be able to discriminate knowledge (\emph{epistēmē}) from its opposite, ignorance (\emph{anepistēmosunēs} 167c2, 166e7–8). This power is actualized in discriminatory activity. For Socrates, this takes place in the medium of conversation where claims to know are tested (\emph{exetasai} 167a2) or examined (\emph{episkopein} 167a3). But then, in the traditional post-Aristotelian vocabulary, this knowledge would be a second potentiality and the discriminatory activity that it gives rise to would be second actualities.

Socrates does not deploy the traditional post-Aristotelian categories. How could he? Nor does he articulate or otherwise imply the semantic insight that motivates Aristotle to mark these distinctions: that the actual and potential are spoken of in many ways (\emph{Metaphysica} {\sbl Δ} 12, {\sbl Θ} 1; for discussions see \citealt{Shields:2002jz}). Nevertheless, \emph{sōphrosunē} and knowledge, though achievements of a mature, adult, human being, and so, in a sense, the realization of moral and epistemic development, remain powers to act \emph{sophron} and to discriminate the object of knowledge from its opposite, respectively, if not in more ways besides. 

% subsection perception (end)

\subsection{\emph{Adunaton}} % (fold)
\label{sub:_emph_adunaton}

The application of Critias' account to the perceptual, conative, and cognitive cases results in impossibility (\emph{adunaton}). Socrates and Critias come to an agreement about this. But what is it about the application of Critias' account that results in impossibility? For recall that there are separable components to that account. (1) Is it the reflexive character of these psychic powers? That their exercise takes itself, that very activity, as its object? (2) Or is it their high-order character, that the exercise of these psychic powers takes, as among their objects, the exercise of their more ordinary counterparts? (3) Is it that the exercise of these psychic powers take, as among their objects, the absence of the exercise of the more ordinary counterparts? (4) Or is it the restriction of the content of \emph{sōphrosunē} to these, with its commitment to intransitivity or non-transparency, that is deemed impossible? Neither Socrates nor Critias say, explicitly, which aspect of his account as applied to more ordinary psychic powers leads to impossibility. What might Plato have in mind here?

Perhaps, considered in the dialectical context, this lack of explicitness is explicable. Recall, Critias' knowledge is strange, if it really exists, since it would be unlike the application of his account to other intentional psychic powers, the results being impossible. Each of the four features of Critias' account, if pinned as the culprit, would result in Critias' own account of \emph{sōphrosunē} being impossible as well. However, Socrates concludes that while we should not yet affirm that Critias' strange knowledge exists, we should instead consider whether it does exist (168a10–1). On this reading, it is the lack of explicitness that invites further inquiry, at least in part, and this is what makes it explicable.

This reading is perhaps obscured by the widespread tendency to understand the disanalogies as constituting an argument against the third offering—The Argument from Analogy—rather than establishing a motive to examine the third offering. If the disanalogies were an epagogic argument against Critias' strange knowledge, then the fact that one of the features of Critias' strange knowledge would be the source of the impossibility would be no embarrassment (though, as we observed, Critias would have grounds to revive his charge of eristic refutation, 166b7–c1). But the fact that Socrates explicitly claims that the disanalogies invite further inquiry, rather than casting doubt on the existence of Critias' strange knowledge itself, at the very least defers any such skepticism. The further inquiry must first be undertaken.

Let us briefly canvass an alternative. While subject to the  previous criticisms, it raises, however, an issue that will be relevant to our understanding of The Argument from Relatives.

The alternative begins with Socrates' emphasis on the proper objects of the ordinary psychic powers. Thus color is the proper object of vision just as sound is the proper object of audition. A related denial is claimed to be the source of the impossibility. First, though, consider the result of applying Critias' account of \emph{sōphrosunē} to these psychic powers—for example, a seeing that sees itself and other seeings and their lack. It is the denial that this hypothetical form of vision sees color that is deemed impossible. Or consider the result of applying Critias' account to audition—a hearing that hears itself and other hearings and their lack. It is the denial that this hypothetical form of audition hears sound that is deemed impossible. It is the failure of the hypothetical analogues of \emph{sōphrosunē} to take the proper objects of their correlative psychic powers, and not their reflexivity, that is the source of the impossibility (compare \citealt[212–4]{Tuozzo:2011aa}). It is the consequent intransitivity or nontransparency, the intrusion of opacity into sensory consciousness, that Critias finds incredible.

But can the intended contrast with reflexivity be sustained? The seeing that sees itself and other seeings but not what these others see, namely color, may be impossible. But is it really intransitivity or nontransparency, as troubling as these may be, that is the source of the impossibility? Or is it rather that if neither sight nor seeing are colored, and sight is only of colored things, then there is no seeing of itself since seeing, while of the colored, is not itself colored? This too is consistent with the emphasis on the proper objects of the intentional psychic powers, but the culprit here is not intransitivity or nontransparency but reflexivity.

The issue, so far formulated, comes to this: Is it that the seeing that sees no color could be no seeing? Or is it, rather, that sight and seeing, not being colored, could not be seen? So starkly put, one may wonder if there is a substantive issue here. If the seeing of sight and seeing is not possible, then sight and seeing are unseen. And if it is not possible for sight and seeing to be seen, then there is no seeing of sight and seeing. These can seem like equivalent descriptions (in the sense of Putnam's generalization of Reichenbach's notion).

% The second alternative pins the blame instead on reflexivity.



% subsection _emph_adunaton (end)

% \citet[214–7]{Tuozzo:2011aa} observes that while some of the proper objects are given a substantive specification others are merely given a formal characterization. The case of vision nicely illustrates this contrast since it begins with the formal characterization of the proper object of vision, what sight is the seeing of, but later provides a substantive specification of this, namely color. The proper objects of vision, hearing, appetite, wish, love, and fear are all substantively specified. However, in the cases of the senses all together, opinion, and knowledge, we are merely given a formal characterization of their proper objects.


% section puzzling_disanalogies (end)

\section{The Argument from Relatives} % (fold)
\label{sec:the_argument_from_relatives}

The discussion of the puzzling disanalogies may not blame any specific feature of Critias' account of \emph{sōphrosunē} for the impossibility (\emph{adunaton}) that results when it is applied to a range of familiar psychic powers, but The Argument from Relatives (168b2–169c2) does. It is the alleged reflexivity of \emph{sōphrosunē}, its sovereign prerogative, alone of all the knowledges, to be of itself that is, if not impossible, then at least raises serious doubts (168e). More generally, what is impossible is for a power to be of something to be applied to itself.

Socrates inaugurates the further inquiry with a statement of a general principle:
\begin{quote}
	\textsc{Socrates}: Well now, knowledge as such [{\sbl αὐτη}] is knowledge of something and has some power to be of something—hasn't it? (\emph{Charmides} 168b2–3)
\end{quote}
\citet{Burnet:1903aa} prefers {\sbl αὒτη} and \citet[60]{Lamb:1927qw} follows him in this. If accepted, the claim would be about the special knowledge that constitutes \emph{sōphrosunē}, that this ({\sbl αὒτη}) knowledge has the power to be of something. \citet{Shorey:1907ys}, \citet[55]{Ben:1985aa}, and \citet[220 n23]{Tuozzo:2011aa}, by contrast, prefer {\sbl αὐτη}. If accepted, the claim would not be about the special knowledge that constitutes \emph{sōphrosunē} but about knowledge generally, that knowledge as such ({\sbl αὐτη}) has the power to be of something. Since the application of this general principle is not restricted to the special knowledge that constitutes \emph{sōphrosunē}, I follow Shorey's emendation.

The power to be of something is general in a further way.

First, though, notice that the power to take an object is expressed in the Greek with a genitive construction (\emph{tinos}). But this genitive construction has broader application in Greek than in English. It includes not only the power of knowledge to be of something, but the power of the greater to be greater than the smaller. Notice that, in English, the power of a magnitude is expressed not with ``of'' but with ``than''. To accomodate this, we shall speak, admittedly awkwardly, of the power to be of or than something. Thus intentional psychic powers are a subset of things that are or have the power to be of or than something.

After discussion of some quantitative cases (magnitudes, numbers and the like 168e5–7), this general principle is elaborated upon. Things that have the power to be of or than something are only of or than things with a certain nature or being (\emph{ousia}). So whatever has its own power applied to itself must itself have the relevant nature or being:
\begin{quotation}
	\textsc{Socrates}: So whatever has its own power applied to itself will also have the being (\emph{ousia}) to which its power was applicable, will it not? For instance, hearing is, as we say, just a hearing of sounds, is it not?
	
	\textsc{Critias}: Yes. (\emph{Charmides}, 168d1–5; \citealt[63]{Lamb:1927qw}, modified)
\end{quotation}

On this basis, an argument against reflexivity seems only to require one further step: The denial that the power, or the activity that it gives rises to, has the characteristic nature or being of what the power is of or than. Consider one of Socrates' own examples. Being colored is the characteristic nature or being of the objects of sight. Neither sight nor seeing are colored. So sight could not be of sight or seeing.

The Argument from Relatives (168b2–169c2) is developed over two stages. After the first general principle is stated—that some things have the power to be of or than something (\emph{tinos})—quantitative cases of this kind are considered on the basis of which the general principle is elaborated—things that have the power to be of or than something are only of or than things with a certain nature or being (\emph{ousia}) (168b2–d3). After the general principle is elaborated, both are applied to perceptual powers, hearing and seeing (168d3–169c2). These are, of course, intentional psychic powers. And at the conclusion of the argument, Socrates mentions, in addition, certain natural powers—the power to move something, the power to burn something, and the like.

Let a relative be something which possesses the power to be of or than something (\emph{tinos}). (The terminology may be Peripatetic but the discussion of relatives in \emph{Categoriae} and \emph{Metaphysica} {\sbl Δ} are clearly inspired by the \emph{Charmides}.) And let a relative power be a power the possession of which makes something a relative, namely a power to be of or than something. 

The relative powers discussed in The Argument from Relatives fall into three groups:
\begin{enumerate}[(1)]
	\item \textsc{The Quantitative}:
	\begin{enumerate}
		\item the power to be greater than the smaller (168b5–c3)
		\item the power to be double the half (168c4–8)
		\item the power to be more than the less (168c9)
		\item the power to be heavier than the lighter (168c9–10)
		\item the power to be older than the younger (168c10)
	\end{enumerate}
	\item \textsc{The Intentional}:
	\begin{enumerate}
		\item the power to hear sound (168d3–8)
		\item the power to see color (168d9–e2)
	\end{enumerate}
	\item \textsc{The Natural}:
	\begin{enumerate}
		\item the power to move something (168e9–10)
		\item the power to burn something (168e10)
	\end{enumerate}
\end{enumerate}
And the conclusion of The Argument from Relatives will be that it is impossible for these relative powers to be reflexive—that it is impossible for the power to be of or than something to apply either to itself or to its exercise.

There is an issue about how to understand talk of \emph{dunamis} here (\emph{Metaphysica} {\sbl Δ} 12). Aristotle (\emph{Metaphysica} {\sbl Θ} 1 46a4–11) at least sees a difference in sense between quantitative and natural powers. The former, such as geometrical powers, are powers only homonymously, spoken of as powers on the basis of mere resemblance. Natural powers, by contrast, though diverse, are non-homonymously powers since they are all spoken of with reference to an origin of change (\emph{arkhē metabolēs}) in another thing or in itself \emph{qua} other. 

Without accepting Aristotle's account, this raises an issue about how to understand intentional powers, such as the power of knowledge to be of something. Should the intentional powers be assimilated to the quantitative powers and so be understood as powers homonymously, on the basis of a mere resemblance? (\citealt[24 n74]{Moore:2019aa} offer such an interpretation: ```Power'\ldots means only an essential relative property.''). Or should intentional powers be understood as powers non-homonymously, bracketing Aristotle's understanding of what this amounts to? 

An independent if related issue is raised as well. How could considerations that pertain to homonymous powers pertain, as well, to non-homonymous powers? The power to be greater than the smaller is an homonymous power. The power to move something is a non-homonymous power. How could considerations against the possibility of the reflexivity of the former also be considerations against the possibility of the reflexivity of the latter?

Though independent this issue is related in the following way. One may be tempted to assimilate intentional powers, such as the power of knowledge to be of something, to geometrical powers, powers only in name, because of some resemblance to genuine powers, thus Moore and Raymond's \citeyearpar[24 n74]{Moore:2019aa} identification of powers to be of or than something with relational properties. Such an interpretation has the virtue of imposing a coherent uniformity among the relevant powers. Unfortunately, natural powers do not fit the pattern, being genuine powers as opposed to powers in name only, and yet The Argument from Relatives is meant to at the very least raise serious doubts if not indeed establish the impossibility of reflexive natural powers such as motion moving itself (\emph{kinēsis autē heautēn kinein} 168e9–10).

\subsection{The Quantitative} % (fold)
\label{sub:the_quantitative}

Having observed that knowledge has the power to be of something, Socrates next observes that the power to be of or than something arises as well with magnitudes, numbers, and the like. Thus the greater has the power to be greater than something (168b5–6), and the thing the greater is greater than is the smaller (168b8). Socrates then applies Critias' account of \emph{sōphrosunē} to these powers. He seems to be pursuing two related goals. First, to establish that the result of applying Critias' account to these powers is impossible. But, second, to establish that whatever has its own power applied to itself will also have the nature or being to which this power is applicable. Notice that achieving the first goal does not advance the discussion much—Socrates has already established analogous impossibilities. It is rather the elaboration of the general principle that things are only of or than something with a certain nature or power that begins to make the case against the application of Critias' account to intentional psychic powers.

Socrates begins with the power to be greater than the smaller:
\begin{quotation}
	\textsc{Socrates}: So if we could find a greater which is greater than other greater things, and than itself, but not greater than the things beside which the others are greater, I take it there can be no doubt that it would be in the situation of being, if greater than itself, at the same time smaller than itself, would it not?
	
	\textsc{Critias}: Most inevitably, Socrates. (\emph{Charmides} 168b10–c3, \citealt[61]{Lamb:1927qw})
\end{quotation}
The hypothetical power to be greater than the smaller has three of the four features of Critias' account of the content of \emph{sōphrosunē}. Another difference is that he reverts to Critias' original ordering. Just as Critias originally spoke of knowledge of knowledge and itself, Socrates now invites him to consider a greater thing that is greater than other greater things and greater than itself. Specifically, Critias is asked to consider whether there is a greater thing that has the power to be greater than:
\begin{enumerate}[(1)]
	\item \textsc{Higher-Order}: other greater things
	\item \textsc{Reflexive}: itself
	\item \textsc{Exclusive}: and no other thing, or at least, no other thing that the other greater things are greater than.
\end{enumerate}
What is missing is an analogue of Socrates insistence that the knowledge of knowledge be, at the same time, a knowledge of the lack of knowledge. 

Why has absence absented itself in the discussion? Recall that this Socratic refinement was motivated by a general conception of knowledge as a power to discriminate a thing from its opposite. But that motive is not applicable in the case of the power to be greater than the smaller since it is not similarly a power to discriminate a thing from its opposite.

With respect to Socrates first goal, to establish the impossibility that results from applying Critias' account to the power of the greater to be greater than the smaller, there are ample means of achieving it. One means sufficient to establish this impossibility, salient to moderns, is the failure of transitivity. The greater is greater than other greater things without also being greater than the things that the other greater things are greater than. However, Socrates' emphasis seems elsewhere. Socrates does not emphasize the failure of transitivity. He rather emphasizes that if the greater has the power to be greater than itself, then it must be smaller than itself. This too is a means sufficient to establish this impossibility. But the real focus is on the fact that if the power to be greater than applies to a thing it must be a certain way, in this case, smaller. And so if that power applies to itself, it too must be that way and so have the appropriate nature or being (\emph{ousia}), in this case being smaller. Socrates is beginning to lead Critias to accept the general claim that whatever has its own power applied to itself will also have the nature or being to which its power is applicable (168d1–3). 

In the second quantitative case, the power to double the half, the failure of transitivity is not even mentioned, the focus exclusively being on the way something must be in order for the power of double to apply to it, namely being half:
\begin{quotation}
	\textsc{Socrates}: Or again, if there is a double of other doubles and of itself, both it and the others must of course be halves, if it is to be there double; for, you know, a double cannot be ``of'' anything else than half.
	
	\textsc{Critias}: True (\emph{Charmides} 16bc4–8; \citealt[61–3]{Lamb:1927qw})
\end{quotation}
\textsc{Exclusive} is dropped and only \textsc{Higher-Order} and \textsc{Reflexive} are retained. The double is double of other doubles and itself. And again the focus is on the way the other doubles and itself must be in order for the power of double to apply to them, namely by being half. The power of doubling only applies to things that are half. In order for the power to apply to something it must have a certain nature or being.

Doubling is intransitive. A double of other doubles is not double of what other doubles double (4 is double 2, 2 is double 1, but 4 is not double 1, it is quadruple). The intransitivity of a double of other doubles and itself is not a means to establish its impossibility. The only available means to establish the impossibility is that the double, being double itself, is also, at the same time, half. But then it is not the failure of transitivity so much as reflexivity that is the source of the impossibility.

The remaining quantitative powers are enumerated emphasizing the way something must be in order for the power to apply. What is more than itself must be less than itself, the heavier will be lighter, the older younger (168c9–10). It is at this point that the general conclusion is drawn: whatever has its own power applied to itself will have the nature or being to which this power is applicable (168d1–5). Though stated thus, the more general principle would be that something must have a certain nature or being in order for a power to apply to it. Notice its generality. If something must have a certain nature or being in order for a specific power to apply to it, then if that power applies to itself or its exercise, these too must have the relevant nature or being. It is this more general principle that has implications for the reflexive case that is doing the work.

% subsection the_quantitative (end)

\subsection{The Intentional} % (fold)
\label{sub:the_intentional}

After the elaboration of the general principle—whatever has its own power applied to itself will have the nature or being to which this power is applicable (168d1–5), it is applied to perceptual powers. There are three differences between the present discussion of perceptual powers and the discussion of perceptual powers in the puzzling disanalogies. First, the ordering differs. Audition now occurs as the initial case, with vision following. Second, there is no analogue of the senses taken altogether as a whole (167d7–10). Socrates considers only hearing and then seeing. Third, only the reflexive aspect of the content of \emph{sōphrosunē}, as Critias understands it, is considered. It is the reflexive application of these perceptual powers, the sovereign prerogative, that is if not impossible then at least subject to serious doubt.

Incorporating the first general principle (168b2–3) along with its elaboration (168d1–5), the puzzle about reflexive perceptual powers can be abstractly put as follows:
\begin{enumerate}[(1)]
	\item Perception is of something.
	\item In order for the perception to be of something it must have a certain nature or being.
	\item So if perception is reflexive and is applied to itself, it must have the requisite nature or being.
	\item But perception lacks the requisite nature or being.
	\item So perception does not perceive itself.
\end{enumerate} 

The first perceptual case is audition and not vision:
\begin{quotation}
	\textsc{Socrates}: For instance, hearing is, as we say, just a hearing of sound, is it not?
	
	\textsc{Critias}: Yes.
	
	\textsc{Socrates}: So if it is to hear itself, it will hear a sound of its own; for it would not hear otherwise.
	
	\textsc{Critias}: Most inevitably. (\emph{Charmides} 168d3–8; \citealt[63]{Lamb:1927qw})
\end{quotation}
The elaborated general principle—whatever has its own power applied to itself will have the nature or being to which this power is applicable (168d1–5)—is applied to audition and the consequence of this for reflexive hearing is worked out.

The reasoning may be elaborated as follows. Audition, like knowledge, is intentional. There is something heard just as there is something known. In order for something to be heard, it must have a certain nature or being (\emph{ousia}), it must have sound. So hearing hears only sound. But if hearing hears only sound, then if it hears itself, hearing must have a distinctive sound by means of which it is heard. Critias accedes to all this without the denial that would secure the impossibility of reflexive hearing being made explicit—that hearing lacks a sound of its own by which it may be heard.

That hearing hears only sounds, while having many venerable adherents, may be doubted. Perhaps we hear sounds and their sources. \citet{Heidegger:1935uq} goes so far as to claim that we hear only sources, though perhaps he was being hyperbolic (for discussion see \citealt[chapters 3 and 4]{Kalderon:2018oe}). Denying that we hear only sounds only accomplishes so much. Consistent with that denial, what is heard may yet have a certain nature or being. The denial only effects the identification of this nature or being with having sound. The nature or being may not be sound, but would still need to be exemplified by the hearing that hears itself. Reflexivity would remain vulnerable.

The second perceptual case is vision and not audition:
\begin{quotation}
	\textsc{Socrates}: And sight, I suppose, my excellent friend, if it is to see itself, must needs have a color; for sight can never see what is colorless.
	
	\textsc{Critias}: No more it can. (\emph{Charmides} 168d9–e2; \citealt[63]{Lamb:1927qw})
\end{quotation}
The focus is on reflexive vision, on what it is for sight to see itself. Something must have a certain nature or being to be seen. It must be colored. So for sight or seeing to see itself, it must itself be colored, for sight can never see what is colorless. And again, Critias accedes to all this without the denial that would secure the impossibility of reflexive vision being made explicit—that neither sight nor seeing are colored.

And again, that sight sees only colors, while having many venerable adherents, may be doubted. But, again, denying that we see only color only accomplishes so much. Consistent with that denial, what is seen may yet have a certain nature or being. The denial only effects the identification of this nature or being with being colored. The nature or being may not be color, but would still need to be exemplified by the sight or seeing that sees itself. Reflexivity would remain vulnerable.

Earlier we wondered whether intentional powers were more like the quantitative powers, and so powers only homonymously, or more like the natural powers, and so powers non-homonymously. A subtle shift in vocabulary may speak in favor of the latter. Socrates introduces the quantitative powers by observing that knowledge has the power to be of something. Perhaps the power to be of or than something is open to be understood homonymously, on the mere resemblance to genuine powers. Perhaps it is open to be understood as merely a relational property \citep[24 n74]{Moore:2019aa}. But the discussion of reflexive audition and vision does not use that vocabulary. We could, of course, speak of the power of hearing to be of something to parallel the power of knowledge that inaugurates the discussion of quantitative powers. But that vocabulary has been quietly dropped. Socrates does not speak of hearing possessing the power to be of something. He rather merely speaks of hearing something. But hearing is a natural power, as is sight. They may be animate natural powers, and so psychic, but they remain natural powers nonetheless, involving as they do corporeal instruments (\emph{Alcibiades} I 129c2–130a2). But if they are natural powers, then they are powers non-homonymously.

If the power to be of or than something is a power only homonymously, then how could it be used to establish a principle that is then applied to non-homonymous powers? Recall that the goal of the discussion of quantitative powers was to establish the principle that in order for a power to apply to itself it must have the nature or being to which the power is applicable. Even if that discussion does establish that principle, why think that it applies to non-homonymous powers such as hearing and sight? Though the worry is genuine, its effect is limited. We need only consider whether the principle that the power is only applicable to things with a certain nature or being holds of hearing and sight. And if it plausibly does, then the argument against reflexive hearing and sight goes through.

A remaining worry is the following. The explicit target of The Argument from Relatives is the reflexive character of intentional psychic powers. But how can we distinguish the failure of reflexivity from the failure of transitivity? Transitivity or transparency fails when there is a seeing that sees itself and other seeings but not what other seeings see, namely color. This can encourage the thought that a seeing that sees no color is not possible. But reflexivity allegedly fails because sight and seeing, being without color, are not possibly seen. But what is the difference here? If the seeing of sight and seeing is not possible, then sight and seeing are unseen. And if it is not possible for sight and seeing to be seen, then there is no seeing of sight and seeing. Again these can seem like equivalent descriptions. But if they are, then the emphasis on reflexivity in the second stage of The Argument from Relatives is a mere accident of presentation.

But is it?

Recall that the impossibility of the double of other doubles and itself is not due to the failure of doubling to be transitive but to its reflexivity requiring that the double be at the same time the half. The failure of transitivity is possible in the case of doubling. Indeed it is actual. Again, 4 is double 2, 2 is double 1, but 4 is not double 1, but quadruple. But reflexivity is not possible. So in this case at least, the failure of reflexivity is not due to the failure of transitivity. That means that the failure of reflexivity and transitivity are not equivalent descriptions.

They may not be equivalent descriptions generally but be equivalent in a restricted domain. This logical possibility raises the question whether they are equivalent when restricted to perceptual powers as they seem to be, or at least can be made to so seem.

A concessive Peripatetic response would be the following. The response is concessive in that it concedes that there is a sense in which the failures of transitivity and reflexivity are equivalent. And the response is Peripatetic both in that there remains, nonetheless, another sense in which they are not equivalent, but also in the specification of the senses in which they are and are not equivalent. The actuality of not seeing sight and seeing and sight and seeing being unseen might be the same, but they might differ in being. So they are equivalent in being actual but remain non-equivalent in being.

% subsection the_intentional (end)

\subsection{The Natural} % (fold)
\label{sub:the_natural}

% subsection the_natural (end)

% section the_argument_from_relatives (end)

\section{The Text Relates Itself to Itself} % (fold)
\label{sec:the_text_relates_itself_to_itself}

% section the_text_relates_itself_to_itself (end)

\section{A Great Man Wanted} % (fold)
\label{sec:a_great_man_wanted}

% section a_great_man_wanted (end)

% Chapter offering (end)