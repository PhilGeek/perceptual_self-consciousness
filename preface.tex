%!TEX root = /Users/markelikalderon/Documents/Git/perceptual_self-consciousness/perceptual_self-consciousness.tex
\chapter*{Preface} % (fold)
\markboth{\MakeUppercase{Preface}}{}
\addcontentsline{toc}{chapter}{Preface}
\label{cha:preface}

Walking through an ancient woodlands, I descend a steep slope and come upon clearing bringing into view the ruined remains of The Abbey of St Mary and St Thomas the Martyr at Lesnes. Founded by Richard de Lucy in 1178, perhaps in penance for his role in the death of Thomas Becket, the Archbishop of Canterbury, it is listed in the Domesday Book of 1086. The community of Lesnes Abbey drained the marshlands, now southeastern suburbs of London, and were responsible for maintaining the southern bank of the Thames nearby. The abbey was closed by Cardinal Wolsey in 1525 as part of a larger program to abolish smaller communities. Moving from the woodlands to the manicured lawn of the clearing, the abbey is disclosed under an overcast sky. Though gray and low hanging, the clouds are warmly luminous. The scene remains, nevertheless, gloomy and claustrophobic. Lesnes Abbey is in a ruined state, lacking a roof and many of the walls having been pulled downed, though the outline of the courtyard can still be seen. 

My attention is drawn to architectural details still discernable: An arched doorway on the Western side of the courtyard, a lancet window (narrow, pointed, and arched), the base, all that remains, of a column. The remains of the pulpit are little more than a heap of stones, and the crypt, containing the heart of the great great granddaughter of Richard de Lucy, Roesia of Dover, is mercilessly exposed to the elements.

I approach the ruined column. It stands a little over a meter high. There is a main column whose base is set upon a larger base. Cascading behind the main column are series of lesser columns. The remains of the column proper are a warm hue, while its base and that which it is set upon are blackened. The structure is before me. In perceiving the structure, it is set before my mind, as Hume would say. My conscious attention is directed upon the structure, a material body distinct from my consciousness. The ruined column may be the object of my visual experience, but in undergoing a visual experience that affords me an awareness of the column, I am aware from within of my seeing that column. While I may be absorbed in the ruined column, in closely observing it, there remains a sense in which I am aware from within of my perceiving it. In perceiving the structure, I undergo a kind of perceptual self-consciousness, but I am not set before this perceptual consciousness in the way that the ruined column is. Nor is my perceptual activity. The seeing of the ruined column is less seen that lived. The intrusion of the Sartrean vocabulary is to highlight the first-personal aspect of the reflexive awareness afforded by consciously seeing the remains of the column. If the seeing were seen, the perceiver would not be aware of the seeing from within, but, rather, it would be set before the perceiver, and though the seeing is their seeing, it would be experienced \emph{qua} other. If the seeing were seen rather than lived, we would stand in an alienated spectatorial relation to our own perceptual activity in a way that we manifestly do not. I may later reflect and explicitly self-attribute the seeing of the column, and in so doing me and my perceptual activity may be the object of my reflection, but even prior to reflection I enjoyed a mode of self-awareness in seeing what I saw. 

It is this mode of perceptual self-consciousness that shall be our quarry throughout the present essay. We shall try to come to understand this mode of perceptual self-consciousness, in the first instance, by examining ancient \emph{aporiai} about reflexive powers, among which reflexive perceptual powers figure prominently. In considering such puzzles, we attend to the phenomena, consider difficulties as to its nature, and, with the phenomena in view, reason to a resolution of these difficulties. If succesful at least a partial gain in understanding is achieved.

At the center of Plato's \emph{Charmides} is a puzzle. That puzzle concerns a mode of reflexive being. Its target is described in such general terms to highlight the potential proleptic nature of this dialogue. Whereas \citet{Kahn:1988aa} interprets the \emph{Charmides} as proleptically anticipating themes discussed in \emph{Res Publica}, it might equally be interpreted as proleptically anticipating the puzzles about reflexive being, such as the self-predicating nature of the Forms, raised in the \emph{Parmenides}. In the \emph{Charmides}, the reflexive mode of being pertains to certain psychic powers such as \emph{sōphrosunē}, self-knowledge, and perceptual self-consciousness. \emph{Sōphrosunē} is the notoriously untranslatable term for a virtue central to the self-conception of Athenian aristocrats, not least those who were admirer's of Sparta. Critias, an aristocratic admirer of Sparta, proposes at certain point that \emph{sōphrosunē} might simply be a kind of self-knowledge. And in discussing whether the relevant kind of self-knowledge is possible, the possibility of analogous kinds of reflexive perceptual powers are discussed. These reflexive perceptual powers are hypothetical analogues introduced by Socrates to emphasize the strangeness of the reflexive epistemic power with which Critias identifies \emph{sōphrosunē}. These hypothetical perceptual powers might reasonably be interpreted as, or at least as involving, capacities for perceptual self-consciousness. 

Though a reasonable interpretation it remains a substantive one. There may be grounds to recommend alternatives. However, my philosophical interest, throughout this essay, concerns the nature of perceptual self-consciousness. In attempting to understand the nature of perceptual self-consciousness, at least a partial advance can be made by successfully resolving certain \emph{aporiai} concerning it. It is in that spirit that I am approaching the \emph{Charmides}. For as long as the hypothetical reflexive perceptual powers may reasonably be interpreted as capacities for perceptual self-consciousness, then a challenge remains for anyone who seeks to understand perceptual self-consciousness.

That is not to say that I am uninterested in exegesis. By no means. Indeed, I shall argue, throughout this present essay, that we can undertand at least some aspects of the nature of perceptual self-consciousness by coming to an understanding of traditional texts concerning concerning that subject matter. That is the sense in which the present investigation is aptly described as a hermeneutic phenomenology. 

Allow me to address, briefly, all too briefly, two objections to the foregoing. My aim is less to quell dissent than to plead for further hearing.

First, one might wonder why focus on ancient texts when a lot of water has flowed under the bridge since? Relatedly, why consult texts from antiquity, as opposed to neuroscience or any other relevant empirical discipline?

There is a lot to be said, but let these preliminary remarks suffice for now. First, we, all of us, enjoy, or suffer if you must, perceptual self-consciousness. So too Plato. So reflection on the problems raised in understanding perceptual self-conscious by a perceptually self-conscious agent is worth considering, not least if raised by a thinker of the stature of Plato. Second, though a lot of water has flowed under the bridge, perhaps not all of it is potable. At any rate, the reader is not being asked to consider ancient accounts of perceptual self-consciousness so much as ancient puzzles as to its nature. Third, neuroscience and other relavant empirical disciplines answer questions against a philosophical background, however implicit (consider the philosophical background behind Anil Seth's claim that perception is a mode of veridical hallucination, this could only be a seventeenth century inheritance). If your questions are not their questions, or if you do not share their philosophical background assumptions, then you must look elsewhere. And \emph{aporiai} raised by respected predecessors is a good place to start.

A Lutheran objection, from one long grown impatient, ``What the right hand gives, the left hand takes away.'' On the one hand, you say that you want to understand perceptual self-consciousness by coming to an understanding of traditional texts concerning that subject matter, but on the other hand, you also say that the puzzle about reflexive perceptual powers is a puzzle about perceptual self-consciousness on an interpretation for which their may be grounds to recommend an alternative. Are you serious about exegesis or not? The implied sense that I am not is driven by the assumption that a traditional text admits of a uniquely true interpretation. But is that really plausible? We tend to think that great works of art or literature are great, in part, because they are hermeneutically fecund, because they admit of endless interpretative possibilities. Moreover, these diverse interpretations potentially yield diverse insights. Might not something similar be true of traditional philosophical texts? If we reflect, say, on the diversity of opinion on display in the commentary tradition on \emph{De anima}, is this best understood as a history of hermeneutic blunders? Or is it rather a history of diverse insights into the subject matter of that treatise yielded by thinkers who came to an understanding of that text, given their own aims, historical background, and philosophical and scientific assumptions? This might be especially true of Platonic texts whose aporetic character often seems like a deliberate provocation for the reader to come to an understanding of its subject matter for themselves. An attitude perhaps reflected in his pedagogy, for he seems to have raised independent thinkers rather than dogmatic adherents to Platonic doctrine.

We shall revisit these methodological issues as we proceed.

% Chapter preface (end)